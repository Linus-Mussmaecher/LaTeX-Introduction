% --------
% Packages
% --------

\usepackage[
	ngerman,
	strings
]{babel}						% Languages

% ===      ===
% === Math ===
% ===      ===

\usepackage{amsmath,amssymb}    % more math symbols
\usepackage{mathtools}          % more math symbols
\usepackage{dsfont}             % double stroked math symbols
\usepackage{amsfonts}           % blackboard math symbols
\usepackage{latexsym}		    % more math symbols
\usepackage{chngcntr}		    % more math symbols
\usepackage{mathrsfs}		    % math-fonts
\usepackage{cancel}             % cancel lines for fractions
\usepackage[luatex]{nchairx}    % additional packages
\usepackage{exscale}            % large summation signs in 11pt
\usepackage{leftindex}          % for indeces to the left of symbols
\usepackage{mleftright}         % better spacing for left and right

% ===          ===
% === Graphics ===
% ===          ===

\usepackage{graphicx}           % Include images
\usepackage{tikz}               % for commutative diagrams and stuff
\usetikzlibrary{patterns}
\usetikzlibrary{cd}
\usepackage[backend=biber,backref=true]{biblatex}

% ===         ===
% === Figures ===
% ===         ===

\usepackage{floatrow}           % Centered Figures etc.
\usepackage{subfig}             % Separate captioning of multiple figures in one environment
\usepackage[margin    =20pt,
            font      ={footnotesize},
            format    =plain,
            labelsep  =period
]{caption}          % Nicer looking captions

% ===                 ===
% === Fonts & Symbols ===
% ===                 ===

% Font related packages
\usepackage[T1]{fontenc}   % Allows usage of custom fonts and improves encoding
% \usepackage[sc,osf]{mathpazo}   % Custom palatino font with smallcaps and old style figures
% \usepackage[scaled=.8]{beramono}% Custom monospace font
\usepackage{fontspec}             % Allows using custom fonts
\usepackage[
	% Turn of certain warnings when using mathtools with unicode-math
	warnings-off={mathtools-colon,mathtools-overbracket},
	% Inform unicode-math we are using an upright math font (euler math)
	math-style=upright
]{unicode-math}                   % Allows input and output of math as unicode characters.
% Font customization
\defaultfontfeatures{Ligatures={TeX}}
\setmainfont{Rosario}
\setmainfont{TeX Gyre Termes}
\setmathfont{Euler Math}[Scale=MatchLowercase]
\setmonofont{JetBrains Mono}[Scale=MatchLowercase]
% Color
\usepackage[dvipsnames]{xcolor} % Colors
\definecolor{codegreen}{rgb}{0,0.6,0}
\definecolor{codegray}{rgb}{0.5,0.5,0.5}
\definecolor{codepurple}{rgb}{0.58,0,0.82}
\definecolor{linkblue}{RGB}{58, 103, 181}
\definecolor{backcolour}{rgb}{0.95,0.95,0.92}
\definecolor{foldercolor}{RGB}{124,166,198}
% Symbols
\usepackage{csquotes}           % Prettier quote signs
\usepackage{ellipsis}           % Fixes spacing of ellipsis
\usepackage{fontspec}           % German special characters with luatex. Needs to be after mathpazo.
\usepackage[
	colorlinks=true,
	linkcolor=black,
	urlcolor=linkblue,
	final=true,                 % always treat as final
	pdfpagelabels,              % use pdf page labels
]{hyperref}                     % hyperrefs are cool!
\usepackage[
	nameinlink
]{cleveref}                     % Nicer automatic references
\usepackage{xurl}               % Better link inclusions

% ===                        ===
% === Technical & Formatting ===
% ===                        ===

\usepackage[footsepline, draft=false]{scrlayer-scrpage}
\usepackage[babel]{microtype}   % Nicer text blocks
\DeclareMicrotypeAlias{Euler Math}{TU-basic} % With unicode math, math is rendered in TU font encoding, where Euler math does not support some of the symbols.
\usepackage{xspace}             % better spacing after macros
\usepackage{marginnote}         % Used to put text into the margins.
\usepackage{gitinfo2}           % show info about the git
\usepackage{ifdraft}            % to determine whether draft mode
\usepackage{titling}            % allows use of thetitle
\usepackage[cleanlook]{isodate} % Prettier date
\usepackage{array,booktabs}     % Improved table formatting options
\newcolumntype{L}[1]{>{\raggedright}p{#1}}
\usepackage{enumitem}           % Better formatting for enumerations
\setlist{nosep}
\setlist[enumerate,1]{label=(\roman*)}
\setlist[enumerate,2]{label=(\alph*)}
\usepackage{silence}           % Disable certain warnings
\WarningFilter{latex}{Unused label}
\WarningFilter{latex}{Unused entry}
\WarningFilter{latex}{Marginpar on page}
\WarningFilter{hypdoc}{hyperref}

% ===                ===
% ===      Code      ===
% ===                ===

\usepackage{listings}		% Allow code blocks in text
\usepackage{lstautogobble}  % Gobbles leading whitespace
\lstnewenvironment{latexlisting}{
\lstset{
    language=[LaTeX]TeX,
    backgroundcolor=\color{backcolour},   
    commentstyle=\color{codegreen},
    keywordstyle=\color{codepurple},
    numberstyle=\tiny\color{codegray},
    stringstyle=\color{codepurple},
    basicstyle=\ttfamily\footnotesize,
	columns=fullflexible,
    breakatwhitespace=false,         
    breaklines=true,                 
    keepspaces=true,                 
    numbers=none,       
    numbersep=5pt,                  
    showspaces=false,                
    showstringspaces=false,
    showtabs=false,                  
    tabsize=4,
    autogobble,
}
}{}
\lstset{
    backgroundcolor=\color{backcolour},   
    commentstyle=\color{codegreen},
    keywordstyle=\color{codepurple},
    numberstyle=\tiny\color{codegray},
    stringstyle=\color{codepurple},
    basicstyle=\ttfamily\footnotesize,
	columns=fullflexible,
    breakatwhitespace=false,         
    breaklines=true,                 
    keepspaces=true,                 
    numbers=none,       
    numbersep=5pt,                  
    showspaces=false,                
    showstringspaces=false,
    showtabs=false,                  
    tabsize=4,
    autogobble,
}
% ===                ===
% === Other Packages ===
% ===                ===

\usepackage{blindtext}          % Simple blindtext for typography testing
\usepackage{fixme}
\FXRegisterAuthor{lm}{alm}{Linus}
\fxsetup{theme=colorsig}

% Folderpaths
\usepackage[edges]{forest}
\tikzset{pics/folder/.style={code={%
    \node[inner sep=0pt, minimum size=#1](-foldericon){};
    \node[folder style, inner sep=0pt, minimum width=0.3*#1, minimum height=0.6*#1, above right, xshift=0.05*#1] at (-foldericon.west){};
    \node[folder style, inner sep=0pt, minimum size=#1] at (-foldericon.center){};}
    },
    pics/folder/.default={20pt},
    folder style/.style={draw=foldercolor!80!black,top color=foldercolor!40,bottom color=foldercolor}
}

\forestset{is file/.style={edge path'/.expanded={%
        ([xshift=\forestregister{folder indent}]!u.parent anchor) |- (.child anchor)},
        inner sep=1pt},
    this folder size/.style={edge path'/.expanded={%
        ([xshift=\forestregister{folder indent}]!u.parent anchor) |- (.child anchor) pic[solid]{folder=#1}}, inner xsep=0.6*#1},
    folder tree indent/.style={before computing xy={l=#1}},
    folder icons/.style={folder, this folder size=#1, folder tree indent=3*#1},
    folder icons/.default={12pt},
}

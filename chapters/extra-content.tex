\chapter{Zusätzliche Inhalte}
In diesem Kapitel beschäftigen wir uns damit, wie wir zusätzliche Inhalte (keinen Fließtext) in unsere Dokumente einbauen können.
Insbesondere lernen wir dabei sogenannte \emph{floats} kennen, Objekte, deren Position nicht unbedingt mit ihrer Code-Position übereinstimmen und die frei über die Seite \enquote{fließen} können.

\section{Bilder}
\LaTeX -Dokumente können Bilder in einer Vielzahl an Formaten einbinden, insbesondere \filetype{png}, \filetype{jpg} und \filetype{jpeg}.
Hierzu verwenden wir das beliebte Package \latexpackage{graphicx}.
\begin{latexlisting}
	% Präambel
	\usepackage{graphicx}
\end{latexlisting}
Dann kann im Text der Befehl \latexcommand{includegraphics} benutzt werden, der an der entsprechenden Stelle ein Bild \enquote{inline}, d.h. wie ein (besonders großes) Wort einfügt:
\begin{latexlisting}
	Hier ist ein Bild:
	\includegraphics{my-image.png}
\end{latexlisting}
Dies funktioniert natürlich nur, wenn das Bild \filepath{my-image.png} auch existiert und im selben Order liegt wie das zentrale File \filepath{main.tex}.
Die Dateiendung \filetype{png} kann auch weggelassen werden:
\begin{latexlisting}
	Hier ist ein Bild:
	\includegraphics{my-image}
\end{latexlisting}
Dies hat Vor- und Nachteile:
\begin{itemize}
	\item \textbf{Vorteil}: Soll das Bild ausgetauscht werden, und ändert sich dabei der Filetype, so muss der Code nicht verändert werden.
	Außerdem muss man beim Schreiben des Codes nur den Filenamen kennen, nicht die genaue Erweiterung.
	\item \textbf{Nachteil}: Liegen im Ordner mehrere Dateien mit gleichem Namen, aber unterschiedlicher Erweiterung, so ist nicht immer ganz klar, welche davon von \LaTeX{} bevorzugt wird.
	Außerdem benötigt das Einfügen des Bildes so etwas mehr Rechenzeit, auch wenn dies in modernen Zeiten kaum noch relevant ist.
\end{itemize}
Weiterhin wird aufgefallen sein, dass das Bild immer in Originalgröße eingefügt wird.
Um das zu beheben, können wir die optionalen Argument von \latexcommand{includegraphics} verwenden:
\begin{latexlisting}
	Bilder:

	\includegraphics[scale=0.1]{my-image}

	\includegraphics[height=2cm]{my-image}

	\includegraphics[width=0.8\textwidth]{my-image}
\end{latexlisting}
Das Argument \latexargument{scale} skaliert das Bild um einen festen Faktor und ist eher selten nützlich.
Mit \latexargument{height} und \latexargument{width} kann dagegen die Höhe oder Breite des Bildes auf einen festen gewünschten Wert gesetzt werden.
Hierzu können auch vordefinierte kontextuelle Breiten wie \latexcommand{textwidth} oder Vielfache derer verwendet werden.
Wird nur eines dieser Argumente verwendet, wird die andere Dimension automatisch skaliert, um das Format des Bildes nicht zu verändern.
Werden beide verwendet, wird das Bild eventuell verzerrt.

Es ist in \LaTeX{} üblich, verwendete Bilder nicht direkt mit dem \filepath{main.tex}-File in einem Ordner zu speichern, sondern in einem getrennten Unterordner, der häufig \filepath{graphics} heißt.
Werden besonders viele Bilder verwendet, kann es sich auch lohnen, diesen Ordner weiter zu untergliedern.
Auf das Bild kann dann durch Angabe des vollständigen, relativen Pfades zugegriffen werden.
\begin{latexlisting}
	\includegraphics{graphics/my-image.png}
	\includegraphics{graphics/chapter1/my-image.png}
\end{latexlisting}
Das \latexpackage{graphicx}-Package erlaubt die Angabe eines Grafikpfades (also z.B. der Ordner \filepath{graphics}).
Wird dieser spezifiziert, so werden Pfade als relativ zum Grafikpfad gelesen statt relativ zum Hauptordner.
Das obige Beispiel sähe dann aus wie folgt:
\begin{latexlisting}
	% Präambel
	\graphicspath{{graphics}}
	% Text
	\includegraphics{my-image.png}
	\includegraphics{chapter1/my-image.png}
\end{latexlisting}
Man beachte, dass der Pfad in doppelte Klammern gesetzt werden muss.

\section{Positionierung von Bildern: Floats, Labels, Refs}
Wie bereits erwähnt wird das gewünschte Bild \enquote{inline} eingefügt, also ohne Zeilenumbrüche oder Rand.
Dies ist gewollt, denn es erlaubt uns Bilder zur Textdekoration zu verwenden, etwa wie folgt:
\begin{latexlisting}
	In der \emph{Sedanschlacht} am 01.09.1870 besiegten Truppen des Norddeutschen Bundes unter \includegraphics{prussian-flag} Helmut Moltke die zahlenmäßig unterlegenen französischen Streitkräfte der Generäle \includegraphics{french-flag} Patrice de Mac-Mahon und \includegraphics{french-flag} Auguste-Alexandre Ducrot.
\end{latexlisting}
Häufig ist das aber nicht die gewünschte Positionierung von Bildern.
Ein erste Verbesserung ist es, den \latexcommand{includegraphics}-Befehl mit Leerzeilen abzutrennen,
\begin{latexlisting}
	Nach dem Aufmarsch am 31. August standen sich die beiden Armeen wie folgt gegenüber:

	\includegraphics{aufmarschskizze}

	Aus dieser Position heraus begann am 01.09. um 4:00 Uhr morgens die Schlacht.
\end{latexlisting}
Hier ist das Bild allerdings noch immer linksbündig.
Um es zu zentrieren, bietet sich naiv die \latexenvironment{center}-Umgebung an:
\begin{latexlisting}
	Nach dem Aufmarsch am 31. August standen sich die beiden Armeen wie folgt gegenüber:

	\begin{center}
		\includegraphics{aufmarschskizze}		
	\end{center}

	Aus dieser Position heraus begann am 01.09. um 4:00 Uhr morgens die Schlacht.
\end{latexlisting}

\subsection{Figures \& Captions}

Das obige Vorgehen ist allerdings generell nicht empfehlenswert, das \LaTeX{} ein viel stärkeres, praktischeres Tool zur Positionierung von Bildern bereitstellt: Floats.
Beim Betrachten von professionell gesetzten Büchern -- oder auch Wikipedia-Artikeln -- fällt schnell auf, dass sich Bildern selten direkt an der Stelle im Fließtext befinden, an der sie referenziert werden, sondern stattdessen an einer passenden Stelle neben (über, unter) dem Text.
Sie \enquote{fließen} also über die Seite, weswegen man solche Objekte, die nicht \emph{im}, sondern \emph{neben} dem Text erscheinen, im Englischen als \emph{floats} bezeichnet.
In \LaTeX{} können wir dies durch die \latexenvironment{figure}-Umgebung umsetzen:
\begin{latexlisting}
	Nach dem Aufmarsch am 31. August standen sich die beiden Armeen wie folgt gegenüber:

	\begin{figure}
		\includegraphics{aufmarschskizze}		
	\end{figure}

	Aus dieser Position heraus begann am 01.09. um 4:00 Uhr morgens die Schlacht.
\end{latexlisting}
Wir bemerken, dass das Bild nun eben nicht genau zwischen den beiden Absätzen erscheint, sondern oben auf der Seite.
Wir sollten also entsprechend umformulieren:
\begin{latexlisting}
	Nach dem Aufmarsch am 31. August standen sich die beiden Armeen wie im beiligenden Bild aufgeführt gegenüber und begannen aus dieser Position heraus am 01.09. um 4:00 Uhr morgens die Schlacht.

	\begin{figure}
		\includegraphics{aufmarschskizze}		
	\end{figure}
\end{latexlisting}
Außerdem ist das Bild -- erneut -- linksbündig statt zentiert.
Dieser Mangel der \latexenvironment{figure}-Umgebung lässt sich mittels der Package \latexpackage{floatrow} beheben:
\begin{latexlisting}
	% Präambel
	\usepackage{floatrow}
\end{latexlisting}
Für gewöhnlich ist es üblich, solche beiligenden Bilder (engl. \emph{figures}, deutsch \emph{Abbildung}) mit einer Bildunterschrift (engl. \emph{caption}) zu versehen.
\begin{latexlisting}
	Nach dem Aufmarsch am 31. August standen sich die beiden Armeen wie im beiligenden Bild aufgeführt gegenüber und begannen aus dieser Position heraus am 01.09. um 4:00 Uhr morgens die Schlacht.

	\begin{figure}
		\includegraphics{aufmarschskizze}		
		\caption{Aufmarschskizze vom 31.08.1870}
	\end{figure}
\end{latexlisting}
Dies führt auch dazu, dass unsere Abbildung automatisch nummeriert wird.
Soll dies in der passenden Sprache geschehen, so ist -- wie so oft -- die \latexpackage{babel}-Package unersetzlich, wobei der verwendete Begriff notfalls auch manuell gesetzt werden kann.
Weiterhin erscheint die Abbildung damit im Abbildungsverzeichnis, das wie folgt angezeigt werden kann:
\begin{latexlisting}
	\listoffigures
\end{latexlisting}
Ist die übergebene Bildunterschrift zu lang, um sinnvoll im Verzeichnis angezeigt werden zu können, kann als optionales Argument eine Kurzversion übergeben werden:
\begin{latexlisting}
	Nach dem Aufmarsch am 31. August standen sich die beiden Armeen wie im beiligenden Bild aufgeführt gegenüber und begannen aus dieser Position heraus am 01.09. um 4:00 Uhr morgens die Schlacht.

	\begin{figure}
		\includegraphics{aufmarschskizze}		
		\caption[Aufmarschskizze]{Aufmarschskizze vom 31.08.1870}
	\end{figure}
\end{latexlisting}

\subsection{Positionierung}
Die genaue Positionierung von Floats ist kompliziert -- der gesamte Algorithmus lässt sich unter \url{https://www.latex-project.org/publications/2014-FMi-TUB-tb111mitt-float-placement.pdf} nachlesen.
Mit einem optionalen Argument von \latexenvironment{figure} können wir eine Positionierungspräferenz angeben.
\begin{latexlisting}
	Nach dem Aufmarsch am 31. August standen sich die beiden Armeen wie im beiligenden Bild aufgeführt gegenüber und begannen aus dieser Position heraus am 01.09. um 4:00 Uhr morgens die Schlacht.

	\begin{figure}[b]
		\includegraphics{aufmarschskizze}		
		\caption{Aufmarschskizze vom 31.08.1870}
	\end{figure}
\end{latexlisting}
Die verschiedenen Optionen sind in \autoref{tab:float-options} aufgeführt, die mit der \latexargument{b}-Position gesetzt wurde.
\begin{table}[b]
	\begin{tabular}{l p{8cm}}
		\toprule
		\textbf{Option} & \textbf{Position}\tabularnewline
		\midrule
		\latexargument{t} &
		Oberer Seitenrand
		\tabularnewline
		\latexargument{b} &
		Unterer Seitenrand
		\tabularnewline
		\latexargument{h} &
		\emph{Here}, also genau an der Position zwischen Absätzen, an der sich der Code befindet.
		\tabularnewline
		\latexargument{p} &
		Auf einer eigenen Seite (evtl. mit anderen Floats, um die Seite zu füllen)
		\tabularnewline
		\latexargument{!} &
		Erzwingt die Positionierung.
		\tabularnewline
		\bottomrule
	\end{tabular}
	\caption{Positionierungsoptionen von Floats}
	\label{tab:float-options}
\end{table}
Insbesondere die Option \latexargument{!h} ist sehr beliebt, denn sie erreicht genau die Positionierung, die der naive Typograph sich oft intuitiv wünscht.
\begin{latexlisting}
	Nach dem Aufmarsch am 31. August standen sich die beiden Armeen wie folgt gegenüber:

	\begin{figure}[!h]
		\includegraphics{aufmarschskizze}
		\caption{Aufmarschskizze vom 31.08.1870}
	\end{figure}

	Aus dieser Position heraus begann am 01.09. um 4:00 Uhr morgens die Schlacht.
\end{latexlisting}
Es sei allerding an dieser Stelle erwähnt, dass dies seltener optimal ist, als man denkt.

\subsection{Subfloats}
Manchmal ist es wünschenswert, mehrere Bilder in einer Abbildung zusammenzufassen.
Dies kann mit der Package \latexpackage{subfig} und dem Befehl \latexenvironment{subfloat} umgesetzt werden:
\begin{latexlisting}
	\begin{figure}
		\subfloat[Helmut Moltke]{
			\includegraphics[height=5cm]{moltke}
		}
		\hfil
		\subfloat[Friedrich Willhelm]{
			\includegraphics[height=5cm]{friedrich-willhelm}
		}
		\caption{Deutsche Oberbefehlshaber}
	\end{figure}
\end{latexlisting}
Der \latexcommand{hfil}-Befehl dient hierbei einer gleichmäßigen horizontalen Verteilung der Subabbildungen.

\subsection{Labels \& Refs}
Wenn das gewünschte Bild nun aber wild über die Seite fließt, ist ein Hinweis wie \enquote{das beiligende Bild} oft nur begrenzt nützlich.
Praktischer wäre ein Beweis wie \enquote{Bild 4.2}.
Wir können diesen natürlich einfach manuell in den Text schreiben, aber was wenn wir die Reihenfolge der Bilder verändern, oder weiter vorne eine neue Abbildung einfügen?
Hierfür bietet sich die Verwendung von \emph{Labels} und \emph{Refs} (\emph{Referenzen})  an.
\begin{latexlisting}
	Nach dem Aufmarsch am 31. August standen sich die beiden Armeen wie in Abbildung \ref{fig:aufmarschskizze} aufgeführt gegenüber und begannen aus dieser Position heraus am 01.09. um 4:00 Uhr morgens die Schlacht.

	\begin{figure}
		\includegraphics{aufmarschskizze}		
		\caption{Aufmarschskizze vom 31.08.1870}
		\label{fig:aufmarschskizze}
	\end{figure}
\end{latexlisting}
Der Befehl \latexcommand{label} versieht dabei die \latexenvironment{figure}-Umgebung mit einem nutzergegebenen, eindeutigen Label.
Dieses kann -- an jeder Stelle im Dokument, auch davor -- durch Verwendung des \latexcommand{ref}-Befehls referenziert werden und \LaTeX{} fügt dann automatich die korrekte Zahl ein.
Wird die Package \latexpackage{hyperref} verwendet, wird diese Zahl sogar mit der Position des Bildes verlinkt.
Der Befehl \latexcommand{ref} hat -- solange das Package \latexpackage{hyperref} eingebunden ist -- einige Varianten, wichtig sind \latexcommand{pageref} (zeigt nicht die Nummer des Ziels an, sondern die Seite und ist nützlich für Beweise über lange Strecken hinweg) sowie \latexcommand{autoref} (zeigt zusätzlich noch den Typ des Ziels an).
\begin{latexlisting}
	Am zweiten Tag der Schlacht hatte sich die ursprüngliche Aufstellung (siehe \autoref{fig:aufmarschskizze}, Seite \pageref{fig:aufmarschskizze}) deutlich verändert.
\end{latexlisting}
Nicht alle Objekte funktionieren out-of-the-box reibungslos mit \latexcommand{autoref}, und teilweise ist hier etwas manuelle Anpassung notwendig, die über das Ziel dieses Skripts hinaus geht.

Nicht nur Abbildungen und andere Floats können gelabelt (und folglich referenziert) werden, sondern auch eine Vielzahl anderer \LaTeX{}-Objekte: Subfloat, Kapitel, (Unter-)Abschnitte, Gleichungen (siehe später), mathematische Umgebungen wie Sätze oder Definitionen (siehe später), Tabellen und viele mehr. 

Das Label kann beliebige gewählt werden, üblich und empfehlenswert ist aber die folgende Formatierung wie oben:
Das Label beginnt mit einer kurzen Beschreibung seines Typs (\latexargument{fig}, \latexargument{sec}, \latexargument{def}, \dots) gefolgt von einem Doppelpunkt und dem tatsächlichen Namen.
Dieser sollte dabei niemals inhaltslos (\latexargument{kapitel3bild}) oder gar eine Nummer sein (\latexargument{kapitel3bild5}) sein, da dies dem Zweck eines Labels und einer automatischen Nummerierung widerspräche.
Leerzeichen in Labeln sind zulässig, aber ungewöhnlich.
Stattdessen werden für gewöhnlich Striche verwendet, wie in \latexargument{fig:general-von-der-tann}.
Sollen Label \enquote{kategorisiert} werden, so werden oft Doppelpunkte oder Schrägstriche als Trennzeichen verwendet, beispelsweise \latexargument{fig:chap3:general-de-wimpffen} oder \latexargument{fig:chap3/general-de-wimpffen}.


\section{Tabellen}
Neben Bildern sind Tabellen eine weitere Klasse an Nicht-Fließtext-Objekten, die wir häufig in unsere Dokumente einbinden wollen (man beachte die zahlreichen Tabellen in diesem Dokument).
In \LaTeX{} geschieht dies mittels der Umgebung \latexenvironment{tabular}:
\begin{latexlisting}
	\begin{tabular}{r l l}
		& Deutscher Bund & Frankreich \\
		Truppen & 200.000 & 130.000 \\
		Kanonen & 774 & 564
	\end{tabular}
\end{latexlisting}
Zur Verwendung:
\begin{itemize}
	\item Als Argument wird das Spaltenlayout der Tabelle übergeben.
	Jeder Buchstabe steht hierbei für eine Spalte, die möglichen Optionen sind in \autoref{tab:column-options} aufgelistet.
	\item Eine neue Spalte wird mit \latexargument{\&} begonnen.
	\item Eine neue Zeile wird mit \latexcommand{\textbackslash} begonnen -- dies ist der einzige Fall, in dem dieses Command verwendet werden sollte.
	Ist die korrekte Anzahl an Spalten noch nicht erreicht, wird ein Fehler geworfen.
\end{itemize}

\begin{table}
	\begin{tabular}{l p{8cm}}
		\toprule
		\textbf{Spaltenoption} & \textbf{Wirkung}\tabularnewline
		\midrule
		\latexargument{l} &
		Linksbündig
		\tabularnewline
		\latexargument{r} &
		Rechtsbündig
		\tabularnewline
		\latexargument{c} &
		Zentriert
		\tabularnewline
		\latexargument{p\{<Breite>\}} &
		Absatz mit gewisser Breite, hier werden auch automatisch Zeilenumbrüche eingefügt.
		\tabularnewline
		\bottomrule
	\end{tabular}
	\caption{Layoutoptionen für Spalten}
	\label{tab:column-options}
\end{table}
Ähnlich wie Bilder ist es üblich, Tabellen in eine Float-Umgebung zu setzen -- die entsprechende Umgebung heißt hier \latexenvironment{table}:
\begin{latexlisting}
	\begin{table}
		\begin{tabular}{r l l}
			& Deutscher Bund & Frankreich \\
			Truppen & 200.000 & 130.000 \\
			Kanonen & 774 & 564
		\end{tabular}
		\caption{Truppenstärken der Sedanschlacht}
		\label{tab:truppenstaerken-sedanschlacht}
	\end{table}
\end{latexlisting}
Eine Liste aller so semantisch markierten Tabellen lässt sich mit
\begin{latexlisting}
	\listoftables
\end{latexlisting}
drucken.

\subsection{Randlinien}
Unserer erzeugten Tabellen sehen bisher noch ein wenig nackt aus.
In \LaTeX{} können sowohl senkrechte als auch waagrechte Trennlinien eingefügt werden.
Obwohl beide Linientypen in \LaTeX{} out-of-the-box möglich sind, empfiehlt sich die Verwendung der Package \latexpackage{booktabs}, die auch einige Spacingprobleme der Standardimplementierung behebt.
\begin{latexlisting}
	\begin{tabular}{r l l}
		\toprule
		& Deutscher Bund & Frankreich \\ \midrule
		Truppen & 200.000 & 130.000 \\ 
		Kanonen & 774 & 564 \\ \bottomrule
	\end{tabular}
\end{latexlisting}
Man beachte die Semantik -- \latexcommand{toprule} und \latexcommand{bottomrule} schließen die Tabelle von oben und unten ab, während \latexcommand{midrule} zur Trennung von Zeilen verwendet wird.
Weiterhin existiert noch das Command \latexcommand{cmidrule}, mit dem Teile von Trennlinien gezogen werden können, etwa
\begin{latexlisting}
	\begin{tabular}{r l l}
		\toprule
		& Deutscher Bund & Frankreich \\ \midrule
		Truppen & 200.000 & 130.000 \\ \cmidrule{2-3}
		Kanonen & 774 & 564 \\ \bottomrule
	\end{tabular}
\end{latexlisting}
Es ist in \LaTeX{} üblich, die verwendeten Linien an das Ende der jeweiligen Zeile zu setzen.

\subsection{Verbotene Linien}
Neben diesen vier Linientypen gibt es noch weitere Möglichkeiten, Trennlinien in Tabellen zu erzeugen.
Zum einen können wir durch Wiederholung der Befehle doppelte Linien nutzen:
\begin{latexlisting}
	\begin{tabular}{r l l}
		\toprule
		& Deutscher Bund & Frankreich \\ \midrule \midrule
		Truppen & 200.000 & 130.000 \\ \cmidrule{2-3}
		Kanonen & 774 & 564 \\ \bottomrule
	\end{tabular}
\end{latexlisting}
Zum anderen haben wir auch die Möglichkeit, in der Layoutspezifikation vertikale Linien zu erzeugen, doppelt oder einfach:
\begin{latexlisting}
	\begin{tabular}{|r||l|l}
		\toprule
		& Deutscher Bund & Frankreich \\ \midrule \midrule
		Truppen & 200.000 & 130.000 \\ \cmidrule{2-3}
		Kanonen & 774 & 564 \\ \bottomrule
	\end{tabular}
\end{latexlisting}
Diese vertikalen Linien sehen mit den Horizontalen der \latexpackage{booktabs}-Package etwas gebrochen aus (aus Gründen, die uns gleich noch klar werden), daher sollte hier das Command \latexcommand{hline} verwendet werden, mit dem in Plain-\LaTeX{} Trennlinien erzeugt werden:
\begin{latexlisting}
	\begin{tabular}{|r||l|l}
		\hline
		& Deutscher Bund & Frankreich \\ \hline \hline
		Truppen & 200.000 & 130.000 \\ \hline
		Kanonen & 774 & 564 \\ \hline
	\end{tabular}
\end{latexlisting}
Dies sollte aber im allgemeinen nicht nötig sein, denn für typografisch gut gesetzte Tabellen gelten die folgenden drei Regeln:
\begin{enumerate}
	\item Verwende niemals vertikale Trennlinien.
	\item Verwende niemals doppelte Trennlinien.
	\item Zeilen, die gleiche Daten enthalten, sollten generell nicht getrennt werden.
\end{enumerate}
Unsere initielle Tabelle
\begin{latexlisting}
	\begin{tabular}{r l l}
		\toprule
		& Deutscher Bund & Frankreich \\ \midrule
		Truppen & 200.000 & 130.000 \\ 
		Kanonen & 774 & 564 \\ \bottomrule
	\end{tabular}
\end{latexlisting}
war also bereits das typografische Optimum.

\subsection{Mehrfachzellen}
Wollen wir mehrere Zellen verschmelzen, können wir die \latexpackage{multirow}-Package verwenden:
\begin{latexlisting}
	% Präambel
	\usepackage{multirow}
\end{latexlisting}
Diese erlaubt die Verwendung der Commands \latexcommand{multirow} und \latexcommand{multicolumn}:
\begin{latexlisting}
	\begin{table}
		\begin{tabular}{l r c c c}
			\toprule
			& & \multicolumn{2}{c}{Deutsche Koalition} & Frankreich \\  \cmidrule{3-4}
			& & Bayern & Andere & \\ \midrule
			\multirow{2}{2cm}{Tote} & Soldaten & 973 & 2.049 & 3.000 \\ 
			& Offiziere & 106 & 84 & Unbekannt \\ 
			\multirow{2}{2cm}{Verwundete} & Soldaten & 3.218 & 2.691 & 14.000 \\ 
			& Offiziere & 107 & 175 & Unbekannt \\ 
			\bottomrule
		\end{tabular}
		\caption{Tote und Verwundete}
		\label{tab:verlust-sedanschlacht}
	\end{table}
\end{latexlisting}
Zu beachten ist, dass dem Command \latexcommand{multicolumn} neben der Spaltenzahl übergeben werden muss, welche Ausrichtung der enthaltene Text haben soll.
Der Inhalt von \latexcommand{multirow} hingegen wird immer als \latexargument{p} interpretiert, mit einer Breite die ebenfalls übergeben wird.
Soll dieser Text daher zentiert oder rechtsbündig werden, muss ein Command wie \latexcommand{raggedright} verwendet werden.

\subsection{Fortgeschrittene Tabellen}
Einige formatierungstechnische Fragen bleiben vielleicht noch offen -- wollen wir beispielsweise die erste Spalte oder Zeile zur Hervorhebung fettdrucken, können wir natürlich jeder Zelle ein \latexcommand{textbf} mitgeben, dies erfüllt unser grundlegendes Prinzip, Inhalt und Formatierung trennen zu wollen, nur begrenzt.
Für solche Zwecke existieren zahlreiche Hilfspackages wie beispielsweise \latexpackage{easytable} oder \latexpackage{tabulararray}, die diese und viele weitere mächtige Optionen und Anpassungsmöglichkeiten bereitstellen, an dieser Stelle aber nicht näher diskutiert werden.

\section{Links}

\section{Code}

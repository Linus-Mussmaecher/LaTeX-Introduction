\chapter{TikZ}

Die Package \emph{tikz} erlaubt in \LaTeX{} die Verwendung der Sprache \emph{TikZ} (kurz für \emph{TikZ ist kein Zeichenprogramm}), mit der sich Vektorgraphiken beschreiben lassen.
TikZ wurde speziell für die Verwendung mit \LaTeX{} entwickelt und erlaubt die Erstellung von zahlreichen Vektorgraphiken: Bilder, Diagramme, Plots, Symbole, Grundrisse, etc.
Eine vollständige Spezifikation und viele nützliche Hinweise lassen sich unter \url{tikz.dev} nachlesen.

Ähnlich wie \LaTeX{} selbst kann TikZ durch viele Libraries erweitert werden.
Dies geschieht mittels des Commands \latexcommand{usetikzlibrary}.
Libraries können klein sein und nur neue Commands hinzufügen, bestehenden Commands mehr Optionen geben (z.B. die Library \latexpackage{patters}, die Musterfüllungen erlaubt) oder gar ganz neue Umgebungen und syntaktische Konstrukte einführen (z.B. die Library \latexpackage{cd}, die zum Erstellen kommutativer Diagramme verwendet wird).

Da TikZ-Code oft etwas mühsam zu schreiben ist, existieren Online-Lösungen wie \url{q.uiver.app} oder \url{tikzmaker.com}, die den User in einer graphischen Umgebungen Diagramme zeichnen und dann generierten Code in \LaTeX{} einfügen lassen.

Die allgemeine Umgebung zur Verwendung von TikZ ist \latexenvironment{tikzpicture}.
Wir nehmen im Folgenden an, dass sich alle Beispiele in diesem Kapitel -- sofern nicht anders erwähnt -- in einer solchen \latexenvironment{tikzpicture}-Umgebung befinden.

\section{Pfeile \& Formen}

\subsection{Linien}
Die grundlegendste in TikZ enthaltene Form ist eine einfache Linie.
\begin{latexlisting}
	\draw (0,0) -- (1,0);
\end{latexlisting}
Das Command \latexcommand{draw} zeichnet die darauffolgende Form, hier ein Strich von \latexargument{(0,0)} nach \latexargument{(1,0)}.
Koordinaten werden hier im mathematischen Sinne verwendet, d.h. die positiven Richtungen sind oben und rechts.
Da es sich um europäisches Package handelt, sind alle Angaben -- außer anders konfiguriert -- in \latexargument{cm}.
Die Form wird durch ein Semikolon (\key{;}) abgeschlossen.
Eine Form kann auch eine Folge an Linien sein.
\begin{latexlisting}
	\draw (0,0) -- (1,0) -- (0.5,0.5) -- (0,0);
\end{latexlisting}
Soll eine Folge an Linien geschlossen werden, kann auch der \enquote{Punkt} \latexargument{cycle} verwendet werden.
\begin{latexlisting}
	\draw (0,0) -- (1,0) -- (0.5,0.5) -- cycle;
\end{latexlisting}
Ähnliche wie viele Strukturen in \LaTeX{} können wir Linien mit optionalen Argumenten ausstatten.
Statt \latexargument{--} muss dann aber \latexargument{to} verwendet werden.
\begin{latexlisting}
	\draw (0,0) to[bend left] (1,0) -- (0.5,0.5) to[bend right,distance=0.1cm] cycle;
\end{latexlisting}
So können wir auch die Stärke und Art der Linie anpassen:
\begin{latexlisting}
	\draw[thick] (0,6) to (1,6);
	\draw[ulta thin] (0,5) to (1,5);
	\draw[color=myownblue] (0,4) to (1,4);
	\draw[color=red] (0,3) to (1,3);
	\draw[dashed] (0,2) to (1,2);
	\draw[dotted] (0,1) to (1,1);
	\draw (0,0) to (1,0);
\end{latexlisting}
Wie wir sehen, können erneut auch durch \latexpackage{xcolor} selbst definierte Farben verwendet werden.
Die Standard-Liniendicke ist \latexargument{thin}.
Argument können direkt an \latexargument{to} geschrieben werden, oder an \latexcommand{draw}, oder an das gesamte \latexenvironment{tikzpicture}.
\begin{latexlisting}
	\begin{tikzpicture}
		\draw (0,0) to[dashed] (1,0) -- (0.5,0.5) -- cycle;
		\draw[dashed] (0,2) to (1,2) -- (0.5,2.5) -- cycle;
	\end{tikzpicture}

	\begin{tikzpicture}[dashed]
		\draw (0,0) to (1,0) -- (0.5,0.5) -- cycle;
		\draw (0,2) to (1,2) -- (0.5,2.5) -- cycle;
	\end{tikzpicture}
\end{latexlisting}
Wichtig sind auch die optionalen Argumente \latexargument{in} und \latexargument{out}, mit der für krumme Linien die Ein- und Ausgangswinkel angepasst werden können.
\begin{latexlisting}
	\draw[out=30,in=150] (0,0) to (3,0);
	\draw[out=30,in=150] (0,0) to (6,0);
	\draw[out=30,in=150] (0,0) to (9,0);
\end{latexlisting}
Schließlich existiert noch die Möglichkeit, Linien mit Pfeilspitzen zu versehen.
\begin{latexlisting}
	\draw[->] (0,0) to (3,0);
	\draw[|->] (0,0) to (2,0);
	\draw[<->] (0,0) to (1,0);
\end{latexlisting}
Für eine vollständige Liste aller optionalen Argument konsultiere man \url{tikz.dev}.

\subsection{Formen}
Auch wenn sich die meisten Formen aus Linien und Kurven zusammensetzen lassen, ist es doch oft praktischer, einige Formen direkt zu zeichnen.
Mit \latexargument{rectangle} beispielsweise lässt sich ein Rechteck zwischen zwei Ecken zeichnen.
\begin{latexlisting}
	\draw (0,0) rectangle (3,2);
\end{latexlisting}
Verwandt dist \latexargument{grid}, das ein ganzes Linienraster zeichnet -- dies ist besonders praktisch, um Hilfslinien einzufügen und sich im Bild zu orientieren. Ist das Bild fertig, kann die Zeile dann auskommentiert werden.
\begin{latexlisting}
	\draw (0,0) grid[help lines, step=0.5cm] (8,5);
\end{latexlisting}
Die \latexargument{help lines}-Option stellt hier die Liniendicke auf das Minimum ein, \latexargument{step} konfiguriert das Intervall zwischen den Rasterlinien -- wird es nicht angegeben, ist der Standard das auf häufigsten nützliche \latexargument{1cm}.
Mit Linien etwas schwerer nachzubilden sind Kreise und Ellipsen.
Hier ist das zweite \enquote{Argument} kein Punkt, sondern der Radius bzw. die Halbachsen.
\begin{latexlisting}
	\draw (0,0) circle (1cm);
	\draw (0,0) ellipse (5cm and 3cm);
\end{latexlisting}
Für Funktionen existiert schließlich noch die \latexargument{plot}-Funktion.
\begin{latexlisting}
	\draw[<->] (0,2.2) to (0,0) to (4,0);
	\draw[thin,blue] plot[domain=0:3.8] (\x,{2/(1+pow(\x,2))});
\end{latexlisting}
Beim Plotten ist zu beachten, dass die Laufvariable \latexcommand{x} einen Backslash benötigt, aber arithmetische Operationen wie \latexargument{pow} nicht (genauso wie bei \latexpackage{fpeval}).
Der zu evaluierende Ausdruck muss außerdem in geschweiften Klammern stehen.
Mit \latexargument{domain} wird der Defintionsbereich angepasst.

\subsection{Füllung}
Für die letztgenannten Plots ist es eher selten nützlich, aber für alle anderen Formen geschieht es oft, dass sie nicht nur aus Linien bestehen sondern auch gefüllt sein sollen.
Hierzu existiert die Alternative \latexcommand{fill} zu \latexcommand{draw}.
\begin{latexlisting}
	\fill (0,0) rectangle (3,2);
\end{latexlisting}
Auch hier kann die Farbe angepasst werden:
\begin{latexlisting}
	\fill[color=red] (0,0) rectangle (3,2);
\end{latexlisting}
Wir können \latexcommand{fill} und \latexcommand{draw} beliebig kombinieren.
Die vier folgenden Zeilen erzeugen dasselbe Rechteck.
\begin{latexlisting}
	\fill[color=red, draw=black] (0,9) rectangle (3,11);
	\draw[fill=red, color=black] (0,6) rectangle (3,8);
	\fill[fill=red, draw=black] (0,3) rectangle (3,5);
	\draw[fill=red, draw=black] (0,0) rectangle (3,2);
\end{latexlisting}
Man beachte, dass \latexcommand{draw} die Liniendicke zur Formgröße hinzufügt.
Die beiden folgenden Zeilen erzeugen also \emph{nicht} dasselbe Rechteck.
\begin{latexlisting}
	\fill[draw=black] (0,0) rectangle (0,2);
	\fill (4,0) rectangle (4,2);
\end{latexlisting}
Dies sieht man besser mit stärkerer Linienstärke.
\begin{latexlisting}
	\fill[draw=black,ulta thick] (0,0) rectangle (0,2);
	\fill (4,0) rectangle (4,2);
\end{latexlisting}
Wem diese Füllungen etwas zu stark sind, der kann sie halbtransparent machen.
\begin{latexlisting}
	\fill[red,opacity=0.5] (360:0.5cm) circle (0.6cm);
	\fill[blue,opacity=0.5] (120:0.5cm) circle (0.6cm);
	\fill[green,opacity=0.5] (240:0.5cm) circle (0.6cm);
\end{latexlisting}
Dieses Beispiel demonstriert auch Polarkoordinaten, für die wir in diesem Skript leider keinen Raum mehr haben.
Ebenfalls nicht erklärt sind \emph{transparency groups}, die verwendet werden können, um komplexere Formen einheitlich transparent zu machen.
Siehe für beides \url{tikz.dev}.
Als Alternative zur Transparenz sind auch Musterfüllungen mit der \latexpackage{patterns}-Library möglich.
\begin{latexlisting}
	\fill[pattern=dots,pattern color=blue] (0,0) rectangle (1,1);
	\fill[draw=black, pattern=north west lines, pattern color=green] (2,0) -- (2.577,1) -- (3.154,0) -- cycle;
\end{latexlisting}
Die verfügbaren Patterns lassen sich im Internet nachlesen.

\section{Nodes}
Möchte man seine Diagramm beschriften, verwendet man dazu in TikZ sogenannte \emph{nodes}.
\begin{latexlisting}
	\draw[|-|] (0,0) -- (2,0):
	\node at (1,0.2) {\qty{2}{cm}};
\end{latexlisting}
Der gewünschte Text wird innerhalb der hinteren geschweiften Klammern platziert.
Hier sind alle gewöhnlichen \LaTeX{}-Commands und Konstrukte, insbesondere die Verwendung des Mathematik-Modus, erlaubt.
Mit den optionalen Argumenten \latexargument{above}, \latexargument{below}, \latexargument{left}, \latexargument{right} und Kombinationen können Nodes genauer positioniert werden.
\begin{latexlisting}
	\fill (0,0) circle (0.5mm);
	\fill (2,0) circle (0.5mm);
	\draw (0,0) -- (2,0);
	\node[left] at (0,0) {$P_1$};
	\node[above right] at (2,0) {$P_2$};
\end{latexlisting}
Längere Texte können mit \latexcommand{\textbackslash} gebrochen werden.
In diesem Fall ist die Angabe eines \latexargument{align}-Parameters (\latexargument{left},\latexargument{center},\latexargument{right}) nötig.
\begin{latexlisting}
	\fill (0,0) circle (0.5mm);
	\fill (2,0) circle (0.5mm);
	\draw (0,0) -- (2,0);
	\node[left] at (0,0) {$P_1$};
	\node[below,align=center] at (2,0) {Dieser Punkt ist\\nicht $P_2$.};
\end{latexlisting}
Nodes können auch wie alle anderen Formen gefüllt oder umrandet werden.
\begin{latexlisting}
	\fill (0,0) circle (0.5mm);
	\fill (2,0) circle (0.5mm);
	\draw (0,0) -- (2,0);
	\node[left, draw=black] at (0,0) {$P_1$};
\end{latexlisting}

\section{Ausblick}
Die hier aufgeführten Features sind nur eine kleine Teilmenge von all dem, was in TikZ möglich ist.
Auch wenn dieses Skript keinen Raum für eine vollständige Ausführung hat, nennen wir im Folgenden einige dieser Features -- bei Interesse kann man sich dann unter \url{tikz.dev} oder allgemein im Internet weiter informieren.
\begin{itemize}
	\item \emph{Scopes} erlauben die Gruppierung von Formen, gemeinsame Verschiebung, Anpassung und Teilen von optionalen Argumenten.
	\item \emph{Clips} erlauben das nur teilweise Zeichnen von Formen, z.B. wenn der Schnitt zwischen zwei Formen eingefärbt werden soll.
	\item \emph{Coordinates} erlauben das Wiederverwenden von Punkten.
	\item Neben dem klassischen kartesischen Koordinatensystem erlaubt TikZ auch die Verwendung von Polarkoordinaten, xyz-Koordinaten, node-Koordinaten, baryzentrischen Koordinaten und relativen Koordinaten.
	\item \emph{Kommutative Diagramme} können mit \latexenvironment{tikz-cd} erstellt werden.
	\item \emph{Fill Rules} konfigurieren die Füllung von Formen.
	\item \emph{Transparency Groups} machen auch komplexe, mehrteilige Formen konsistent transparent.
	\item Mit \latexpackage{pgf} können auch 3-dimensionale Objekte geplottet werden.
\end{itemize}

\chapter{Unser erstes Dokument}
Nun, da die Installation abgeschlossen ist, können wir unser erstes \LaTeX{}-File erstellen.

Im Gegensatz zu traditionellen Textverarbeitungsprogrammen besteht ein \LaTeX{}-Projekt meist nicht aus einem einzigen File, sondern wird jeweils in einem eigenen Ordner verwaltet.
Wir erstellen daher an einem angemessenen Ort, z.B. \texttt{Dokumente/LaTeX-Seminar/}, einen Ordner für unser erstes Dokument und in diesem Order legen wir ein neues File an.
Bei angemessener Benennung sollte der Pfad diese Files dann \texttt{Dokumente/LaTeX-Seminar/my-first-document/my-first-document.tex} sein.
Dieses File können wir nun mit dem Editor unserer Wahl, z.B.~TeXStudio, öffnen und mit dem folgenden Text füllen:
\begin{latexlisting}
	\documentclass{article}

	\begin{document}
		Hallo, LaTeX!
	\end{document}
\end{latexlisting}
Drücken wir nun in TeXStudio auf den grünen \enquote{Kompilieren}-Pfeil, so wird unser erstes LaTeX-Dokument kompiliert und wir sollten ein PDF mit dem einsamen Satz \enquote{Hallo, LaTeX!} vor uns sehen.
Wenn wir jetzt in den vorhin erstellten Ordner sehen, so ist unser \filetype{tex}-File nicht mehr alleine.
\LaTeX{} hat nicht nur ein \filetype{pdf}-File erstellt und im Ordner abgelegt, sondern auch eine Vielzahl andere Hilfs- und Cache-Files, die zukünftige Kompilierungen beschleunigen und für verschiedenste Features von TeXStudio benötigt werden.
Insbesondere kann \LaTeX{} bei der Kompilierung nicht \enquote{in die Zukunft schauen}, d.h. wenn das Inhaltsverzeichnis kompiliert wird, weiß \LaTeX{} noch gar nicht, welche Kapitel etc. überhaupt im Dokument vorkommen.
Daher werden Kapitel in das \filetype{toc}-File geschrieben, damit zukünftige Kompilierung ein sinnvolles Inhaltsverzeichnis erzeugen können.
Die meisten \LaTeX{}-Dokumente müssen deswegen mehrmals kompiliert werden, glücklicherweise übernimmt diesen Schritt TeXStudio für uns.
Diese Nebenprodukte jedes Kompilierungsvorgangs sind einer der Gründe, warum jedes \LaTeX{}-Projekt generell einen eigenen Ordner verdient.

\section{Nebenprodukte \& Dokumentstruktur}
Wir betrachten in \autoref{tab:out-files} kurz einige der wichtigen Filetypen, die beim Kompilieren entstehen und in \autoref{tab:source-files} einige Filetypen, die beim Kompilieren als Quelldokumente verwendet werden und deren Einbindung wir in zukünftigen Kapiteln behandeln werden.

Aufgrund dieser Vielzahl an Files ist es wie bereits erwähnt wichtig, jedes \LaTeX{}-Projekt in einem eigenen Ordner zu verwalten.
Es kann sich auch lohnen, ein Projekt in mehrere \filetype{tex}-Files aufzuteilen, wie wir in späteren Kapiteln noch lernen werden.

\begin{table}
	\begin{tabular}{l p{10cm}}
		\toprule
		\textbf{Filetyp} & \textbf{Beschreibung} \tabularnewline
		\midrule
		\filetype{aux} &
		Hilfsfile, das Informationen speichert, die zukünftigen Kompilierungen die Nutzung von Referenzen und Zitationen ermöglicht. \tabularnewline
		\filetype{out} &
		Hilfsfile, das von \latexpackage{hyperref} verwendet wird, um PDF-Lesezeichen zu erzeugen. \tabularnewline
		\filetype{log} &
		Logfile, das die Kompilierung protokolliert und auch eventuell aufgetretene Fehler oder Warnungen enthält.
		Erlaubt TeXStudio, eine Zusammenfassung dieser Fehler anzuzeigen. \tabularnewline
		\filetype{synctex} &
		File, das es TeXStudio ermöglicht, den Nutzer von Code zu PDF und umgekehrt springen zu lassen.
		\tabularnewline
		\filetype{toc} &
		File, das Informationen über Kapitel und Abschnitte enthält, damit zukünftige Kompilierungen ein Inhaltsverzeichnis erzeugen können. \tabularnewline
		\filetype{fls}, \filetype{bbl}, \filetype{bcl} &
		Files für verschiedenste Packages. \tabularnewline
		\bottomrule
	\end{tabular}
	\caption{Nebenprodukte einer Kompilierung}
	\label{tab:out-files}
\end{table}

\begin{table}
	\begin{tabular}{l p{10cm}}
		\toprule
		\textbf{Filetyp} & \textbf{Beschreibung} \tabularnewline
		\midrule
		\filetype{tex} & File, dass \LaTeX{}-Quellcode enthält. \tabularnewline
		\filetype{bib} & Bibliographie-File, in dem Quellen gelistet sind. \tabularnewline
		\filetype{ttf} & Schriftart-File \tabularnewline
		\filetype{sty} & Package-File für ein eigenes, lokale Package. \tabularnewline
		\filetype{cls} & Klassen-File für eine eigene Dokumentklasse. \tabularnewline
		\filetype{png}, \filetype{jpeg}, \filetype{jpg} & Bild-File für die Inklusion von Bildern im Dokument. \tabularnewline
		\bottomrule
	\end{tabular}
	\caption{Weitere Quell-Files}
	\label{tab:source-files}
\end{table}

\section{Dokumentklasse}

\chapter{Bibliographie}

Ein gutes wissenschaftliches Dokument enthält eine Vielzahl an Quellen und Referenzen.
Natürlich stellt \LaTeX{} auch hierfür einige praktische Tools bereit, und wie üblich sind im Laufe der Geschichte einige standardmäßige Packages hinzugekommen, die wir im Folgenden kennenlernen werden.

\section{Das Bibfile}
Unser gutes Prinzip der Trennung von Inhalt und Format fortführend, werden Quellen in \LaTeX{} in einem eigenen File, dem sogenannten \emph{bibfile} gespeichert.
Dieses dient hier als eine Art Datenbank, in der Einträge für alle möglichen Quellen eingetragen werden.
Ein Dokument kann dann eines oder mehrere solche Bibfiles einbinden und daraus zitieren, wobei die Zitationen automatisch mit den gespeicherten Daten gefüllt werden.
Quellen, die nicht zitiert werden, tauchen auch nicht im automatischen Literaturverzeichnis auf, so dass ein Bibfile deutlich mehr Quellen enthalten kann als nötig -- dies erlaubt es erneut, ein Bibfile zwischen vielen Dokumenten und Nutzern zu teilen.
Der Lehrstuhl X beispielsweise verwaltet ein Git-Repository mit einer Vielzahl an Bibfiles, die insgesamt Hunderte forschungsrelevante Quellen enthalten.

Ein Bibfile hat für gewöhnlich die Endung \filetype{bib} und einen Namen wie \filepath{bibliography.bib} oder \filepath{references.bib}.
Das Format der Einträge setzt wie folgt zusammen:
\begin{latexlisting}
	@book{mussmaecher:2025,
		title = {Schreiben wissenschaftlicher Arbeiten in LaTeX},
		author = {Mußmächer, Linus},
		year = {2025},
		publisher = {Universtität Würzburg},
	}
\end{latexlisting}
Wir beginnen mit dem Typ des Eintrags nach einem \key{\@}-Zeichen.
Mögliche Typen sind in \autoref{tab:bib-types} aufgeführt.
In geschweiften Klammern folgend anschließend die Daten des Eintrags, allen voran die ID.
Die ID ist der einzige Punkt, der unter allen verwendeten Bibliographien eindeutig sein muss und sollte sinnvoll gewähnlt werden.
Eine übliche Konvention ist beispielsweise
\begin{latexlisting}
	<autor1>-<autor2>:<jahr>
\end{latexlisting}
wobei Kleinbuchstaben angehängt werden, um Veröffentlichungen mit gleichen Autoren und Jahr zu unterscheiden.
Die übrigen Einträge sind im Key-Value-Format und enthalten Informationen über die Veröffentlichung.
Einige mögliche Keys sind in \autoref{tab:bib-keys} aufgeführt.

\begin{table}
	\begin{tabular}{l p{6cm}}
		\toprule
		\textbf{Typ} & \tabularnewline
		\midrule
		\latexargument{book}
		& Bücher
		\tabularnewline
		\latexargument{article}
		& Artikel, Paper, Skript
		\tabularnewline
		\latexargument{online}
		& Webquelle
		\tabularnewline
		\latexargument{misc}
		& Sonstiges
		\tabularnewline
		\bottomrule
	\end{tabular}
	\caption{Bibliographieeintragstypen}
	\label{tab:bib-types}
\end{table}
\begin{table}
	\begin{tabular}{l p{8cm}}
		\toprule
		\textbf{Key} & \textbf{Bedeutung} \tabularnewline
		\midrule
		\latexargument{author}
		& Autor
		\tabularnewline
		\latexargument{title}
		& Titel
		\tabularnewline
		\latexargument{publisher}
		& Verlag
		\tabularnewline
		\latexargument{adress}
		& Adresse des Verlags
		\tabularnewline
		\latexargument{year}
		& Jahr der Veröffentlichung
		\tabularnewline
		\latexargument{url}
		& Web-Adresse
		\tabularnewline
		\latexargument{volume},\latexargument{number} 
		& Für Objekte in Journalen oder Sammlungen
		\tabularnewline
		\latexargument{comment}
		& Weitere Anmerkungen
		\tabularnewline
		\bottomrule
	\end{tabular}
	\caption{Verhalten von Command-Erzeugern}
	\label{tab:bib-keys}
\end{table}
Beim Feld \latexargument{author} ist zu beachten, dass Autoren im Format \latexargument{<nachname>, <vorname>} angegeben und mehrere Autoren immer (alle!) mit \latexargument{and} getrennt werden.
\LaTeX{} übernimmt dann selbständig die Formatierung der Autoren.
Viele Seiten, die wissenschaftliche Artikel bereitstellen (z.B. arXiV oder SciHub), können auch automatisch Bib-Einträge mit allen ihnen bekannten Informationen erzeugen.

Um ein solches Bibfile nun einbinden zu können, benötigen wir zuerst die Package \latexpackage{biblatex}.
In TeXStudio ist es außerdem nötig, entweder in den Einstellungen das Bibliography-Backend auf \latexargument{biber} umzustellen oder die Option
\begin{latexlisting}
	% Präambel
	\usepackage[backend=bibtex]{biblatex}
\end{latexlisting}
zu verwenden, damit das von TeXStudio verwendete Backend mit dem von \latexpackage{biblatex} erwarteten übereinstimmt.
In Overleaf ist nichts derartige nötig.
Ist die Package eingebunden, können wir mit
\begin{latexlisting}
	% Präambel
	\addbibresource{references.bib}
	% Text
	\printbibliography
\end{latexlisting}
Wie bereits erwähnt können auf diese Art und Weise mehrere Bibfiles eingebunden werden.
In der gedruckten Bibliography steht -- mangels Zitaten -- im Augenblick noch nicht viel, doch das wird sich mit dem nächsten Kapitel ändern.
Man beachte auch, dass, wie bei vielen Package zuvor, die \latexpackage{babel}-Package für eine automatische Übersetzung von Schlüsselworten sorgt.

\section{Zitationen}

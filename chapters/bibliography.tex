\chapter{Literatur}

Ein gutes wissenschaftliches Dokument enthält eine Vielzahl an Quellen und Referenzen.
Natürlich stellt \LaTeX{} auch hierfür einige praktische Tools bereit, und wie üblich sind im Laufe der Geschichte einige standardmäßige Packages hinzugekommen, die wir im Folgenden kennenlernen werden.
Zu allem, was wir im Folgenden lernen, ist zusätzlich auch das Cheatsheet \url{https://tug.ctan.org/info/biblatex-cheatsheet/biblatex-cheatsheet.pdf} wärmstens zu empfehlen.

\section{Das Bibfile}
Unser gutes Prinzip der Trennung von Inhalt und Format fortführend, werden Quellen in \LaTeX{} in einem eigenen File, dem sogenannten \emph{bibfile} gespeichert.
Dieses dient hier als eine Art Datenbank, in der Einträge für alle möglichen Quellen eingetragen werden.
Ein Dokument kann dann eines oder mehrere solche Bibfiles einbinden und daraus zitieren, wobei die Zitationen automatisch mit den gespeicherten Daten gefüllt werden.
Quellen, die nicht zitiert werden, tauchen auch nicht im automatischen Literaturverzeichnis auf, so dass ein Bibfile deutlich mehr Quellen enthalten kann als nötig -- dies erlaubt es erneut, ein Bibfile zwischen vielen Dokumenten und Nutzern zu teilen.
Der Lehrstuhl X beispielsweise verwaltet ein Git-Repository mit einer Vielzahl an Bibfiles, die insgesamt Hunderte forschungsrelevante Quellen enthalten.

Ein Bibfile hat für gewöhnlich die Endung \filetype{bib} und einen Namen wie \filepath{bibliography.bib} oder \filepath{references.bib}.
Das Format der Einträge setzt wie folgt zusammen:
\begin{latexlisting}
	@book{mussmaecher:2025,
		title = {Schreiben wissenschaftlicher Arbeiten in LaTeX},
		author = {Mußmächer, Linus},
		year = {2025},
		publisher = {Universtität Würzburg},
	}
\end{latexlisting}
Wir beginnen mit dem Typ des Eintrags nach einem \key{\@}-Zeichen.
Mögliche Typen sind in \autoref{tab:bib-types} aufgeführt.
In geschweiften Klammern folgend anschließend die Daten des Eintrags, allen voran die ID.
Die ID ist der einzige Punkt, der unter allen verwendeten Bibliographien eindeutig sein muss und sollte sinnvoll gewähnlt werden.
Eine übliche Konvention ist beispielsweise
\begin{latexlisting}
	<autor1>-<autor2>:<jahr>
\end{latexlisting}
wobei Kleinbuchstaben angehängt werden, um Veröffentlichungen mit gleichen Autoren und Jahr zu unterscheiden.
Die übrigen Einträge sind im Key-Value-Format und enthalten Informationen über die Veröffentlichung.
Einige mögliche Keys sind in \autoref{tab:bib-keys} aufgeführt, alle weiteren finden sich im Cheatsheet.

\begin{table}
	\begin{tabular}{l p{6cm}}
		\toprule
		\textbf{Typ} & \tabularnewline
		\midrule
		\latexargument{book}
		& Bücher
		\tabularnewline
		\latexargument{article}
		& Artikel, Paper, Skript
		\tabularnewline
		\latexargument{online}
		& Webquelle
		\tabularnewline
		\latexargument{misc}
		& Sonstiges
		\tabularnewline
		\bottomrule
	\end{tabular}
	\caption{Bibliographieeintragstypen}
	\label{tab:bib-types}
\end{table}
\begin{table}
	\begin{tabular}{l p{8cm}}
		\toprule
		\textbf{Key} & \textbf{Bedeutung} \tabularnewline
		\midrule
		\latexargument{author}
		& Autor
		\tabularnewline
		\latexargument{title}
		& Titel
		\tabularnewline
		\latexargument{publisher}
		& Verlag
		\tabularnewline
		\latexargument{adress}
		& Adresse des Verlags
		\tabularnewline
		\latexargument{year}
		& Jahr der Veröffentlichung
		\tabularnewline
		\latexargument{url}
		& Web-Adresse
		\tabularnewline
		\latexargument{volume},\latexargument{number} 
		& Für Objekte in Journalen oder Sammlungen
		\tabularnewline
		\latexargument{comment}
		& Weitere Anmerkungen
		\tabularnewline
		\bottomrule
	\end{tabular}
	\caption{Verhalten von Command-Erzeugern}
	\label{tab:bib-keys}
\end{table}
Beim Feld \latexargument{author} ist zu beachten, dass Autoren im Format \latexargument{<nachname>, <vorname>} angegeben und mehrere Autoren immer (alle!) mit \latexargument{and} getrennt werden.
\LaTeX{} übernimmt dann selbständig die Formatierung der Autoren.
Viele Seiten, die wissenschaftliche Artikel bereitstellen (z.B. arXiV), können auch automatisch Bib-Einträge mit allen ihnen bekannten Informationen erzeugen.
Ähnliches gilt für die meisten handelsüblichen Zitationssoftware wie Zotero, Citavi etc.

Um ein solches Bibfile nun einbinden zu können, benötigen wir zuerst die Package \latexpackage{biblatex}.
In TeXStudio ist es außerdem nötig, entweder in den Einstellungen das Bibliography-Backend auf \latexargument{biber} umzustellen oder die Option
\begin{latexlisting}
	% Präambel
	\usepackage[backend=bibtex]{biblatex}
\end{latexlisting}
zu verwenden, damit das von TeXStudio verwendete Backend mit dem von \latexpackage{biblatex} erwarteten übereinstimmt.
In Overleaf ist nichts derartige nötig.
Ist die Package eingebunden, können wir mit
\begin{latexlisting}
	% Präambel
	\addbibresource{references.bib}
\end{latexlisting}
Wie bereits erwähnt können auf diese Art und Weise mehrere Bibfiles eingebunden werden.

\section{Zitationen}
Ist die Bibliographie erst einmal verlinkt, ist das Zitieren von Quellen recht einfach.
\begin{latexlisting}
	Wie man Quellen zitiert, haben wir aus \cite{mussmaecher:2025} gelernt.
\end{latexlisting}
Verwendet man dabei die \latexpackage{hyperref}-Package, wird das Zitat sogar zum Literaturverzeichnis (siehe \autoref{sec:literaturverzeichnis}) verlinkt.
Mehrere Quellen können auch auf einmal zitiert werden
\begin{latexlisting}
	Wie man Quellen zitiert, haben wir aus \cite{mussmaecher:2025,mussmaecher:2025a} gelernt.
\end{latexlisting}
Um anzupassen, wie eine solche Zitation aussieht, können wir den \emph{Zitationsstil} anpassen.
\begin{latexlisting}
	\usepackage[citestyle=alphabetic]{biblatex}
\end{latexlisting}
Mögliche Zitationsstile sind in \autoref{tab:cite-styles} aufgeführt.
\begin{table}
	\begin{tabular}{l l}
		\toprule
		\latexargument{numeric} &
		\latexargument{alphabetic} \tabularnewline
		\latexargument{authoryear} &
		\latexargument{authortitle} \tabularnewline 
		\latexargument{verbose} &
		\latexargument{reading} \tabularnewline
		\latexargument{draft} &
		\dots \tabularnewline
		\bottomrule
	\end{tabular}
	\caption{Zitationsstile}
	\label{tab:cite-styles}
\end{table}
Mittels eines optionalen Arguments können wir außerdem Anmerkungen zu unserer Zitation hinzufügen:
\begin{latexlisting}
	Wie man Quellen zitiert, haben wir aus \cite[Kapitel 7]{mussmaecher:2025} gelernt.
\end{latexlisting}
Besteht diese Anmerkunge nur aus einer Seite, genügt es, die Zahl anzugeben.
\begin{latexlisting}
	Wie man Quellen zitiert, haben wir aus \cite[69]{mussmaecher:2025} gelernt.
\end{latexlisting}
Die Packages \latexpackage{biblatex} und \latexpackage{babel} kümmern sich dann um ein sprachlich richtiges Label wie \enquote{S.} oder \enquote{pp.}.
Je nach PDF-Viewer sieht man hier eventuell grüne Ränder um seine Zitationen -- diese tauchen beim Drucken nicht auf, und können im PDF mittels einer Option der \latexpackage{hyperref}-Package unterdrückt werden.
\begin{latexlisting}
	\usepackage[hidelinks=true]{hyperref}
\end{latexlisting}
Neben dem gewöhnlichen Command \latexcommand{cite} existieren auch einige Alternativen, die in speziellen Situationen verwendet werden können.
\begin{latexlisting}
	In \textcite{mussmaecher:2025} finden wir interessante Informationen, auch das andere Buch \parencite{mussmaecher:2025a} ist toll.
	Eventuell findet man auch anderswo\footcite{mussmaecher:2025} etwas?
	Oder hier\supercite{mussmaecher:2025a}?
\end{latexlisting}

\section{Literaturverzeichnis}\label{sec:literaturverzeichnis}
Haben wir nun alle Quellen zitiert, können wir uns mittels \latexcommand{printbibliography} ein Literaturverzeichnis anzeigen lassen.
\begin{latexlisting}
	\printbibliography
\end{latexlisting}
Zur Sortierung dieses Literaturverzeichnises existiert wieder eine Package-Option.
\begin{latexlisting}
	\usepackage[sorting=nyt]{biblatex}
\end{latexlisting}
Mögliche Optionen sind dabei \latexargument{nyt}, \latexargument{nty}, \latexargument{ynt} und \latexargument{ydnt}, wobei die Buchstaben für \enquote{Name}, \enquote{Author} und \enquote{Year} (sowie \enquote{descending}) stehen.
Andere Sortierung sind standardmäßig nicht möglich.
Der Stil der Bibliographie kann ähnlich wie der Zitierstil angepasst werden.
\begin{latexlisting}
	\usepackage[sorting=nyt,bibstyle=numeric]{biblatex}
\end{latexlisting}
wobei man für gewöhnlich Zitier- und Bibliographiestil gemeinsam festlegt.
\begin{latexlisting}
	\usepackage[sorting=nyt,style=numeric]{biblatex}
\end{latexlisting}
Möglich Stile sind in \autoref{tab:bib-styles} aufgeführt.
\begin{table}
	\begin{tabular}{l l l l}
		\toprule
		\latexargument{numeric} &
		\latexargument{numeric-comp} &
		\latexargument{numeric-verb} &
		\tabularnewline
		\latexargument{alphabetic} &
		\latexargument{alphabetic-verb} &
		&
		\tabularnewline
		\latexargument{authoryear} &
		\latexargument{authoryear-comp} &
		\latexargument{authoryear-ibid} &
		\latexargument{authoryear-icomp} \tabularnewline
		\latexargument{authortitle} &
		\latexargument{authortitle-comp} &
		\latexargument{authortitle-ibid} &
		\latexargument{authortitle-icomp} \tabularnewline
		\latexargument{verbose} &
		\latexargument{reading} &
		\latexargument{draft} &
		\dots \tabularnewline		
		\bottomrule
	\end{tabular}
	\caption{Zitationsstile}
	\label{tab:bib-styles}
\end{table}
Es existieren noch weitere Stile, die von \latexpackage{biblatex} automatisch geladen werden können, z.B. \latexargument{apa}.
Man kann sich hierzu im Internet (\url{ctan.org}) oder dem Cheatsheet informieren.
Weiterhin können wir sogenannte Rückverweise in der Bibliographie anzeigen.
\begin{latexlisting}
	\usepackage[backref=true]{biblatex}
\end{latexlisting}
Ein letztes nützliches Feature ist die Möglichkeit, die Bibliographie nach dem Typ der Zitation aufzusplitten.
\begin{latexlisting}
	\printbibliography[title={Bücher}, type=book]
	\printbibliography[title={Andere Quellen}, nottype=book]
\end{latexlisting}

\section{Glossare}
In langen Texten ist oft nützlich ein Glossar an wichtigen Begriffen und Definitionen zu halten.
Hierzu existiert die Package \latexpackage{makeidx}.
Mittels des Commands \latexcommand{index} können Indexeinträge erstellt werden.
\begin{latexlisting}
	\begin{definition}[Magma]\label{def:magma}\index{Magma}
		Eine Menge $M$ gemeinsam mit einer zweistelligen Verknüpfung $\cdot: M \times M \to M$ heißt \emph{Magma.}
	\end{definition}
\end{latexlisting}
Anschließend muss in der Präambel, nach Inklusion der Package aber vor Beginn des Dokuments, das Command \latexcommand{makeindex} inkludiert werden.
Alles was noch zu tun ist, ist das fertige Glossar an einer angemessenen Stelle (im Anhang, seltener am Anfang zwischen Vorwort und Einleitung) durch \latexcommand{printindex} abzudrucken.
Mit der Erstellung eines Index sind wieder einige Kompilierungsprozesse und Hilfsfiles verbunden, um die sich glücklicherweise Overleaf oder Hilfsprogramme wie \filepath{latexmk} automatisch kümmern -- in TeXStudio ist es nötig, das Projekt zu kompilieren, dann unter \texttt{Tools/Index} den Index zu bauen und erneut zu kompilieren.
% Siehe: https://tex.stackexchange.com/questions/298271/makeindex-is-not-called-when-compiling-in-texstudio

Möchten wir \emph{Nomenklaturen} erzeugen, eine spezielle Art von Index, in der auch Kurzdefinitionen enthalten sind, können wir stattdessen die Package \latexpackage{nomencl} verwenden, deren Dokumentation unter \url{https://ctan.net/macros/latex/contrib/nomencl/nomencl.pdf} ihre Verwendung erklärt.

\chapter{Commands}

In den vorhergehenden 5 Kapiteln haben wir viele der in \LaTeX{} enthaltenen Features kennengelernt und benutzt.
Eine der großen Stärken von \LaTeX{} ist es aber, dass wir diese Funktionalitäten selbstständig erweitern und anpassen können.
In diesem Kapitel lernen wir, eigene Command zu definieren -- von simplen Abkürzungen bis hin zu komplexen Funktionen mit mehreren Argumenten.

\section{Makros in \TeX}

Die einfachste Art, ein neues Makro zu definieren, ist das \TeX -Command \latexcommand{def}.
\begin{latexlisting}
	\def\derivative{\del}

	Betrachte $\derivative x$.
\end{latexlisting}
Was passiert hier?
\LaTeX{} evaluiert Commands in einem Prozess namens \emph{expansion}, d.h. immer wenn \LaTeX{} ein \emph{Token} findet, das expandierbar ist (beispielsweise \latexcommand{derivative}), so wir dieses durch seine Definition ersetzt, in diesem Fall \latexcommand{del}.
Auch \latexcommand{del} ist expandierbar, wird also erneut ersetzt.
Dieser Prozess wird so lange wiederholt, bis wir bei nur un-expandierbaren Tokens -- hier der Unicode-Character der partiellen Ableitung -- angekommen sind, wie beispielsweise Buchstaben oder einige spezielle Anweisungen wie \latexcommand{def}.
Wann und wie lange \LaTeX{} expandiert, ist kompliziert.
Man betrachte das folgende Beispiel und überlege sich, was das zu erwartende Ergebnis ist.
\begin{latexlisting}
	\def\aaa{Hello}
	\def\bbb{\aaa}
	\def\aaa{Hallo}

	Ich sage \bbb .
\end{latexlisting}
Was passiert, wenn man das \latexcommand{def} in der zweiten Zeile durch \latexcommand{edef} ersetzt?
Wir sehen, dass \LaTeX{} in einigen Kontexten (wie beispielsweise dem hinteren Teil von \latexcommand{def}) nicht sofort expandiert.
Theoretisch sind die geschweiften Klammern um die Definition nicht nötig, wenn es sich nur um ein einzelnes Token handelt, aber sie sind trotzdem guter Stil.
Weiterhin sollten solche Definitionen auch nur in der Präambel, vorzugsweise in einem eigenen File mit einen Titel wie \filepath{macros.tex}, leben und nicht im Fließtext, auch wenn dies möglich ist.
Neben \latexcommand{def} und \latexcommand{edef} existiert auch \latexcommand{let}, das seine Definition nur ein einziges Mal expandiert, bevor das Makro festgelegt wird.
Somit kann \latexcommand{let} genutzt werden, um einem Makro einen neuen Namen zu geben und wird daher oft gern genutzt, um Definitionen zu vertauschen, etwa wir folgt:
\begin{latexlisting}
	\let\oldepsilon\epsilon
	\let\epsilon\varepsilon
	\let\varepsilon\oldepsilon
\end{latexlisting}
Mithilfe des un-expandierbaren Command \latexcommand{undefined} können wir existierende Makros löschen.
Dies ist beispielsweise nützlich, um Makro-Konflikte zwischen verschiedenen Packages zu vermeiden.
\begin{latexlisting}
	% Präambel
	\usepackage{nchairx}
	\let\unit\undefined
	\usepackage{siunitx}
\end{latexlisting}

\subsection{Blöcke}

Weiterhin sei noch erwähnt, dass Makros sich auf Blöcke beschränken.
\begin{latexlisting}
	\def\aaa{Hello}
	{
		\def\aaa{Hallo}
		Ich sage \aaa .
	}
	Ich sage \aaa .
\end{latexlisting}
Dieses Verhalten lässt sich mittels des (nicht-expandierbaren) Makros \latexcommand{global} deaktivieren.
\begin{latexlisting}
	\def\aaa{Hello}
	{
		\global\def\aaa{Hallo}
		Ich sage \aaa .
	}
	Ich sage \aaa .
\end{latexlisting}
Beide diese Features sind allerdings meist nur in Situationen nützlich, in denen man spaßige Dinge mit dem \TeX-Makro-System anstellen möchte.

\section{Makros in \LaTeX}

Statt des alten \TeX-Systems ist es empfehlenswert, das modernere und sicherere \LaTeX-Makro-System zu nutzen.

\subsection{Makros ohne Argumente}
Zuerste betrachten wir wieder einfache Makros wie oben, die keine Argumente akzeptieren.
Statt \latexcommand{def} nutzen wir hier \latexcommand{newcommand}:
\begin{latexlisting}
	\newcommand{\integers}{\mathds{Z}}

	Die ganzen Zahlen schreiben wir als $\integers$.
\end{latexlisting}
Use-Cases für diese Art von simplen Makros sind vielfältig:
\begin{itemize}
	\item Verkürzung langer Fachbegriffe
	\item Vermeidung von Rechtschreibfehlern:
	\begin{latexlisting}
		\newcommand{\DNA}{Desoxyribonukleinsäure}
	\end{latexlisting}
	\item Semantisierung, insbesondere in mathematischen Formeln
	\begin{latexlisting}
		\newcommand{\iff}{\Leftrightarrow}
	\end{latexlisting}
	\item Vermindeter Schreibaufwand
	\item Einheitliche Formatierung zentraler Begriffe
	\begin{latexlisting}
		\newcommand{\germany}{\includegraphics[height=0.6\baselineskip]{germany-flag.png} \textbf{Germany}}
	\end{latexlisting}
	\item Erlaubt zentrale Änderungen
	\begin{latexlisting}
		\newcommand{\toolname}{NetVisard}
	\end{latexlisting}
\end{itemize}

\subsection{Makros mit Argumenten}
Zusätzlich können wir aber auch Makros definieren, deren Verhalten sich mithilfe von Argumenten anpassen lässt und die dadurch viel flexibler sind als die vorher bekannten\footnote{Es sei erwähnt, dass ich dies auch mit \latexcommand{def} umsetzen lässt, der Syntax ist nur weniger durchschaubar}.
\begin{latexlisting}
	\newcommand{\differential}[1]{\frac{\D}{\D #1}}

	Betrachte $\differential{x} f(x) = \differential{t} f(t)$.
\end{latexlisting}
Hier geben wir in eckigen Klammern die Anzahl der zu erwartenden Argumente an und können diese dann im Körper des Makros mit \latexargument{\#1}, \latexargument{\#2}, etc. nutzen (und sie werden zur Auswertungszeit expandiert).
Dies kann zur Semantiesierung von Formatierungsentscheidungen nützlich sein.
Das \filepath{macros.tex}-File dieses Dokuments, beispielsweise, besteht zu einem großen Teil aus verschiedenen Varianten, Schrift in Monospace zu setzen.
\begin{latexlisting}
	\newcommand{\latexcommand}[1]{\texttt{\textbackslash #1}}
	\newcommand{\latexpackage}[1]{\texttt{#1}}
	\newcommand{\latexargument}[1]{\texttt{#1}}
	\newcommand{\latexenvironment}[1]{\texttt{#1}}
	\newcommand{\filetype}[1]{.\texttt{#1}}
	\newcommand{\filepath}[1]{\texttt{#1}}
	\newcommand{\key}[1]{\texttt{#1}}
\end{latexlisting}
Wir können natürlich auch mehrere Argumente verwenden:
\begin{latexlisting}
	\newcommand{\add}[2]{#1 + #2 = \fpeval{#1 + #2}}

	Betrachte $\add{3}{5}$
\end{latexlisting}
Wobei das Command \latexcommand{fpeval} (bereitgestellt durch eine Package, die in KOMA bereits enthalten ist) in \LaTeX{} während des Kompilierungsprozesses Rechnungen durchführen kann.
Außerdem ist es möglich, Default-Werte für Argumente anzugeben und diese Argumente dadurch in optionale umzuwandeln:
\begin{latexlisting}
	\newcommand{\reals}[1][n]{\mathbb{R}^{#1}}

	Es ist $\reals = \reals[4]$ für $n = 4$.
\end{latexlisting}
Commands mit Argumenten haben erneut alle Vorteile derer ohne Argumente, und zusätzlich:
\begin{itemize}
	\item Deutlich weniger Schreibaufwand.
	\item Vermeidung von Code-Duplikation.
	\item Kontextsensitive Anpassung.
	\item Übersichtlicherer Code.
\end{itemize}
Abschließend sei noch ein Command definiert, dass mir bei der Erstellung von Arbeitsblättern sehr nützlich war und das wir im Laufe des nächsten Abschnitts erweitern werden:
\begin{latexlisting}
	\newcommand{\blank}[1]{%
		\underline{\phantom{#1#1#1}}%
	}
\end{latexlisting}
Die Kommentare am Ende der Zeile sind in \LaTeX{}-Commands, die mehrere Zeilen überspannen oft üblich und verhindern, dass sich ungewollte Leerzeichen oder gar Zeilenumbrüche bei der Verwendung einschleichen.
Dieses Command erzeugt eine unterstrichene Lücke groß genug, dass handschriftlich das passende Wort gut hineinpasst.
Später werden wir das Command so modifieren, dass auf einfache Art und Weise auch eine Lösung erstellt werden kann, in der das gewollte Wort bereits eingetragen ist.

\section{Kontrollstrukturen}


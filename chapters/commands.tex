\chapter{Commands}

In den vorhergehenden 5 Kapiteln haben wir viele der in \LaTeX{} enthaltenen Features kennengelernt und benutzt.
Eine der großen Stärken von \LaTeX{} ist es aber, dass wir diese Funktionalitäten selbstständig erweitern und anpassen können.
In diesem Kapitel lernen wir, eigene Command zu definieren -- von simplen Abkürzungen bis hin zu komplexen Funktionen mit mehreren Argumenten.

\section{Makros in \TeX}

Die einfachste Art, ein neues Makro zu definieren, ist das \TeX -Command \latexcommand{def}.
\begin{latexlisting}
	\def\derivative{\del}

	Betrachte $\derivative x$.
\end{latexlisting}
Was passiert hier?
\LaTeX{} evaluiert Commands in einem Prozess namens \emph{expansion}, d.h. immer wenn \LaTeX{} ein \emph{Token} findet, das expandierbar ist (beispielsweise \latexcommand{derivative}), so wir dieses durch seine Definition ersetzt, in diesem Fall \latexcommand{del}.
Auch \latexcommand{del} ist expandierbar, wird also erneut ersetzt.
Dieser Prozess wird so lange wiederholt, bis wir bei nur un-expandierbaren Tokens -- hier der Unicode-Character der partiellen Ableitung -- angekommen sind, wie beispielsweise Buchstaben oder einige spezielle Anweisungen wie \latexcommand{def}.
Wann und wie lange \LaTeX{} expandiert, ist kompliziert.
Man betrachte das folgende Beispiel und überlege sich, was das zu erwartende Ergebnis ist.
\begin{latexlisting}
	\def\aaa{Hello}
	\def\bbb{\aaa}
	\def\aaa{Hallo}

	Ich sage \bbb .
\end{latexlisting}
Was passiert, wenn man das \latexcommand{def} in der zweiten Zeile durch \latexcommand{edef} ersetzt?
Wir sehen, dass \LaTeX{} in einigen Kontexten (wie beispielsweise dem hinteren Teil von \latexcommand{def}) nicht sofort expandiert.
Theoretisch sind die geschweiften Klammern um die Definition nicht nötig, wenn es sich nur um ein einzelnes Token handelt, aber sie sind trotzdem guter Stil.
Weiterhin sollten solche Definitionen auch nur in der Präambel, vorzugsweise in einem eigenen File mit einen Titel wie \filepath{macros.tex}, leben und nicht im Fließtext, auch wenn dies möglich ist.
Neben \latexcommand{def} und \latexcommand{edef} existiert auch \latexcommand{let}, das seine Definition nur ein einziges Mal expandiert, bevor das Makro festgelegt wird.
Somit kann \latexcommand{let} genutzt werden, um einem Makro einen neuen Namen zu geben und wird daher oft gern genutzt, um Definitionen zu vertauschen, etwa wir folgt:
\begin{latexlisting}
	\let\oldepsilon\epsilon
	\let\epsilon\varepsilon
	\let\varepsilon\oldepsilon
\end{latexlisting}
Mithilfe des un-expandierbaren Command \latexcommand{undefined} können wir existierende Makros löschen.
Dies ist beispielsweise nützlich, um Makro-Konflikte zwischen verschiedenen Packages zu vermeiden.
\begin{latexlisting}
	% Präambel
	\usepackage{nchairx}
	\let\unit\undefined
	\usepackage{siunitx}
\end{latexlisting}

\subsection{Blöcke}

Weiterhin sei noch erwähnt, dass Makros sich auf Blöcke beschränken.
\begin{latexlisting}
	\def\aaa{Hello}
	{
		\def\aaa{Hallo}
		Ich sage \aaa .
	}
	Ich sage \aaa .
\end{latexlisting}
Dieses Verhalten lässt sich mittels des (nicht-expandierbaren) Makros \latexcommand{global} deaktivieren.
\begin{latexlisting}
	\def\aaa{Hello}
	{
		\global\def\aaa{Hallo}
		Ich sage \aaa .
	}
	Ich sage \aaa .
\end{latexlisting}
Beide diese Features sind allerdings meist nur in Situationen nützlich, in denen man spaßige Dinge mit dem \TeX-Makro-System anstellen möchte.

\section{Makros in \LaTeX}

Statt des alten \TeX-Systems ist es empfehlenswert, das modernere und sicherere \LaTeX-Makro-System zu nutzen.

\subsection{Makros ohne Argumente}
Zuerste betrachten wir wieder einfache Makros wie oben, die keine Argumente akzeptieren.
Statt \latexcommand{def} nutzen wir hier \latexcommand{newcommand}:
\begin{latexlisting}
	\newcommand{\integers}{\mathds{Z}}

	Die ganzen Zahlen schreiben wir als $\integers$.
\end{latexlisting}
Das Expansions-Verhalten ist hier dasselbe wie bei $\latexcommand{def}$, d.h. es wird erst später rekursive expandiert.
Use-Cases für diese Art von simplen Makros sind vielfältig:
\begin{itemize}
	\item Verkürzung langer Fachbegriffe
	\item Vermeidung von Rechtschreibfehlern:
	\begin{latexlisting}
		\newcommand{\DNA}{Desoxyribonukleinsäure}
	\end{latexlisting}
	\item Semantisierung, insbesondere in mathematischen Formeln
	\begin{latexlisting}
		\newcommand{\iff}{\Leftrightarrow}
	\end{latexlisting}
	\item Vermindeter Schreibaufwand
	\item Einheitliche Formatierung zentraler Begriffe
	\begin{latexlisting}
		\newcommand{\germany}{\includegraphics[height=0.6\baselineskip]{germany-flag.png} \textbf{Germany}}
	\end{latexlisting}
	\item Erlaubt zentrale Änderungen
	\begin{latexlisting}
		\newcommand{\toolname}{NetVisard}
	\end{latexlisting}
\end{itemize}

\subsection{Makros mit Argumenten}
Zusätzlich können wir aber auch Makros definieren, deren Verhalten sich mithilfe von Argumenten anpassen lässt und die dadurch viel flexibler sind als die vorher bekannten\footnote{Es sei erwähnt, dass ich dies auch mit \latexcommand{def} umsetzen lässt, der Syntax ist nur weniger durchschaubar}.
\begin{latexlisting}
	\newcommand{\differential}[1]{\frac{\D}{\D #1}}

	Betrachte $\differential{x} f(x) = \differential{t} f(t)$.
\end{latexlisting}
Hier geben wir in eckigen Klammern die Anzahl der zu erwartenden Argumente an und können diese dann im Körper des Makros mit \latexargument{\#1}, \latexargument{\#2}, etc. nutzen (und sie werden zur Auswertungszeit expandiert).
Dies kann zur Semantiesierung von Formatierungsentscheidungen nützlich sein.
Das \filepath{macros.tex}-File dieses Dokuments, beispielsweise, besteht zu einem großen Teil aus verschiedenen Varianten, Schrift in Monospace zu setzen.
\begin{latexlisting}
	\newcommand{\latexcommand}[1]{\texttt{\textbackslash #1}}
	\newcommand{\latexpackage}[1]{\texttt{#1}}
	\newcommand{\latexargument}[1]{\texttt{#1}}
	\newcommand{\latexenvironment}[1]{\texttt{#1}}
	\newcommand{\filetype}[1]{.\texttt{#1}}
	\newcommand{\filepath}[1]{\texttt{#1}}
	\newcommand{\key}[1]{\texttt{#1}}
\end{latexlisting}
Wir können natürlich auch mehrere Argumente verwenden:
\begin{latexlisting}
	\newcommand{\add}[2]{#1 + #2 = \fpeval{#1 + #2}}

	Betrachte $\add{3}{5}$
\end{latexlisting}
Wobei das Command \latexcommand{fpeval} (bereitgestellt durch eine Package, die in KOMA bereits enthalten ist) in \LaTeX{} während des Kompilierungsprozesses Rechnungen durchführen kann.
Außerdem ist es möglich, Default-Werte für Argumente anzugeben und diese Argumente dadurch in optionale umzuwandeln:
\begin{latexlisting}
	\newcommand{\reals}[1][n]{\mathbb{R}^{#1}}

	Es ist $\reals = \reals[4]$ für $n = 4$.
\end{latexlisting}
Commands mit Argumenten haben erneut alle Vorteile derer ohne Argumente, und zusätzlich:
\begin{itemize}
	\item Deutlich weniger Schreibaufwand.
	\item Vermeidung von Code-Duplikation.
	\item Kontextsensitive Anpassung.
	\item Übersichtlicherer Code.
\end{itemize}
Komplexere Commands werden oft durch Zeilenumbrüche und Einrückungen strukturiert und lesbarer gemacht:
\begin{latexlisting}
	\newcommand{\blank}[1]{%
		\underline{\phantom{#1#1#1}}%
	}
\end{latexlisting}
In diesem Fall sind die Kommentar-Zeichen am Ende der Zeile unerlässlich und verhindern, dass sich ungewollte Leerzeichen oder gar Zeilenumbrüche bei der Verwendung einschleichen -- eine der vielen kleinen Eigenheiten, die das Programmieren in \LaTeX{} deutlich anstrengder machen als in konventionellen Programmiersprachen.

\subsection{Varianten}
Auch zu \latexcommand{newcommand} existieren mehrere Alternativen.
Beide haben dasselbe Expasions-Verhalten wie \latexcommand{newcommand} und unterscheiden sich nur in ihrem Verhalten, wenn das zu definierende Command bereits existiert.
Das entsprechende Verhalten ist in Tabelle 

\begin{table}[!h]
	\begin{tabular}{l p{4cm} p{4cm}}
		\toprule
		\textbf{Command-Erzeuger} & \textbf{Fehler, falls Command existiert} & \textbf{Fehler, falls Command noch nicht existiert}\tabularnewline
		\midrule
		\latexcommand{def} etc.
		& Nein
		& Nein
		\tabularnewline
		\latexcommand{newcommand}
		& Ja
		& Nein
		\tabularnewline
		\latexcommand{renewcommand}
		& Nein
		& Ja
		\tabularnewline
		\latexcommand{providecommand}
		& Nein
		& Nein
		\tabularnewline
		\bottomrule
	\end{tabular}
	\caption{Verhalten von Command-Erzeugern}
	\label{tab:commands-creation}
\end{table}

\section{Kontrollstrukturen}

Auch mit Argumenten ist das Verhalten von Commands bisher noch recht linear.
Um Makros noch flexibler und potenter zu machen, können wir sogenannte \emph{Kontrollstrukturen} verwenden, wie sie aus Programmiersprachen bekannt ist.
\LaTeX{} ist zwar keine Programmiersprache im engeren Sinne, aber trotzdem Turing-vollständig (d.h. jedes \emph{berechenbare} Programm lässt sich, wenn auch mit Aufwand, in \LaTeX{} simulieren).
Knuth wollte sich ursprünglich nicht die Mühe machen, vollständige Programmierbarkeit in einem einfachen Textprozessor zu implementieren, wurde aber von einem Kollegen überzeugt.
Dies führte zur Entstehung von primitiven Kontrollkontrukten wie \latexcommand{if} und \latexcommand{for}.
Wir werden im Folgenden modernere, sicherere Versionen kennenlernen, die von Packages bereitgestellt werden.

\subsection{Bedingungen}
Die erste und wichtigste Kontrollstruktur ist die \emph{bedingte Anweisung}.
Sie wird von der Package \latexpackage{ifthen} bereitgstellt:
\begin{latexlisting}
	\ifthenelse{\isodd{\thepage}}{
		Wir sind auf einer ungeraden Seite.
	}{
		Wir sind auf einer geraden Seite.
	}
\end{latexlisting}
Falls die Bedingung im ersten Argument wahr ist, wird das zweite Argument expandiert und ausgegeben, andernfalls das zweite.
Wir können nicht nur Counter und Zahlen vergleichen, sondern auch andere Bedingungen verwenden:
\begin{itemize}
	\item \latexargument{<number> < <number>} überprüft, ob eine Zahl kleiner als eine andere ist.
	\item \latexargument{<number> = <number>} überprüft, ob eine Zahl gleich einer anderen ist.
	\item \latexargument{<number> > <number>} überprüft, ob eine Zahl größer als eine andere ist.
	\item \latexcommand{isodd\{<number>\}} überprüft, ob eine Zahl ungerade ist.
	\item \latexcommand{equal\{<string1>\}\{<string2>\}} überprüft, ob zwei Texte gleich sind.
	\item \latexcommand{isundefined\{<commandname>\}} überprüft, ob ein Command nicht definiert ist.
\end{itemize}
Ein Beispiel:
\begin{latexlisting}
	\newcommand{\points}[2]{%
		#1 von #2 \ifthenelse{#2 = 1}{Punkt}{Punkten}
	}

	\points{3}{3}

	\points{1}{3}

	\points{1}{1}

	\points{0}{1}
\end{latexlisting}
Weiterhin können wir sogenannte \emph{Boolsche Variablen} (engl. \emph{booleans}) definieren, die als Stellschrauben genutzt werden können.
\begin{latexlisting}
	\newboolean{solution}
	\newcommand{\addtask}[2]{%
		#1 + #2 = \ifthenelse{%
			\boolean{solution}%
		}{%
			\fpeval{#1 + #2}%
		}{%
			\rule{1cm}{0.5pt}%
		}%
	}
	
	\setboolean{solution}{false}

	\begin{enumerate}
		\item \addtask{14}{91}
		\item \addtask{37}{29}
		\item \addtask{53}{17}
		\item \addtask{17}{53}
		\setboolean{solution}{true}
		\item \addtask{14}{91}
		\item \addtask{37}{29}
		\item \addtask{53}{17}
		\item \addtask{17}{53}
	\end{enumerate}
\end{latexlisting}
Als weiteres Beispiel können wir unsere Implementierung von \latexcommand{blank} entsprechend anpassen:
\begin{latexlisting}
	\newcommand{\blank}[1]{%
		\underline{\phantom{#1}%
		\ifthenelse{\boolean{solution}}{%
		#1%
		}{%
		\phantom{#1}%
		}%
		\phantom{#1}%
		}%
	}
\end{latexlisting}
Man möge das Command selbstständig testen und seine Funktion verstehen!

\subsection{Schleifen}

Neben der bedingten Anweisung ist die \emph{Schleife}, die eine oder mehrere Anweisungen wiederholt, ein wichtiges Konstrukt.
Auch sie wird von \latexpackage{ifthen} bereitgestellt, in Form des Commands \latexcommand{whiledo}.
(Bonusfrage: Warum schreiben wir kein Prozentzeichen hinter \enquote{Hello}?)
\begin{latexlisting}
	\newcounter{loop}	
	\whiledo{\value{loop} < 10}{%
		Hello!
		\stepcounter{loop}%
	}
\end{latexlisting}
Die Bedingung im ersten Argument von \latexcommand{whiledo} wird hier ausgewertet und, falls sie wahr ist, der Block im zweiten Argument ausgeführt.
Anschließend wird der Prozess so lange wiederholt, bis die Bedingung falsch ist.
Dabei ist wichtig, die Bedingung im Block so zu formulieren, dass sie auch irgendwann falsch werden \emph{kann}, sonst endet der Kompilierungsprozess niemals\footnote{Man beachte den Unterschied zu gewöhnlichen Programmiersprachen: Hier würde das Programm kompilieren, aber beim Ausführen in einer Endlosschleife hängen. In \LaTeX{} sind Kompilierung und Ausführung verknüpft.}.
Befürchtet man, dies zu vergessen, ist statt dem \emph{while-loop} ein \emph{for-loop} zu empfehlen, wie er beispielsweise von der \latexpackage{forloop}-Package bereitgestellt wird.

Für begrenzte Iterationen gibt es letztlich auch die Option eines \emph{enhanced-for} oder \emph{for-each}, beispielsweise aus der Package \latexpackage{pgffor}:
\begin{latexlisting}
	\foreach \i in {1,...,5}{%
		\i,
	}\\
	Everybody in the car, so come on, let's ride.\\
	To the liquor store around the corner\\
	The boys say they want some gin and juice\\
	But I really don't wanna

	\foreach \wish in {
		Monica in my life,
		Erica by my side,
		Rita's all I need,
		Tina's what I see,
		Sandra in the sun,
		Mary all night long,
		Jessica here I am,
		you makes me your man,
	}{%
		A little bit of \wish,\\
	}

	\foreach \x in {1,21,24,145,973}{%
		\ifthenelse{\fpeval{trunc(\x / 7) * 7 - \x} = 0}{%
			$\x$ ist durch $7$ teilbar.%
		}{%
			$\x$ ist nicht durch $7$ teilbar.%
		}%
	}
\end{latexlisting}
Auch wenn sich Rechnungen in \LaTeX{} mit der Funktion \latexcommand{fpeval} aus der Package \latexpackage{fp} (die in KOMA standardmäßig enthalten ist) recht einfach durchführen lassen, wollen wir dieses Kapitel mit einer \latexpackage{fpeval}-freien Implementierung des euklidischen Algorithmus zur Berechnung des größten gemeinsamen Teilers abschließen:
\begin{latexlisting}
	\newcounter{a}
	\newcounter{b}
	\newcommand{\ggT}[2]{%
		\setcounter{a}{#1}%
		\setcounter{b}{#2}%
		\whiledo{\not \value{b} = 0}{%
			\ifthenelse{\value{a} > \value{b}}{%
				\addtocounter{a}{-\value{b}}%
			}{%
				\addtocounter{b}{-\value{a}}%
			}%
		}%
		Der größte gemeinsame Teiler von $#1$ und $#2$ ist $\arabic{a}$.%
	}

	\ggT{143}{65}
	\ggT{9}{32}
	\ggT{1071}{462}
\end{latexlisting}
Es sei noch erwähnt, dass solche Berechnungen und Algorithmen bei jeder Kompilierung erneut durchgeführt werden, es ist also empfehlenswert, keine ganzen Simulationen in \LaTeX{}-Dokumente einzubauen.

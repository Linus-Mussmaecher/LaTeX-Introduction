\chapter{Mathematik}
Neben der wunderbaren überlegenen Typographie, dem großartigen Blocksatzalgorithmus und der Erweitarbeit ist einer der häufigsten Gründe, warum Wisschenschaftler \LaTeX{} verwenden, das formschöne Setzen mathematischer Formeln.

Noch in den 70ern wurden mathematische Paper gestaltet, indem man zuerst den Text per Hand schrieb.
Anschließend wurden mit einer Schreibmaschine die alphanumerischen Symbole getippt und die Sonderzeichen per Hand eingefügt -- wer im selben Text $a$, $\hat{a}$ und $\tilde{a}$ verwenden wollte, musste sich gut merken, wo er handschriftlich welche Verzierung einzufügen hatte.
Solche Dokumente konnten dann zur Verfielfältigung gefaxt oder fotokopiert werden.
Teilweise wurden sogar ganze Bücher mit dieser Technik geschrieben.

Über Zeit entstanden weitere Techniken, z.B. \emph{Varitype}, eine Art Schreibmaschine mit austauschbaren Rollen, auf der gleichzeitig lateinische und griechische Buchstaben geschrieben werden konnten, der \emph{IBM-Golfball}, eine Art Schreibmaschine, die einen rotierenden Golfball mit Sonderzeichen besaß, sowie \emph{tipits}, kleine, vom Nutzer anpassbare Schreibmaschinenlettern. 
Eines der dominierenden Systeme war für lange Zeit \emph{Photocomposition}, eine Schriftsatzmaschine für den Druck von Büchern oder professionellen Dokumenten, bei der statt Bleisatzlettern spezielle Belichtungsfolien verwendet wurden, von denen einige vom Nutzer mittels kleiner \emph{Symbolarme} angepasst werden konnten.
Diese Maschinen waren teilautomatisiert und übernamen Aufgaben wie Blocksatz oder Linespacing bereits selbst, einige konnten sogar vollautomatisch auf Disketten (\emph{Floppies}) gespeicherten Text setzen und en masse drucken.
Mathematik musste aber nach wie vor durch speziell ausgebildete Experten per Hand gesetzt werden.

Inspieriert von all diesen Systemen entwickelten sich schließlich \TeX{} und \LaTeX{}, die es Nutzern erstmals ermöglichen Mathematik am Computer zu spezifizieren und -- genauso wie normalen Text -- automatisch auszurichten und spacen zu lassen.
Das Ergebnis war, wie eingangs erwähnt, ein druckbares \filetype{dvi}-File.

\section{Grundlagen}
Um in \LaTeX{} Mathematik zu setzen, müssen wir in den sogeannten \emph{math mode} wechseln.
Dies geschieht mittels umgebenden Dollarzeichen:
\begin{latexlisting}
	Let $a$ be a number.
\end{latexlisting}
Man beachte den Unterschied zwischen den beiden a/$a$!
Nicht alle Zeichen sehen im Mathematikmodus anders aus, wie das folgende Beispiel zeigt (man vergleiche die 2 in 2000 mit der $2$ weiter hinten im Text):
\begin{latexlisting}
	The old greeks, more than 2000 years ago, already discussed wether the square root of $2$ was rational.
\end{latexlisting}
Trotzdem ist es aus semantischen Gründen immer sinnvoll, Variablen und mathematisch verwendete Zahlen im Mathemodus zu setzen, während z.B. Jahreszahlen als Textziffern gesetzt werden\footnote{Insbesondere mag auffallen, das je nach Schriftart eben doch ein Unterschied zwischen diesen Zahlen gemacht wird -- beispielsweise in diesem Dokument.}.
Ohne Modifikationen werden mathematische Formel außerdem kursiv gesetzt und verschlucken selbst einzelne Leerzeichen.
\begin{latexlisting}
	Dies ist ein Text. $Dies     ist auch Text und sollte nicht im Mathemodus stehen.$
\end{latexlisting}
Wer trotzdem etwas Raum haben möchte, kann die Commands \latexcommand{.}, \latexcommand{,}, \latexcommand{ }, \latexcommand{qquad} oder \latexcommand{qquad} oder das Sonderzeichen $\sim$ verwenden.
Außerdem funktionieren Trennung und Spacing im Mathematikmodus anders. So wird etwa \latexargument{sin} in
\begin{latexlisting}
	Eine trigonometrische Identität ist $sin(\pi) = 0$.
\end{latexlisting}
als \enquote{$s$ mal $i$ mal $n$} interpretiert, nicht als einzelnes Zeichen.
Es sollte stattdessen \latexcommand{operatorname\{sin\}} oder noch einfacher \latexcommand{sin} verwendet werden.
Besonderes Spacing wird auch für Operatoren verwendet:
\begin{latexlisting}
	Die einzige Lösung der Gleichung $2a - 23 = -5$ ist $a = 9$.
\end{latexlisting}
Man beachte, dass das Zeichen \key{-} nicht als Gedanken- oder Spiegelstrich, sondern als Minus interpretiert wird, und das Spacing um das unäre Minus bei $-5$ anders ist als das um $2a - 23$.
Dies ändert sich auch nicht, wenn die Leerzeichen in der Gleichung verändert werden -- es ist zwar gute \LaTeX{}-Praxis, Mathematik mit sinnvollen Leerzeichen zu schreiben, die den Code gut lesbar machen, aber es ist nicht zwingend notwendig.

Neben dieser sogenannten \emph{inline math} existiert auch \emph{display math}, d.h. Formeln, die abgehoben vom Text in einer eigenen Zeile gesetzt werden.
Die Begrenzer (engl. \emph{delimiters}) hierfür sind Doppeldollarzeichen:
\begin{latexlisting}
	Die Dreiecksungleichung $$a + b > c$$ ist in der nicht-euklidischen Geometrie manchmal falsch.
\end{latexlisting}
Für gewöhnlich wird Display-Mathematik wie eine Umgebung in einer eigenen Zeile und ein wenig eingerückt geschrieben:
\begin{latexlisting}
	Die Dreiecksungleichung
	$$
		a + b > c
	$$
	ist in der nicht-euklidischen Geometrie manchmal falsch.
\end{latexlisting}
Die Delimitation durch Dollarzeichen ist der Standard in \TeX{}.
\LaTeX{} führte stattdessen die Begrenzer \latexcommand{(}, \latexcommand{)} sowie \latexcommand{[}, \latexcommand{]} ein, also
\begin{latexlisting}
	Die einzige Lösung der Gleichung \(2a - 23 = -5\) ist \(a = 9\).
	Die Dreiecksungleichung
	\[
		a + b > c
	\]
	ist in der nicht-euklidischen Geometrie manchmal falsch.
\end{latexlisting}
und die Package \latexpackage{amsmath} (für \emph{American Mathematical Society}) stellt noch einige weitere Möglichkeiten, Displaymathematik zu begrenzen, zur Verfügung.
\begin{latexlisting}
	Die Dreiecksungleichung
	\begin{equation}
		a + b > c
	\end{equation}
	ist in der nicht-euklidischen Geometrie manchmal falsch.

	Die Dreiecksungleichung
	\begin{equation*}
		a + b > c
	\end{equation*}
	ist in der nicht-euklidischen Geometrie manchmal falsch.
\end{latexlisting}
Die Umgebung \latexenvironment{equation} ist dabei wie \latexcommand{[}, \latexcommand{]} einfach \enquote{nur} Displaymathematik ohne zusätzliche Funktionalitäten (tatsächlich definiert die \latexpackage{amsmath}-Package sogar \latexcommand{[}, \latexcommand{]} durch \latexenvironment{equation} um, da die Implementierung in \latexpackage{amsmath} leicht besseres Spacing hat).
Umgebungen ohne Sternchen erhalten in \latexpackage{amsmath} stets eine Nummerierung, solche mit Sternchen werden durchnummeriert.
Eine \latexenvironment{equation}-Umgebung erlaubt generell nur eine Zeile und schießt, falls die Formel zu lange ist, über den Rand hinaus.
Wer einen Zeilenumbruch in seiner Formel möchte, kann die \latexenvironment{gather}-Umgebung verwenden.
\begin{latexlisting}
	Wir lösen die Gleichung:
	\begin{gather*}
		2a + 7 = 25 \\
		2a = 18\\
		a = 9
	\end{gather*}
\end{latexlisting}
Auch diese existiert in einer gesternten und ungesternten Version, und -- ähnlich wie Tabellen -- sind hier Zeilenumbrüche mit \latexcommand{\textbackslash} sogar erwünscht.
Möchte man die verschiedenen Zeilen noch aneinander ausrichten, so bietet sich die \latexenvironment{align}-Umgebung an.
\begin{latexlisting}
	Wir berechnen:
	\begin{gather*}
		2a + 7(a + 5) - 13a + 12
		&= 2a + 7a + 35 - 13a + 12 \\
		&= (2 + 7 - 13)a + 35 + 12 \\
		&= -4a + 47
	\end{gather*}
\end{latexlisting}
Wie in Tabellen werden \key{\&}-Zeichen zur Spaltentrennung verwendet.
Hierbei sind die Spalten abwechselnd rechts- und linksbündig ausgerichtet, was für \enquote{Kommentarspalten} verwendet werden kann.
\begin{latexlisting}
	Wir berechnen:
	\begin{gather*}
		7a + 5
		&= 3a - 7
		&| -5 \\
		7a 
		&= 3a - 12
		&| -3a \\
		4a
		&= -12
		&| :4 \\
		a
		&= -3
	\end{gather*}
\end{latexlisting}
Es ist hierbei -- gerade bei längeren Zeilen -- guter Stil, jede Spalte in eine eigene Zeile zu setzen.

Wie wählen wir nun zwischen all diesen Umbegungen aus?
Für \emph{inline math} wäre theoretisch \latexcommand{(}, \latexcommand{)} die optimale Wahl, praktisch hat sich aber aufgrund des geringeren Tippaufwands und der typografischen Äquivalenz die Verwendung von \key{\$} durchgesetzt.
Für \emph{display math} gilt dies nicht!
Das alte \key{\$\$} hat merklich schlechteres Spacing und sollte \emph{niemals} verwendet werden!
Stattdessen sind immer die passenden \latexpackage{amsmath}-Commands zu nutzen, \latexenvironment{align} und \latexenvironment{gather} wo nötig und \latexenvironment{equation} überall sonst (natürlich je nach Wunsch in der unnummerierten Version).

\section{Sehr viel Syntax}
In diesem Unterkapitel besprechen wir eine lange Liste an Befehlen und Schreibweisen, die wir zum Setzen von Mathematik benötigen.

\subsection{Hoch- und Tiefstellung}
Hoch- und Tiefstellung erreichen wir durch die Symbole \enquote{~\key{\^}} und \enquote{\key{\_}}.
\begin{latexlisting}
	Meine Lieblings-Quadratzahl ist $q_2 = 2^2 = 4$.
\end{latexlisting}
Wir können diese auch an einem einzigen Symbol kombinieren.
\begin{latexlisting}
	Das Zeichen für Beryllium ist $Be^7_4$, oder besser $\mathrm{Be}^7_4$.
\end{latexlisting}
Es gibt hier keine kanonische Reihenfolge von Hoch- und Tiefstellung, aber es lohnt sich, zumindest selbst konsistent zu sein.
Mehrere Hochstellungen am selben Symbol hingegen erweisen sich als problematisch (und auch uneindeutig):
\begin{latexlisting}
	Berechne $2^2^3$. Ist es $4^3 = 64$ oder $2^8 = 256$?
\end{latexlisting}
Hier können teilweise Klamern helfen:
\begin{latexlisting}
	Ich meine natürlich $(2^2)^3 = 64$.
\end{latexlisting}
Das funktioniert aber nur, solange nur ein Zeichen hochgestellt werden soll, das folgende sieht etwas seltsam aus.
\begin{latexlisting}
	Oh, meinte doch $2^(2^3)$.
\end{latexlisting}
Mehrere Zeichen auf einmal können stattdessen mittels einer \emph{Gruppe}, die durch \key{\{} und \key{\}} begrenzt wird, hochgestellt werden:
\begin{latexlisting}
	Oh, meinte doch $2^{2^3} = 256$, oder noch klarer $2^{(2^3)} = 256$.
\end{latexlisting}
Die gleichen Techniken funktionieren natürlich auch für Tiefstellung, und können auch genutzt werden, um Hoch- und Tiefstellung verschieden zu kombinieren:
\begin{latexlisting}
	Betrachte das $4$-te Folgendglied, $a_{2^2}$.
\end{latexlisting}

\subsection{Schriftarten}
Um mathematische Symbole voneinander abzugrenzen, ist es oft üblich, sie in speziellen Schriftarten zu schreiben, allen voran die reellen Zahlen.
Die Package \latexpackage{amssymb} stellt hierzu einige nette Commands bereit.
\begin{latexlisting}
    Die reellen Zahlen sind $\mathbb{R}$.
    Die Menge aller reellwertigen $k$-mal differenzierbaren Funktionen auf einer Mannigfaltigkeit $M$ schreiben wir als $\mathcal{C}^k(M)$.
    Die Lie-Algebra der Lie-Gruppe $\mathscr{G}$ bezeichnen wir mit $\mathfrak{g}$.
\end{latexlisting}
Dies sind die Commands für \emph{Blackboard} (doppelgestrichene Schrift), \emph{caligraphy} (geschwungene Schrift), \emph{script} (anders geschwungene Schrift) sowie \emph{fraktur} (Frakturschrift).
Leider sind insbesondere in der \latexcommand{mathbb}-Schrift nicht alle Symbole enthalten -- beispielsweise fehlt die $\mathds{1}$.
Hierzu können wir stattdessen die Package \latexpackage{dsfont} und deren Command verwenden.
\begin{latexlisting}
	Das Symbol $\mathbb{1}$ gibt es nicht, $\mathds{1}$ aber schon.
\end{latexlisting}
Leider sind diese Commands semantisch nicht sehr wertvoll.
Sie kommunizieren nur, dass etwas doppelt gestrichen werden soll, nicht aber \emph{warum}.
Es wäre daher ratsam, eigene Commands definieren, die die obigen aliasen, aber semantisch wertvoll sind, etwa wie folgt\footnote{Der genaue Syntax der Commanddefinition wird später erklärt und muss hier nicht verstanden werden -- das Beispiel dient nur der Illustration.}:
\begin{latexlisting}
	\newcommand{\reals}{\mathds{R}}
	\newcommand{\field}[1]{\mathds{#1}}
	Die reellen Zahlen sind $\reals$ oder $\field{R}$.
\end{latexlisting}
Die Package \latexpackage{nchairx}, die hauseigen in Würzburg am Lehrstuhl 10 (Chair X) entwickelt wurde, nimmt einem einige dieser Arbeit ab und definiert semantisch sinnvolle mathematische Befehle (hauptsächlich für Differentialgeometrie, Analysis und lineare Algebra).
Dies erlaubt die Verwendung vieler semantischer Befehle.
\begin{latexlisting}
    Die reellen Zahlen sind $\field{R}$.
    Die Menge aller reellwertigen $k$-mal differenzierbaren Funktionen auf einer Mannigfaltigkeit $M$ schreiben wir als $\module{C}^k(M)$.
    Die Lie-Algebra der Lie-Gruppe $\group{G}$ bezeichnen wir mit $\liealg{g}$.
\end{latexlisting}

\subsection{Viele viele bunte Commands}
Das Setzen von Mathematik beruht auf der Verwendung vieler kurioser Zeichen.
Für all diese (oder, hoffentlich, die meisten) existiert ein entsprechendes \LaTeX{}-Command.
Wir werden in diesem Teilabschnitt eine Vielzahl der häufig verwendeten Commands durchgehen.
Sucht man jemals nach dem Command für ein Symbol und kennt es gerade nicht auswendig, kann die wundervolle Webseite \url{https://detexify.kirelabs.org/} verwendet werden, die eine Freihandzeichnung des Symbols nutzt, um entsprechende Commands herauszusuchen.
Alle Beispiele in diesem Abschnitt sollten in einer \latexenvironment{equation}-Umgebung gesetzt werden und benötigen die Packages \latexpackage{amsmath} und \latexpackage{amssymb}.

\paragraph{Mengen} Mengen und ihre Definition benötigen nur wenige mathematische Commands.
\begin{latexlisting}
	2\mathds{N} = \{ n \in \mathds{N} \mid \exists_{m \in N}: 2m = n \} \subsetneq \mathds{N} 
\end{latexlisting}
Ähnlich zum Existenzquantor \latexcommand{exists} existiert auch der Allquantor \latexcommand{all}, und analog zur echten Teilmenge \latexcommand{subsetneq} gibt es auch \latexcommand{subseteq}, \latexcommand{subset} sowie \latexcommand{supsetneq}, \latexcommand{supseteq} und \latexcommand{supset}.
Als Gegenstück zu \latexcommand{in} existiert außerdem \latexcommand{ni}.
Man beachte auch, dass der zentrale Strich \latexcommand{mid} der Menge anderes (passenderes) Spacing hat als einfach Verwendung von \key{|}.

\paragraph{Funktionen} Für das Setzen von Funktionen sind vor allem die Pfeile relevant, doch auch die Benutzung des \enquote{richtigen} Doppelpunktes statt \key{:} ist vielen \LaTeX{}-Nutzern leider nicht bekannt.
\begin{latexlisting}
	f\colon \mathds{R} \to \mathds{R}, x \mapsto x^2 
\end{latexlisting}

\paragraph{Pfeile} Neben diesen Pfeile existieren noch zahlreiche weitere, insbesondere \latexcommand{rightarrow}, \latexcommand{leftarrow}, \latexcommand{Rightarrow} und \latexcommand{Leftarrow}.
Diese sind natürlich semantisch nicht so sinnvoll wie die oben erwähnten.

\paragraph{Brüche} Brüche existieren in drei Varianten.
Man teste die folgenden Commands in Inline- und Displaymath und vergleiche insbesondere die Größe der entstehenden Brüche.
\begin{latexlisting}
	\frac{1}{12} = \dfrac{1}{3} - \tfrac{1}{4}
\end{latexlisting}
Brüche können natürlich beliebig verschachtelt werden.

\paragraph{Vergleiche} Das \enquote{Größer}- und \enquote{Kleiner}-Zeichen können einfach mittels ihrer Unicode-Zeichen \key{<} und \key{>} gesetzt werden, ebenso \key{=}.
Für die anderen Vergleiche existieren Commands.
\begin{latexlisting}
	\frac{1}{4} < \frac{1}{3} \leq a \neq b \geq 4 > 3
\end{latexlisting}

\paragraph{Summen} Eine Summe ist nicht nur ein Sigma-Zeichen, sondern kann mit Dekorationen versehen werden.
Ebenso existiert ein ähnlich funktionales Produkt- und Integralzeichen.
\begin{latexlisting}
	e = \sum_{k=0}^\infty \frac{1}{k!}
\end{latexlisting}
\begin{latexlisting}
	g_{ij} = \prod_{k=i+1}^p c_{k, j}
\end{latexlisting}
\begin{latexlisting}
	1 = \int_1^e \frac{1}{x} \D x
\end{latexlisting}
Das Command \latexcommand{D} für das Infinitesimale am Integralende benötigt insbesondere die Package \latexpackage{nchairx}, andernfalls muss es selbst definiert oder \latexcommand{mathrm\{d\}} verwendet wereden.

\paragraph{Limites} Auch Limites werden in Inline- und Displaymath unterschiedlich gesetzt.
\begin{latexlisting}
	e = \lim_{n \to \infty} (1 + \frac{1}{n})^n
\end{latexlisting}

\paragraph{Normen \& Skalarprodukt} Man mag versucht sein, Normen mit vielen \key{|} zu setzen, zu empfehlen ist stattdessen die Verwendung des passenden Commands.
\begin{latexlisting}
	\|x\| = \langle x, x \rangle
\end{latexlisting}
Einstrichige Beträge werden für gewöhnlich tatsächlich mit \key{|} gesetzt.

\paragraph{Trigonometrische Funktionen} Einige wichtige Funktionen existieren bereits als Commands.
\begin{latexlisting}
	\exp i x = \sin x + i \cos x
\end{latexlisting}
Alle anderen ähnlichen Operatoren sollten mit dem Befehl \latexcommand{operatorname} gesetzt werden, etwa
\begin{latexlisting}
	\operatorname{ker} A = \{x \in X \mid Ax = 0\}
\end{latexlisting}

\paragraph{Text} Manchmal ist es nötig, in einer mathematischen Formel oder Definition noch Fließtext einzubringen.
\begin{latexlisting}
	\tilde{g}_1 = \tilde{g} = g_1 \quad \text{ and }\quad \tilde{h}_{r,r} = \tilde{h} = h_{r,r}
\end{latexlisting}

\paragraph{Dekorationen} Oft möchte man denselben Buchstaben mehrmals verwenden.
Um solche Formelzeichen trotzdem zu unterscheiden, kann man sie \emph{dekorieren}.
\begin{latexlisting}
	a \neq \hat{a} \neq \tilde{a} \neq \overline{a}
\end{latexlisting}
Bei breiten Zeichen, oder solchen mit Index etc. sollten die expandierenden Formen dieser Dekorationen verwendet werden.
\begin{latexlisting}
	a_1 \neq \widehat{a_1} \neq \widetilde{a_1} \neq \overline{a_1}
\end{latexlisting}

\paragraph{Skalierende Klammern} Häufig kommt es vor, dass die vertikale Höhe eines Klammerinhalts größer als das Klammerzeichen ist.
Wir können Klammern dazu manuell vergrößern.
\begin{latexlisting}
	e = \lim_{n \to \infty} \bigl(1 + \frac{1}{n} \bigr)^n
\end{latexlisting}
Analog existieren die Command \latexcommand{Bigl}, \latexcommand{Bigr}, \latexcommand{biggl}, \latexcommand{biggr}, \latexcommand{Biggl} und \latexcommand{Biggr}.
Diese Commands funktionieren mit allen gängigen Klammern (\key{[}, \key{]}, \latexcommand{\{}, \latexcommand{\}}, \latexcommand{langle}, \latexcommand{rangle}, \latexcommand{\|}, \latexcommand{\|}, \key{|}, \dots)
Möchte man nicht manuell überlegen, welche Größe die Klammer nun haben sollte, kann man das auch von \LaTeX{} übernhemen lassen.
\begin{latexlisting}
	e = \lim_{n \to \infty} \left(1 + \frac{1}{n} \right)^n
\end{latexlisting}
Für Mengen, bei denen man dann meist auch den senkrechten Trennstrich vergrößern möchte, und ähnliche Fälle lässt sich diese \latexcommand{left}-\latexcommand{right}-Konstruktion noch um \latexcommand{middle} erweitern.
\begin{latexlisting}
	\mathds{N} = \left\{ \prod_{i=1}^n p_i \middle \mid p_i \in \mathds{P} \right\}
\end{latexlisting}

\paragraph{Griechische Buchstaben} Genügt einem das lateinische Alphabet nicht, kann man auf das griechische Zurückgreifen.
Was passiert, wenn man diese Befehle außerhalb des Mathematik-Modus verwendet?
\begin{latexlisting}
	\alpha^2 = \beta^2 + \gamma^2 \qquad \omega \neq \Omega
\end{latexlisting}
Einige dieser Buchstaben besitzen Varianten (insbesondere \latexcommand{epsilon} und \latexcommand{phi}), die mit dem Befehl \latexcommand{var<buchstabe>}, z.B. \latexcommand{varepsilon} erzeugt werden können.

\paragraph{Matrizen} Es existieren einige verschiedene Commands für Matrizen.
Das grundlegendste ist die Umgebung \latexenvironment{matrix}.
Der Syntax innerhalb einer Matrix entspricht dem von Tabellen.
\begin{latexlisting}
	\begin{matrix}
		a & b \\
		c & d	
	\end{matrix}
\end{latexlisting}
Klammern um die Matrix können entweder mittels \latexcommand{left} und \latexcommand{right} erzeugt werden, oder mittels der dedizierten Umgebung \latexenvironment{pmatrix}.
\begin{latexlisting}
	\left(\begin{matrix}
		a & b \\
		c & d	
	\end{matrix}\right)
	= 
	\begin{pmatrix}
		a & b \\
		c & d	
	\end{pmatrix}
\end{latexlisting}
Für andere Klammern existieren ähnliche Commands.


\subsection{Noch ein Wort zur Semantik}
Gut verständliches, typografisch richtiges Typesetting ist nicht immer einfach.
Optimalerweise sollten Operatoren wie $\ker A$, $\exp A$, $\D x$ und auch bestimmte Zeichen wie $\E$ und $\I$ aufrecht und mit anderen Spacing gesetzt werden.
Da es vielen \LaTeX{}-Nutzern zu aufwendig ist, stets 
\begin{latexlisting}
	\mathrm{e}^{\mathrm{i} x} = \operatorname{exp} \mathrm{i} x = \operatorname{sin} x + i \operatorname{cos} x
\end{latexlisting}
statt
\begin{latexlisting}
	e^{ix} = exp (i x) = sin(x) + cos(x)
\end{latexlisting}
zu schreiben, geht dies oft unter.
Hier können semantisch wertvolle Commands helfen. Die Befehle \latexcommand{exp}, \latexcommand{sin} und \latexcommand{cos} existieren bereits in \latexcommand{amsmath}, und die Package \latexpackage{nchairx} stellt noch viele weitere zur Verfügung.
\begin{latexlisting}
	\E^{\I x} = \exp \I x = \sin x + \cos x
\end{latexlisting}
\begin{latexlisting}
	\ker A = \{x \in X \mid Ax = 0\}
\end{latexlisting}
Eine vollständige Liste dieser semantischen Makros kann unter \url{https://satztexnik.com/tex-archive/macros/latex/contrib/nchairx/doc/nchairx.pdf} (ab Kapitel 3.4) abgerufen werden.
Ein Blick lohnt sich!

\section{Referenzieren von Gleichungen}

\section{Mathematische Umgebungen}

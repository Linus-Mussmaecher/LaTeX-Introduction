\chapter{Installation}
Bevor wir \LaTeX{} verwenden können, müssen wir es zuerst installieren.
Hier beginnen bereits die ersten Schwierigkeiten im Vergleich zu traditionellen Textverarbeitungsprogrammen: \LaTeX{} ist nicht einfach ein einziges Programm, sondern eine ganze Sammlung von Programmen, die auf unterschiedlichen Ebenen agieren.
Wir klären also zuerst ein paar Begrifflichkeiten.

\begin{table}[h]
	\begin{tabular}{>{\itshape}l p{8cm} >{\raggedright}p{3cm}}
		\toprule
		\emph{\textbf{Begriff}} & \textbf{Beschreibung} & \textbf{Beispiele} \tabularnewline
		\midrule
		\TeX{} &
		Grundlegendes Typesetting-Programm.
		Enthält Algorithmen für Zeilenumbrüche, Seitenumbrüche, Hyphenation (Setzung von Bindestrichen), Kerning, Linespacing, Positionierung von Bilder \& Tabellen, etc. & \tabularnewline
		Format &
		Makro-Sammlung, die \TeX{} auf nutzerfreundliche Art und Weise erweitert, beispielsweise durch Befehle für Kapitel und Abschnitte, Mathematikumgebungen, etc. &
		\LaTeX{}, ConTeXt, plainTeX, OpTeX \tabularnewline
		Package &
		Gebündelte Sammlung an Befehlen, die für bestimmte Zwecke geeignet sind. &
		\latexpackage{amsmath}, \latexpackage{booktabs}, \latexpackage{babel} \tabularnewline
		Engines &
		Programm, das \TeX{}- oder \LaTeX{}-Code einliest und zu einem Output-File (\filetype{pdf}, \filetype{dvi}, \filetype{html}) kompiliert.
		Manche Features stehen dabei nur bestimmten Engines zur Verfügung. &
		pdfTeX, XeTeX, LuaTeX, Tectonic \tabularnewline
		Distribution &
		Bündelt \TeX, eine oder mehrere Engines, Formate und eine grundlegende Sammlung an Packages mit weiteren Hilfsprogrammen wie latexmk und Package Managern. &
		TeXLive, MikTeX \tabularnewline
		Editoren &
		Programm, mit dem \filetype{tex}-Files bearbeitet werden und das eventuell auch Engines oder Hilfsprogramme aufrufen kann. &
		Overleaf, TeXStudio, VSCode, Emacs, Vim, Helix, Notepad \tabularnewline
		\bottomrule
	\end{tabular}
	\caption{Bestandteile von \LaTeX{}}
\end{table}

Wir verwenden im Folgenden den Begriff \LaTeX{} -- wie in der Umgangssprache üblich -- gleichzeitig für die Sprache, in der \filetype{tex}-Files verfasst werden, als auch für das Format und den Compiler.

Um \LaTeX{} also auf einem handelsüblichen PC installieren zu können, benötigen wir zwei Dinge -- eine Distribution und einen Editor.
Wir stellen im Folgenden zwei Möglichkeiten, \LaTeX{} zu installieren kurz vor.

\section{Overleaf}
\emph{Overleaf} bricht das obige Schema: Es ist kein Editor, der auf einem Computer installiert wird, sondern eine Web-Applikation.
Damit vereinigt Overleaf alle obigen Bestandteile in sich und es ist keine Installation notwendig -- nach Erstellung eines Accounts kann sofort losgelegt werden.
Außerdem ist es auch nicht nötig, die eigene \TeX{}-Installation zu warten und regelmäßig mit Updates zu versorgen, da sich der Overleaf-Server hier um alles kümmern.
Innerhalb von Overleaf können dann verschiedene Dokumente angelegt und bearbeitet werden, welches Format und welche Engine dabei verwendet wird lässt sich pro Projekt einzeln konfigurieren.
Die Online-Natur von Overleaf macht es außerdem leicht, \LaTeX{}-Projekte mit anderen zu teilen und gemeinsam an Ihnen zu arbeiten.

Auch die Nachteile von Overleaf basieren auf der Tatsache, dass es sich um eine Web-Anwendung handelt:
Hat man gerade keine Internetverbindung oder findet gerade eine Serverwartung statt, so kann Overleaf nicht verwendet werden\footnote{Dies geschah während einer Sitzung des Seminars im Wintersemester 2024/25.}.
Außerdem gibt man die Verantwortung über seine Files an eine dritte Partei ab.
Letztlich wird die Kompilation auf dem Overleaf-Server durchgeführt, dessen Ressourcen zwischen allen Nutzern geteilt werden, wobei nicht-zahlende Nutzer verständlicherweise benachteiligt werden.
Kompilation in Overleaf ist daher oft langsamer als lokal (insbesondere bei performancestarken Computern) und in seltenen Fällen auch gar nicht verfügbar.

Overleaf kann unter \url{https://www.overleaf.com} verwendet werden.

\section{TeXStudio \& MikTeX/TeXLive}
Eine traditionellere Methode zur Verwendung von \LaTeX{} ist die Kombination des Editors \emph{TeXStudio} mit einer betriebssystemangepassten \LaTeX{}-Distribution.
Dabei sind die folgenden Distributionen empfohlen:
\begin{table}[h]
	\begin{tabular}{l l}
		\toprule
		Linux & TeXLive \tabularnewline
		MacOS & TeXLive/MacTeX \tabularnewline
		Windows & MikTeX \tabularnewline
		\bottomrule
	\end{tabular}
	\caption{\LaTeX{}-Distributionen je Betriebssytem}
\end{table}

\subsection{TeXStudio}
TeXStudio ist ein speziell für \LaTeX{} entwickelter Editor und bietet eine grafische Benutzeroberfläche, die die wichtigsten Funktionalitäten und Konfigurationsoptionen beim Arbeiten mit \LaTeX{} kompakt darstellt.

TeXStudio kann unter \url{https://www.texstudio.org/#download} für alle oben genannten Betriebssysteme heruntergeladen werden.

\subsection{MikTeX}
MikTeX ist eine \LaTeX -Distribution, die wie oben beschrieben verschiedene Hilfsprogramme und Packages bereitstellt, die dann von TeXStudio verwendet werden.
MikTeX integriert nach der Installation nahtlos in TeXStudio und es ist selten nötig, direkt mit MikTeX zu interagieren.
Insbesondere bemerkt MikTeX während der Kompilierung von \LaTeX -Files, welche Packages benötigt werden und lädt diese on-demand herunter.

MikTeX kann unter \url{https://miktex.org/download} heruntergeladen werden.
Nach der Installation sollte MikTeX über den System Try gestartet werden, um dann alle Packages zu updaten.
Je nach Internetverbindung und PC-Leistung kann dies eine Weile dauern.
Das Updaten der Packages sollte regelmäßig (in etwa monatlich) oder falls Kompilierungsprobleme auftreten wiederholt werden.

\subsection{TeXLive}
Unter Linux kann TeXLive über den hoffentlich in der Linux-Distribution enthaltenen Package-Manager installiert werden und integriert dann ähnlich wie MikTeX nahtlos in TeXStudio (oder einen anderen \LaTeX -Editor).
TeXLive sollte dann vom Package Manager up-to-date gehalten werden und es ist keine weitere Interaktion notwendig.

Unter MacOS sollte die Mac-spezifische Variante MacTeX verwendet werden, die unter \url{https://www.tug.org/mactex/} zur Verfügung steht.
Dem Autor sind hier nähere Details nicht bekannt, der Leser mag, falls er Mac-User ist, diese gerne beisteuern.



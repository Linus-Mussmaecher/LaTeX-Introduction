\documentclass[
	11pt,
	% draft, % for easier editing
	ngerman,
]{scrbook}

\KOMAoptions{
  numbers=noenddot,
  headings=standardclasses,
  chapterprefix=false,
  fontsize=11pt
}

% === Packages ===
% --------
% Packages
% --------

\usepackage[
	ngerman,
	strings
]{babel}						% Languages

% ===      ===
% === Math ===
% ===      ===

\usepackage{amsmath,amssymb}    % more math symbols
\usepackage{mathtools}          % more math symbols
\usepackage{dsfont}             % double stroked math symbols
\usepackage{amsfonts}           % blackboard math symbols
\usepackage{latexsym}		    % more math symbols
\usepackage{chngcntr}		    % more math symbols
\usepackage{mathrsfs}		    % math-fonts
\usepackage{cancel}             % cancel lines for fractions
\usepackage[luatex]{nchairx}    % additional packages
\usepackage{exscale}            % large summation signs in 11pt
\usepackage{leftindex}          % for indeces to the left of symbols

% ===          ===
% === Graphics ===
% ===          ===

\usepackage{graphicx}           % Include images
\usepackage{tikz}               % for commutative diagrams and stuff
\usetikzlibrary{patterns}
\usetikzlibrary{cd}
\usepackage[backend=biber,backref=true]{biblatex}

% ===         ===
% === Figures ===
% ===         ===

\usepackage{floatrow}           % Centered Figures etc.
\usepackage{subfig}             % Separate captioning of multiple figures in one environment
\usepackage[margin    =20pt,
            font      ={footnotesize},
            format    =plain,
            labelsep  =period
]{caption}          % Nicer looking captions

% ===                 ===
% === Fonts & Symbols ===
% ===                 ===

% Font related packages
\usepackage[T1]{fontenc}   % Allows usage of custom fonts and improves encoding
% \usepackage[sc,osf]{mathpazo}   % Custom palatino font with smallcaps and old style figures
% \usepackage[scaled=.8]{beramono}% Custom monospace font
\usepackage{fontspec}             % Allows using custom fonts
\usepackage[
	% Turn of certain warnings when using mathtools with unicode-math
	warnings-off={mathtools-colon,mathtools-overbracket},
	% Inform unicode-math we are using an upright math font (euler math)
	math-style=upright
]{unicode-math}                   % Allows input and output of math as unicode characters.
% Font customization
\defaultfontfeatures{Ligatures={TeX}}
\setmainfont{Rosario}
\setmainfont{TeX Gyre Termes}
\setmathfont{Euler Math}[Scale=MatchLowercase]
\setmonofont{JetBrains Mono}[Scale=MatchLowercase]
% Color
\usepackage[dvipsnames]{xcolor} % Colors
\definecolor{codegreen}{rgb}{0,0.6,0}
\definecolor{codegray}{rgb}{0.5,0.5,0.5}
\definecolor{codepurple}{rgb}{0.58,0,0.82}
\definecolor{linkblue}{RGB}{58, 103, 181}
\definecolor{backcolour}{rgb}{0.95,0.95,0.92}
\definecolor{foldercolor}{RGB}{124,166,198}
% Symbols
\usepackage{csquotes}           % Prettier quote signs
\usepackage{ellipsis}           % Fixes spacing of ellipsis
\usepackage{fontspec}           % German special characters with luatex. Needs to be after mathpazo.
\usepackage[
	colorlinks=true,
	linkcolor=black,
	urlcolor=linkblue,
	final=true,                 % always treat as final
	pdfpagelabels,              % use pdf page labels
]{hyperref}                     % hyperrefs are cool!
\usepackage[
	nameinlink
]{cleveref}                     % Nicer automatic references
\usepackage{xurl}               % Better link inclusions

% ===                        ===
% === Technical & Formatting ===
% ===                        ===

\usepackage[footsepline, draft=false]{scrlayer-scrpage}
\usepackage[babel]{microtype}   % Nicer text blocks
\DeclareMicrotypeAlias{Euler Math}{TU-basic} % With unicode math, math is rendered in TU font encoding, where Euler math does not support some of the symbols.
\usepackage{xspace}             % better spacing after macros
\usepackage{marginnote}         % Used to put text into the margins.
\usepackage{gitinfo2}           % show info about the git
\usepackage{ifdraft}            % to determine whether draft mode
\usepackage{titling}            % allows use of thetitle
\usepackage[cleanlook]{isodate} % Prettier date
\usepackage{array,booktabs}     % Improved table formatting options
\newcolumntype{L}[1]{>{\raggedright}p{#1}}
\usepackage{enumitem}           % Better formatting for enumerations
\setlist{nosep}
\setlist[enumerate,1]{label=(\roman*)}
\setlist[enumerate,2]{label=(\alph*)}
\usepackage{silence}           % Disable certain warnings
\WarningFilter{latex}{Unused label}
\WarningFilter{latex}{Unused entry}
\WarningFilter{latex}{Marginpar on page}
\WarningFilter{hypdoc}{hyperref}

% ===                ===
% ===      Code      ===
% ===                ===

\usepackage{listings}		% Allow code blocks in text
\usepackage{lstautogobble}  % Gobbles leading whitespace
\lstnewenvironment{latexlisting}{
\lstset{
    language=[LaTeX]TeX,
    backgroundcolor=\color{backcolour},   
    commentstyle=\color{codegreen},
    keywordstyle=\color{codepurple},
    numberstyle=\tiny\color{codegray},
    stringstyle=\color{codepurple},
    basicstyle=\ttfamily\footnotesize,
	columns=fullflexible,
    breakatwhitespace=false,         
    breaklines=true,                 
    keepspaces=true,                 
    numbers=none,       
    numbersep=5pt,                  
    showspaces=false,                
    showstringspaces=false,
    showtabs=false,                  
    tabsize=4,
    autogobble,
}
}{}
\lstset{
    backgroundcolor=\color{backcolour},   
    commentstyle=\color{codegreen},
    keywordstyle=\color{codepurple},
    numberstyle=\tiny\color{codegray},
    stringstyle=\color{codepurple},
    basicstyle=\ttfamily\footnotesize,
	columns=fullflexible,
    breakatwhitespace=false,         
    breaklines=true,                 
    keepspaces=true,                 
    numbers=none,       
    numbersep=5pt,                  
    showspaces=false,                
    showstringspaces=false,
    showtabs=false,                  
    tabsize=4,
    autogobble,
}
% ===                ===
% === Other Packages ===
% ===                ===

\usepackage{blindtext}          % Simple blindtext for typography testing
\usepackage{fixme}
\FXRegisterAuthor{lm}{alm}{Linus}
\fxsetup{theme=colorsig}

% Folderpaths
\usepackage[edges]{forest}
\tikzset{pics/folder/.style={code={%
    \node[inner sep=0pt, minimum size=#1](-foldericon){};
    \node[folder style, inner sep=0pt, minimum width=0.3*#1, minimum height=0.6*#1, above right, xshift=0.05*#1] at (-foldericon.west){};
    \node[folder style, inner sep=0pt, minimum size=#1] at (-foldericon.center){};}
    },
    pics/folder/.default={20pt},
    folder style/.style={draw=foldercolor!80!black,top color=foldercolor!40,bottom color=foldercolor}
}

\forestset{is file/.style={edge path'/.expanded={%
        ([xshift=\forestregister{folder indent}]!u.parent anchor) |- (.child anchor)},
        inner sep=1pt},
    this folder size/.style={edge path'/.expanded={%
        ([xshift=\forestregister{folder indent}]!u.parent anchor) |- (.child anchor) pic[solid]{folder=#1}}, inner xsep=0.6*#1},
    folder tree indent/.style={before computing xy={l=#1}},
    folder icons/.style={folder, this folder size=#1, folder tree indent=3*#1},
    folder icons/.default={12pt},
}


% === Layout ===
% Taken from Florian Kranhold and Philipp Schwarz.

% ----- 
% Utility Commands
% -----

\usepackage{accsupp}
\makeatletter
% Command that displays the text given in it's second argument, but when copying text out of the pdf-file, puts the first argument into the clipboard.
\newcommand*{\@changeClipboardCopy}[2]{\texorpdfstring{%
    \protect\BeginAccSupp{%
      method=pdfstringdef,%
      ActualText=#1%
    }%
    #2%
    \protect\EndAccSupp{}}%
  {#1}%
}

% ------
% Special fonts
% ------

% Font commands
\newcommand*{\name}[1]{\@changeClipboardCopy{#1}{\textsc{\textls[40]{\MakeLowercase{#1}}}}}
\newcommand*{\tableHead}[1]{\@changeClipboardCopy{#1}{\textsc{\textls[60]{\MakeLowercase{#1}}}}}

% small caps with different spacing
\SetTracking[spacing={100*,166,}]{encoding=*,shape=ssc}{15}
\newcommand*{\SAC}[1]{\@changeClipboardCopy{#1}{\textls{\MakeUppercase{#1}}}}    % spaced all caps
\newcommand*{\SLSC}[1]{\@changeClipboardCopy{#1}{\textssc{#1}}}  % spaced lowercase small caps
\makeatother

% -----
% Headers
% ---

% Paragraph
%\setkomafont{paragraph}{\sscshape\mdseries}

% Sections
\addtokomafont{section}{\large\bfseries}
\renewcommand*{\sectionformat}{\upshape\bfseries\thesection\enskip\hspace*{2px}}
\RedeclareSectionCommand[afterskip=1\baselineskip plus 2pt minus 2pt]{section}

% Subsections
\addtokomafont{subsection}{\normalsize\mdseries\scshape}
\renewcommand*{\subsectionformat}{\upshape\mdseries\scshape\thesubsection\enskip}
\RedeclareSectionCommand[beforeskip=2\baselineskip,afterskip=1.0\baselineskip plus 2pt minus 2pt]{subsection}

% Chapter
\RedeclareSectionCommand[beforeskip=0\baselineskip,afterskip=1\baselineskip,afterindent=false]{chapter}
\addtokomafont{chapter}{\mdseries\normalsize\lsstyle}
\renewcommand*{\chapterformat}{\thechapter}%\normalsize\textssc{kapitel \thechapter}}
\renewcommand{\chapterlinesformat}[3]{%
  \MakeUppercase{#3}\par%
  \marginnote{\smash{\fontsize{70}{60}\selectfont #2}}
  \rule[-.15\baselineskip]{\linewidth}{.5pt}\par\nobreak
  \thispagestyle{scrheadings}
}%

% ------
% Page Design
% ------
% \pagestyle{plain}

% page size
%\KOMAoptions{DIV=10}
% \areaset{288pt}{480pt}%
% \setlength{\marginparwidth}{8em}%
% \setlength{\marginparsep}{1.5em}%
% \setlength{\footskip}{2.5\baselineskip}

% footnote
% \usepackage{footmisc}
% \setlength{\footnotemargin}{0em}
% \deffootnote{0em}{0em}{%
%   \textsuperscript{\normalfont\thefootnotemark\ }}

% header and footer
\makeatletter
\ihead{}
\ohead{}
\chead{}
\ifoot{\normalfont\footnotesize\leftmark}
\cfoot{}
\ofoot{%
%\normalfont\footnotesize \gitAuthorIsoDate{} Hash: \texttt{\gitAbbrevHash}
\normalfont\thepage
}
\makeatother


% === Macros ===
% --------
% Mathematical Macros
% --------

% Logical operators
\newcommand{\then}{\ensuremath{\Rightarrow}}
\newcommand{\since}{\ensuremath{\Leftarrow}}
\renewcommand{\iff}{\ensuremath{\Leftrightarrow}}


% --------
% Other Macros
% --------
\newcommand{\latexcommand}[1]{\texttt{\textbackslash #1}}


% === Bibliography ===
\addbibresource{bibliography.bib}

% === Document ===

\begin{document}

	\frontmatter

	% Titelseite
\begin{titlepage}
  \centering\large
  ~
  \vfill  
  \textcolor{Maroon}{\Huge\SAC{Eine Einführung in \LaTeX{}}} \\ \medskip
  \SLSC{Linus Mußmächer}
  \vfill
  \vspace{5cm}
  \vfill
  Skript für ein Seminar im Rahmen des MINT+-Programms\\
  Julius-Maximilians-University Würzburg\\
  Würzburg, 2025
  \vfill
\end{titlepage}

\thispagestyle{empty}%
~%
\vfill%
% Additional comment, thanks, etc.
\noindent Würzburg, 2025
\cleardoublepage

%%% Local Variables:
%%% mode: latex
%%% TeX-master: "main"
%%% End:


	\chapter{Vorwort}

	Das vorliegende Skript bietet eine Einführung in die Grundlagen der Typesetting Engine \LaTeX{} für Studenten ohne Vorerfahrung in \LaTeX{} oder anderen Systemen wie \TeX{} oder \texttt{Typst}.
	Es ist im Rahmen des Seminars \enquote{Erstellen wissenschaftlicher Arbeiten in \LaTeX{}} an der Julius-Maximilians-Universität Würzburg im Sommersemester 2025 entstanden und basiert unter anderem auf Vorerfahrung aus der gleichnamigen Veranstaltung im vorausgehenden Semester.

	Diskutiert werden sämtliche Voraussetzungen zur Erstellung einer wissenschaftlichen Arbeit, z.B.\ einer Abschlussarbeit, in \LaTeX{}, von der Installation der benögtiten Programme über grundlegende Befehle bis hin zu tieferen Themen wie Bibliographien und eigenen nutzerdefinierten Commands.
	Der Leser wird freundlich aufgefordert, die zahlreichen Codebeispiele und Aufgaben in diesem Text selbständig zu komplieren, auszuprobieren und zu bearbeiten.

	Neben den techischen Informationen enthält dieses Dokument auch eine Reihe an Nutzungsempfehlungen, Best Practices und typographischen Anmerkungen.
	Besucher des Seminares werden den Begriff \enquote{Semantik} zu schätzen wissen.

	Eventuelle Fehler, Wünsche oder Anmerkungen können an den Autor, Linus Mußmächer (\texttt{linus.mussmaecher@gmail.com}), gemeldet werden.
	Der Quellcode dieses Skript steht unter \url{https://github.com/Linus-Mussmaecher/LaTeX-Introduction} zur Verfügung (wo sich, Git sei dank, leider auch einsehen lässt zu welch zweifelhaften Uhrzeiten dieses Skript seine Entstehung fand).

	\chapter{Einführung}\label{sec:introduction}

\paragraph{Eine kurze Historie}
Die Vorform von \LaTeX{}, \TeX{}, entstand, als der amerikanische Mathematikprofessor Donald Knuth sein -- inzwischen auf mehrere Volumen und viele tausend Seiten angewachsenes -- Monumentalwerk \emph{The Art of Computer Programming} in den Druck geben lassen wollte und mit dem \emph{Schriftsatz}, d.h. der Art und Weise wie sein Text vom Herausgeber auf den Seiten angeordet und formatiert wurde, höchst unzufrieden war.
Typografie hatte damals erst kürzlich den Schritt von manuellem Layouten auf den Computer gemacht und steckte noch in den Kinderschuhen.
Knuth begann daher, das Programm \TeX{} zu entwickeln, das eben genausolche Aufgaben -- wie beispielsweise die Positionierung von Zeilenumbrüchen und das Strecken und Stauchen von Leerzeichen zum Erreichen eines schönen Blocksatzes -- übernimmt und aus einem \filetype{tex}-File voll mit dem Text des zu druckenden Werkes so wie vielen Anweisungen, wie denn dieser Text nun zu formatieren ist, ein fertiges druckbares Dokument erstellt.
Damals war der Output von \TeX{} noch ein \filetype{dvi}-File, im Laufe der Zeit wurde \TeX{}, das inzwischen weite Verbreitung unter amerikanischen und europäischen Akademikern gefunden hatte, aber immer weiter entwickelt und der Output-Standard wandelte langsam zu \filetype{pdf}.

Insbesondere Leslie Lamport veröffentlichte eine Sammlung von \TeX{}-Macros, Konfigurationen und Hilsfprogrammen, die so populär wurde, dass sie unter dem Namen Lamport-\TeX{}, kurz \LaTeX{}, heute den Standard darstellt.
Donald Knuth selbst arbeitet auch heute noch am \TeX{}-Kern, der immer noch tief in \LaTeX{} enthalten ist, weiter.

\paragraph{\LaTeX{} vs. WYSIWIG-Editoren}
\LaTeX{} unterscheidet sich in der Bedienung deutlich von den weiter verbreiteten Texteditoren wie Microsoft Word, Libre Office Writer, Google Docs oder dem großartigen Word Star.
Diesen Programmen basieren auf der WYSIWIG-Methode (= What You See Is What You Get), bei der Nutzer direkt Modifikationen in einem interaktiven, grafisch dargestellten Dokument vornimmt, das bereits genauso aussieht wie das finale Produkt und direkt gedruckt oder auch als \filetype{pdf} exportiert werden kann.

In \LaTeX{} hingegen modifiziert der Nutzer eines oder mehrere \filetype{tex}-Dokumente, die nur reinen Text enthalten und mit jedem beliebigen Text-Editor bearbeitet werden können.
Dieser Text besteht zum Einen aus dem Inhalt des Dokuments, zum anderen aus sogenannten \emph{Commands}, die \LaTeX{} mitteilen, wie der Text später formatiert werden soll. Das Command \latexcommand{section} beispielsweise markiert einen neuen Abschnitt und veranlasst \LaTeX{} zum Einfügen von Abständen und dazu, den Titel des Abschnitts fett zu drucken, zu nummerieren und ins Inhaltsverzeichnis aufzunehmen.


\paragraph{Vorteile der \LaTeX{}-Methode}
\begin{itemize}
	\item Style und Inhalt werden getrennt und können unabhängig voneinander modifiziert und weiterverwendet werden.
	\item Klare semantische Markierung von Textelementen.
	\item Große Erweiterbarkeit und Anpassbarkeit.
	\item Kein festes Programm, sondern anpassbare Sammlung.
	\item Simples und zeitloses Speicherformat.
	\item Kostenlos \& Open Source, unabhängig von Konzernpolitik oder ungewollten Programmupdates.
	\item Bessere Perfomance, gerade bei längeren Dokumenten.
	\item Mehr Fähigkeiten, mehr Features.
	\item Professionelleres Typografie-Algorithmen.
\end{itemize}

\paragraph{Vorteile von WYSIWIG}
\begin{itemize}
	\item Einfache Installation.
	\item Einfache Bedienung.
	\item Keine technischen Kenntnisse notwendig.
\end{itemize}

~

Der Leser mag bemerken, dass der Autor nicht ganz unvoreingenommen ist.

\paragraph{Ein Wort zur Semantik}
Ein Thema, dass in diesem Skript und im Seminar immer wieder aufkommt ist die \emph{Semantik}.
Was ist hiermit gemeint?
Wer in Microsoft Word eine Zeile als Überschrift hervorheben will, markiert das Wort mit seiner Maus und verändert manuell die Schriftgröße und vielleicht den Fettheitsgrad\footnote{Oder kennt sich etwas besser mit Word aus und verwendet Formatvorlagen.}.
In \LaTeX{} hingegen markieren wir die Zeile etwa wie folgt:
\begin{latexlisting}
	\section{Ein Wort zur Semantik}
\end{latexlisting}
Statt also das Format direkt einzustellen, teilen wir \LaTeX{} semantisch mit, um welche Art von Text es sich hier handelt (eine Überschrift).
An einer anderen Stelle wird dann definiert, wie genau eine Überschrift in unserem Dokument aussehen soll (beispielsweise fett, zentriert, nummeriert, etc.).
Aus ähnlichen Gründen stellen wir Worte in \LaTeX{} auch nicht einfach kursiv (mit dem Command \latexcommand{textit} für \emph{italic}), sondern markieren, ob wir ein Wort beispielsweise einfach hervorherben wollen (\latexcommand{emph} für \emph{emphasize}) oder es sich um ein Zitat (\latexcommand{quote}) handelt.
\LaTeX{} übernimmt dann die Formatierung.
Diese mag kursiv sein, aber falls wir später beispielweise entscheiden, dass wir Hervorhebungen doch gerne unterstrichen hätten\footnote{Bitte nicht.}, so können wir die Bedeutung von \latexcommand{emph} zentral ändern -- und schon sind alle Hervorhebungen unterstrichen, ohne dass wir einzeln durch das Dokument gehen müssen, während alle Zitate weiterhin kursiv sind.


	\setcounter{tocdepth}{1}
	\tableofcontents

	\mainmatter

	\chapter{Installation}
Bevor wir \LaTeX{} verwenden können, müssen wir es zuerst installieren.
Hier beginnen bereits die ersten Schwierigkeiten im Vergleich zu traditionellen Textverarbeitungsprogrammen: \LaTeX{} ist nicht einfach ein einziges Programm, sondern eine ganze Sammlung von Programmen, die auf unterschiedlichen Ebenen agieren.
Wir klären also zuerst ein paar Begrifflichkeiten.

\begin{table}[h]
	\begin{tabular}{>{\itshape}l p{8cm} >{\raggedright}p{3cm}}
		\toprule
		\emph{\textbf{Begriff}} & \textbf{Beschreibung} & \textbf{Beispiele} \tabularnewline
		\midrule
		\TeX{} &
		Grundlegendes Typesetting-Programm.
		Enthält Algorithmen für Zeilenumbrüche, Seitenumbrüche, Hyphenation (Setzung von Bindestrichen), Kerning, Linespacing, Positionierung von Bilder \& Tabellen, etc. & \tabularnewline
		Format &
		Makro-Sammlung, die \TeX{} auf nutzerfreundliche Art und Weise erweitert, beispielsweise durch Befehle für Kapitel und Abschnitte, Mathematikumgebungen, etc. &
		\LaTeX{}, ConTeXt, plainTeX, OpTeX \tabularnewline
		Package &
		Gebündelte Sammlung an Befehlen, die für bestimmte Zwecke geeignet sind. &
		\latexpackage{amsmath}, \latexpackage{booktabs}, \latexpackage{babel} \tabularnewline
		Engine &
		Programm, das \TeX{}- oder \LaTeX{}-Code einliest und zu einem Output-File (\filetype{pdf}, \filetype{dvi}, \filetype{html}) kompiliert.
		Manche Features stehen dabei nur bestimmten Engines zur Verfügung. &
		pdfTeX, XeTeX, LuaTeX, Tectonic \tabularnewline
		Distribution &
		Bündelt \TeX, eine oder mehrere Engines, Formate und eine grundlegende Sammlung an Packages mit weiteren Hilfsprogrammen wie latexmk und Package Managern. &
		TeXLive, MikTeX \tabularnewline
		Editor &
		Programm, mit dem \filetype{tex}-Files bearbeitet werden und das eventuell auch Engines oder Hilfsprogramme aufrufen kann. &
		Overleaf, TeXStudio, VSCode, Emacs, Vim, Helix, Notepad \tabularnewline
		\bottomrule
	\end{tabular}
	\caption{Bestandteile von \LaTeX{}}
\end{table}

Wir verwenden im Folgenden den Begriff \LaTeX{} -- wie in der Umgangssprache üblich -- gleichzeitig für die Sprache, in der \filetype{tex}-Files verfasst werden, als auch für das Format und den Compiler.

Um \LaTeX{} also auf einem handelsüblichen PC installieren zu können, benötigen wir zwei Dinge -- eine Distribution und einen Editor.
Wir stellen im Folgenden zwei Möglichkeiten, \LaTeX{} zu installieren kurz vor.
Die Beschreibungen des Skripts nehmen im Weiteren generell die Verwendung von TeXStudio an.

\section{Overleaf}
\emph{Overleaf} bricht das obige Schema: Es ist kein Editor, der auf einem Computer installiert wird, sondern eine Web-Applikation.
Damit vereinigt Overleaf alle obigen Bestandteile in sich und es ist keine Installation notwendig -- nach Erstellung eines Accounts kann sofort losgelegt werden.
Außerdem ist es auch nicht nötig, die eigene \TeX{}-Installation zu warten und regelmäßig mit Updates zu versorgen, da sich der Overleaf-Server hier um alles kümmern.
Innerhalb von Overleaf können dann verschiedene Dokumente angelegt und bearbeitet werden, welches Format und welche Engine dabei verwendet wird lässt sich pro Projekt einzeln konfigurieren.
Die Online-Natur von Overleaf macht es außerdem leicht, \LaTeX{}-Projekte mit anderen zu teilen und gemeinsam an Ihnen zu arbeiten.

Auch die Nachteile von Overleaf basieren auf der Tatsache, dass es sich um eine Web-Anwendung handelt:
Hat man gerade keine Internetverbindung oder findet gerade eine Serverwartung statt, so kann Overleaf nicht verwendet werden\footnote{Dies geschah während einer Sitzung des Seminars im Wintersemester 2024/25.}.
Außerdem gibt man die Verantwortung über seine Files an eine dritte Partei ab.
Letztlich wird die Kompilation auf dem Overleaf-Server durchgeführt, dessen Ressourcen zwischen allen Nutzern geteilt werden, wobei nicht-zahlende Nutzer verständlicherweise benachteiligt werden.
Kompilation in Overleaf ist daher oft langsamer als lokal (insbesondere bei performancestarken Computern) und in seltenen Fällen auch gar nicht verfügbar.

Overleaf kann unter \url{https://www.overleaf.com} verwendet werden.

\section{TeXStudio \& MikTeX/TeXLive}
Eine traditionellere Methode zur Verwendung von \LaTeX{} ist die Kombination des Editors \emph{TeXStudio} mit einer betriebssystemangepassten \LaTeX{}-Distribution.
Dabei sind die folgenden Distributionen empfohlen:
\begin{table}[h]
	\begin{tabular}{l l}
		\toprule
		Linux & TeXLive \tabularnewline
		MacOS & TeXLive/MacTeX \tabularnewline
		Windows & MikTeX \tabularnewline
		\bottomrule
	\end{tabular}
	\caption{\LaTeX{}-Distributionen je Betriebssytem}
\end{table}

\subsection{TeXStudio}
TeXStudio ist ein speziell für \LaTeX{} entwickelter Editor und bietet eine grafische Benutzeroberfläche, die die wichtigsten Funktionalitäten und Konfigurationsoptionen beim Arbeiten mit \LaTeX{} kompakt darstellt.

TeXStudio kann unter \url{https://www.texstudio.org/#download} für alle oben genannten Betriebssysteme heruntergeladen werden.

\subsection{MikTeX}
MikTeX ist eine \LaTeX -Distribution, die wie oben beschrieben verschiedene Hilfsprogramme und Packages bereitstellt, die dann von TeXStudio verwendet werden.
MikTeX integriert nach der Installation nahtlos in TeXStudio und es ist selten nötig, direkt mit MikTeX zu interagieren.
Insbesondere bemerkt MikTeX während der Kompilierung von \LaTeX -Files, welche Packages benötigt werden und lädt diese on-demand herunter.

MikTeX kann unter \url{https://miktex.org/download} heruntergeladen werden.
Nach der Installation sollte MikTeX über den System Try gestartet werden, um dann alle Packages zu updaten.
Je nach Internetverbindung und PC-Leistung kann dies eine Weile dauern.
Das Updaten der Packages sollte regelmäßig (in etwa monatlich) oder falls Kompilierungsprobleme auftreten wiederholt werden.

\subsection{TeXLive}
Unter Linux kann TeXLive über den hoffentlich in der Linux-Distribution enthaltenen Package-Manager installiert werden und integriert dann ähnlich wie MikTeX nahtlos in TeXStudio (oder einen anderen \LaTeX -Editor).
TeXLive sollte dann vom Package Manager up-to-date gehalten werden und es ist keine weitere Interaktion notwendig.

Unter MacOS sollte die Mac-spezifische Variante MacTeX verwendet werden, die unter \url{https://www.tug.org/mactex/} zur Verfügung steht.
Dem Autor sind hier nähere Details nicht bekannt, der Leser mag, falls er Mac-User ist, diese gerne beisteuern.



	\chapter{Unser erstes Dokument}
Nun, da die Installation abgeschlossen ist, können wir unser erstes \LaTeX{}-File erstellen.

Im Gegensatz zu traditionellen Textverarbeitungsprogrammen besteht ein \LaTeX{}-Projekt meist nicht aus einem einzigen File, sondern wird jeweils in einem eigenen Ordner verwaltet.
Wir erstellen daher an einem angemessenen Ort, z.B. \texttt{Dokumente/LaTeX-Seminar/}, einen Ordner für unser erstes Dokument und in diesem Order legen wir ein neues File an.
Bei angemessener Benennung sollte der Pfad diese Files dann \texttt{Dokumente/LaTeX-Seminar/my-first-document/my-first-document.tex} sein.
Dieses File können wir nun mit dem Editor unserer Wahl, z.B.~TeXStudio, öffnen und mit dem folgenden Text füllen:
\begin{latexlisting}
	\documentclass{article}

	\begin{document}
		Hallo, LaTeX!
	\end{document}
\end{latexlisting}
Drücken wir nun in TeXStudio auf den grünen \enquote{Kompilieren}-Pfeil, so wird unser erstes LaTeX-Dokument kompiliert und wir sollten ein PDF mit dem einsamen Satz \enquote{Hallo, LaTeX!} vor uns sehen.
Wenn wir jetzt in den vorhin erstellten Ordner sehen, so ist unser \filetype{tex}-File nicht mehr alleine.
\LaTeX{} hat nicht nur ein \filetype{pdf}-File erstellt und im Ordner abgelegt, sondern auch eine Vielzahl andere Hilfs- und Cache-Files, die zukünftige Kompilierungen beschleunigen und für verschiedenste Features von TeXStudio benötigt werden.
Insbesondere kann \LaTeX{} bei der Kompilierung nicht \enquote{in die Zukunft schauen}, d.h. wenn das Inhaltsverzeichnis kompiliert wird, weiß \LaTeX{} noch gar nicht, welche Kapitel etc. überhaupt im Dokument vorkommen.
Daher werden Kapitel in das \filetype{toc}-File geschrieben, damit zukünftige Kompilierung ein sinnvolles Inhaltsverzeichnis erzeugen können.
Die meisten \LaTeX{}-Dokumente müssen deswegen mehrmals kompiliert werden, glücklicherweise übernimmt diesen Schritt TeXStudio für uns.
Diese Nebenprodukte jedes Kompilierungsvorgangs sind einer der Gründe, warum jedes \LaTeX{}-Projekt generell einen eigenen Ordner verdient.

\section{Nebenprodukte \& Dokumentstruktur}
Wir betrachten in \autoref{tab:out-files} kurz einige der wichtigen Filetypen, die beim Kompilieren entstehen und in \autoref{tab:source-files} einige Filetypen, die beim Kompilieren als Quelldokumente verwendet werden und deren Einbindung wir in zukünftigen Kapiteln behandeln werden.

Aufgrund dieser Vielzahl an Files ist es wie bereits erwähnt wichtig, jedes \LaTeX{}-Projekt in einem eigenen Ordner zu verwalten.
Es kann sich auch lohnen, ein Projekt in mehrere \filetype{tex}-Files aufzuteilen, wie wir in späteren Kapiteln noch lernen werden.

\begin{table}
	\begin{tabular}{l p{10cm}}
		\toprule
		\textbf{Filetyp} & \textbf{Beschreibung} \tabularnewline
		\midrule
		\filetype{aux} &
		Hilfsfile, das Informationen speichert, die zukünftigen Kompilierungen die Nutzung von Referenzen und Zitationen ermöglicht. \tabularnewline
		\filetype{out} &
		Hilfsfile, das von \latexpackage{hyperref} verwendet wird, um PDF-Lesezeichen zu erzeugen. \tabularnewline
		\filetype{log} &
		Logfile, das die Kompilierung protokolliert und auch eventuell aufgetretene Fehler oder Warnungen enthält.
		Erlaubt TeXStudio, eine Zusammenfassung dieser Fehler anzuzeigen. \tabularnewline
		\filetype{synctex} &
		File, das es TeXStudio ermöglicht, den Nutzer von Code zu PDF und umgekehrt springen zu lassen.
		\tabularnewline
		\filetype{toc} &
		File, das Informationen über Kapitel und Abschnitte enthält, damit zukünftige Kompilierungen ein Inhaltsverzeichnis erzeugen können. \tabularnewline
		\filetype{fls}, \filetype{bbl}, \filetype{bcl} &
		Files für verschiedenste Packages. \tabularnewline
		\bottomrule
	\end{tabular}
	\caption{Nebenprodukte einer Kompilierung}
	\label{tab:out-files}
\end{table}

\begin{table}
	\begin{tabular}{l p{10cm}}
		\toprule
		\textbf{Filetyp} & \textbf{Beschreibung} \tabularnewline
		\midrule
		\filetype{tex} & File, dass \LaTeX{}-Quellcode enthält. \tabularnewline
		\filetype{bib} & Bibliographie-File, in dem Quellen gelistet sind. \tabularnewline
		\filetype{ttf} & Schriftart-File \tabularnewline
		\filetype{sty} & Package-File für ein eigenes, lokale Package. \tabularnewline
		\filetype{cls} & Klassen-File für eine eigene Dokumentklasse. \tabularnewline
		\filetype{png}, \filetype{jpeg}, \filetype{jpg} & Bild-File für die Inklusion von Bildern im Dokument. \tabularnewline
		\bottomrule
	\end{tabular}
	\caption{Weitere Quell-Files}
	\label{tab:source-files}
\end{table}

\section{Dokumentklasse}

	\chapter{Struktur}
In diesem Kapitel behandeln wir \LaTeX -Commands, die es uns erlauben, etwas Struktur in unsere Dokumente zu bringen.
Erneut ist hier die Semantik wichtig:
Strukturcommands wie \latexcommand{chapter} in \LaTeX{} haben nicht nur Formatierungsauswirkungen (Schriftgröße, neue Seite, etc.) sondern teilen dem Compiler auch semantisch mit, dass hier ein neuer Abschnitt beginnt.

\section{Seitenumbrüche}
Um eine neue Seite zu beginnen, gibt es in \LaTeX{} verschiedene Commands, die wir kurz durchgehen:
\begin{itemize}
	\item \latexcommand{newpage}: Beginnt auf der Stelle eine neue Seite.
	Im zweispaltigen Modus wird nur eine neue Spalte begonnen.
	\item \latexcommand{clearpage}: Beginnt ebenfalls eine neue Seite.
	Außerdem werden sogenannte \emph{float-Objekte} wie Abbildungen oder Tabellen noch abgedruckt (man nennt dies \emph{flushing}), bevor die neue Seite beginnte.
	Generell das vorzuziehende Command.
	\item \latexcommand{cleardoublepage}: Selbe Effekte wie \latexcommand{clearpage}, aber eventuell wird eine Leerseite gelassen, so dass die neue Seite auf einer ungeraden Seite beginnt.
	Dies ist insbesondere für doppelseitige Dokumente wie Bücher nützlich.
	\item \latexcommand{pagebreak}: Dieses Command erzeugt nicht zwangsweise einen Seitenumbruch.
	Stattdessen wird es nach dem Schreiben des Textes verwendet, um \LaTeX{} Stellen mitzuteilen, an denen natürliche Seitenumbrüche (also solche, die aufgrund der Textlänge nötig sind), bevorzugt werden sollen, falls einen die automatische Wahl des Algorithmus nicht gefällt.
	Mit einem optionalen Argument im Bereich $[0,4] \cap \mathbb{N}$ kann die Stärke dieser Bevorzugung gesetzt werden.

	Wird \latexcommand{pagebreak} fälschlicherweise verwendet, um eine neue Seite nach beispielsweise einem Absatz zu beginnen, wird der Content der vorherigen Seite vertikal gleichmäßig verteilt, was nicht immer erwünscht ist.
\end{itemize}

\section{Abschnitte \& Anverwandte}

Einen neuen Abschnitt erzeugen wir mit dem Command \latexcommand{section}:
\begin{latexlisting}
	\section{Loomings}
	Call me Ishmael.
	Some years ago -- never mind how long precisely -- having little or no money in my purse, and nothing particular to interest me on shore, I thought I would sail about a little and see the watery part of the world.
\end{latexlisting}
Dies hat mehrere Effekte:
\begin{itemize}
	\item An der angebenen Stelle wird der Titel mit passender Formatierung gedruckt.
	\item Der Abschnitt bekommt eine passende Nummer zugewiesen.
	\item Die Zähler von Unterabschnitten werden zurückgesetzt.
	\item Der Inhalt von \latexcommand{thesection} wird aktualisiert.
	\item Das Kapitel wird im Inhaltsverzeichnis registriert und taucht beim nächsten Kompilieren dort auf.
	\item Übriger vertikaler Platz auf einer Seite wird bevorzugt vor dem Abschnitt verteilt.
	\item \dots
\end{itemize}
Ähnlich zu \latexcommand{section} existieren auch die Commands \latexcommand{subsection} und \latexcommand{subsubsection} (jeweils mit eigener Formatierung), die zu \latexcommand{section} untergeordnet gezählt werden.
Weiterhin existiert noch das Command \latexcommand{paragraph}, mit dem ein einzelner Absatz mit einem einleitenden, hervorgehobenen Titel versehen werden kann.
Letzteres sehen wir beispielsweise in \hyperref[sec:introduction]{der Einführung}.

Generell wird empfohlen, \latexcommand{subsubsection} eher zu vermeiden und sich auf die anderen Commands zu verlassen.
Wer das Gefühl hat, \latexcommand{subsubsubsection} zu benötigen (ein Command, das aus gutem Grund nicht existiert), sollte vermutlich eher seine Textstruktur überdenken.

Die Auswahl an verfügbaren Abschnittsebenen hängt von der gewählten Dokumentklasse ab -- oben genanntes gilt für \latexargument{article} oder \latexargument{scrartcl}.
Die Klasse \latexargument{minimal} stellt keinerlei solche Commands zu Verfügung, die Klasse \latexargument{beamer} für Präsentationsfolien definiert zwar auch \latexcommand{section} und \latexcommand{subsection}, diese haben allerdings andere Auswirkungen und erzeugen je nach Konfiguration z.B. eigene Titelfolien für Abschnitte.
Die Klassen \latexargument{book} und \latexargument{scrbook} bietet zusätzlich zu oben genannten noch die Commands \latexcommand{chapter} und \latexcommand{part}, die den Abschnitten übergeordnet sind.
Diese haben zusätzlich noch die Auswirkungen, dass ein Kapitel (chapter) stets auf einer neuen, ungeraden Seite beginnt, während bei einem Teil (part) der Titel sogar alleine auf einer sonst leeren, ungeraden Seite steht.
\begin{latexlisting}
	\chapter{The Carpet-Bag}
	I stuffed a shirt or two into my old carpet-bag, tucked it under my arm, and started for Cape Horn and the Pacific.
\end{latexlisting}
All diese Commands besitzen außerdem sogenannte \emph{gesternte} Versionen.
Gesternte Commands in \LaTeX{} werden für leichte Abwandlungen anderer Commands verwendet:
\begin{latexlisting}
	\subsection*{The Spouter-Inn}
	Entering that gable-ended Spouter-Inn, you found yourself in a wide, low, straggling entry with old-fashioned wainscots, reminding one of the bulwarks of some condemned old craft.
\end{latexlisting}
Bei den Commands für die Abschnittsebenen hat dies den Effekt eines unnummerierten Abschnitts (Kapitels, Unterabschnitts, etc.), der auch nicht im Inhaltsverzeichnis geführt wird.

\section{Inhaltsverzeichnis}
Wenn man im Textkörper brav die obigen Commands verwendet, dann ist das Erstellen eines Inhaltsverzeichnises sehr einfach:
\begin{latexlisting}
	\tableofcontents
\end{latexlisting}
Das Inhaltsverzeichnis erhält, je nach Dokumentklasse, ein unnumeriertes Chapter oder Section mit dem Titel \enquote{Content}.
Wird die Package \latexpackage{babel} verwendet, so wird der Titel des Inhaltsverzeichnises automatisch an die gewählte Sprache angepasst:
\begin{latexlisting}
	% Präambel
	\usepackage[ngerman]{babel}
\end{latexlisting}
Dieses Verhalten wird sich bei vielen \LaTeX -Elementen wiederholen -- es ist daher immer gut, die \latexpackage{babel}-Package stets einzubinden.
Wird nur eine Sprache verwendet, so genügt es, diese im optionalen Argument von \latexcommand{documentclass} zu nennen und \latexpackage{babel} ohne optionales Argument zu inkludieren; wird im Dokument zwischen mehreren Sprachen gewechselt, so sollten diese alle im optionalen Argument genannt werden.
Die erste Sprache ist hierbei die primäre:
\begin{latexlisting}
	% Präambel
	\documentclass{scrbook}
	\usepackage[ngerman, english]{babel}
	% ODER
	\documentclass[ngerman]{scrbook}
	\usepackage{babel}
\end{latexlisting}

Soll nach dem Inhaltsverzeichnis eine neue Seite beginnen, so kann dies manuell mit \latexcommand{clearpage} getan werden.
Vor Einfügen des Verzeichnisses kann mittels des Commands
\begin{latexlisting}
	\setcounter{tocdepth}{3}
\end{latexlisting}
festgelegt werden, bis zu welcher Ebene Abschnitte im Inhaltsverzeichnis inkludiert werden sollen.
Die Nummerierung verhält sich wie in Tabelle \autoref{tab:tocdepth} aufgelistet.

\begin{table}
	\begin{tabular}{l p{5cm} p{5cm}}
		\toprule
		\textbf{Nummer} & \textbf{Command} & \textbf{Ebene} \tabularnewline
		\midrule
		$-1$ &
		\latexcommand{part} &
		 Teil
		\tabularnewline
		$0$ &
		\latexcommand{chapter} &
		Kapitel
		\tabularnewline
		$1$ &
		\latexcommand{section} &
		Abschnitt
		\tabularnewline
		$2$ &
		\latexcommand{subsection} &
		Unterabschnitt
		\tabularnewline
		$3$ &
		\latexcommand{subsubsection} &
		Unterunterabschnitt
		\tabularnewline
		\bottomrule
	\end{tabular}
	\caption{Abschnittsebenen}
	\label{tab:tocdepth}
\end{table}

Zum Testen des Inhaltsverzeichnises (und anderer Dinge in der Zukunft) wird auch die Package \latexpackage{blindtext} sehr empfohlen.
Diese stellt die Commands \latexcommand{blindtext}, das einen große Menge sprachensentiven Blindtext einfügt, und \latexcommand{blinddocument}, das ein langes Testdokument voll mit Abschnitten und Kapitel einfügt, bereit.

\section{Anpassung}
Häufig möchte man das Aussehen der Überschriften noch ein wenig anpassen.
Die KOMA-Klassen stellen dazu einige relativ simple Commands zur Verfügung, die man in der Präambel verwenden kann, um verschiedenste Aspekte der Abschnittscommands zu verändern.
Beim Verwenden der Standardklassen muss auf die -- nach der Meinung des Autors deutlich unintuitivere -- Package \latexpackage{titlese} zurückgegriffen werden, die hier nicht näher thematisiert wird.

Im Folgenden werden wir das Aussehen von \latexcommand{section} modifizieren, aber die selben Commands funktionieren -- mit den offentsichtlichen Abwandlungen -- auch für \latexcommand{part}, \latexcommand{chapter} und (mit kleinen Ausnahmen) \latexcommand{subsection} sowie \latexcommand{subsubsection}.

\subsection{Schrift}
Die einfachste mögliche Abwandlung ist die der Schrift des Abschnittstitels.
In KOMA geschieht dies mittels des Commands \latexcommand{addtokomafont}:
\begin{latexlisting}
	\addtokomafont{section}{\normalsize\rmfamily\mdseries\itshape}
\end{latexlisting}
Hier werden einige Formatierungsoptionen über sogenannte \emph{font commands} gesetzt.
Diese unterscheiden sich leicht von den bisher bekannten Commands wie etwa \latexcommand{textit}: Letzteres erhält ein Argument, das kursiv gesetzt wird, während \latexcommand{itshape} ähnlich wie \latexcommand{selectfont} einen Effekt bis zum Ende des aktuellen Blocks aktiviert.
Neben diesen in \autoref{tab:basic-font-commands} aufgeführten Font-Commands sind auch die in \autoref{tab:font-sizes} aufgeführten Commands zur Schriftgröße sowie das manuelle Setzen der Schriftgröße hier möglich und Verändern das Aussehen von Abschnittstiteln entsprechend.
Es ist dabei guter Stil, jeweils genau eine Size, Family, Series und Shape anzugeben, also beispielsweise \latexcommand{mdseries} explizit auszuschreiben, obwohl es sich um den Default handelt und weggelassen werden könnte.
Natürlich ist in der Theorie jeder mögliche \LaTeX -Code zugelassen, aber wer sich auf diese Commands beschränkt, vermeidet unangenehme Überraschungen und unerwartetes Verhalten.
Wer kompliziertere Änderungen an seinen Abschnittstiteln vornehmen will, kann stattdessen die in den nächsten Unterabschnitten vorgestellen Commands verwenden.

\begin{table}
	\begin{tabular}{l p{10cm}}
		\toprule
		\textbf{Command} & \textbf{Wirkung} \tabularnewline
		\midrule
		\latexcommand{rmfamily} &
		Gewöhnliche Schriftart (mit Serifen)
		\tabularnewline
		\latexcommand{sffamily} &
		Serifenlose Schrift
		\tabularnewline
		\latexcommand{ttfamily} &
		Schreibmaschinen-Schrift (Monospace)
		\tabularnewline
		\latexcommand{mdseries} &
		Normalfette Schrift
		\tabularnewline
		\latexcommand{bfseries} &
		Fette Schrift
		\tabularnewline
		\latexcommand{upshape} &
		Normalausgerichtete Schrift
		\tabularnewline
		\latexcommand{itshape} &
		Kursive Schrift
		\tabularnewline
		\latexcommand{slshape} &
		Geschrägte Schrift
		\tabularnewline
		\latexcommand{scshape} &
		Kapitälchen
		\tabularnewline
		\latexcommand{centering} &
		Zentriert
		\tabularnewline
		\latexcommand{raggedleft} &
		Rechtsbündig
		\tabularnewline
		\latexcommand{raggedright} &
		Linksbündig
		\tabularnewline
		\bottomrule
	\end{tabular}
	\caption{Grundlegende Font-Commands}
	\label{tab:basic-font-commands}
\end{table}

\subsection{Abstände}
Weiterhin lassen sich die Abstände vor und nach der Abschnittsüberschrift dynamisch anpassen.
Hierzu wird das Command \latexcommand{RedeclareSectionCommand} verwendet.
\begin{latexlisting}
	\RedeclareSectionCommand[
		beforeskip=1\baselineskip,
		afterskip=5cm,
		afterindent=false,
	]{section}
\end{latexlisting}
Man beachte hier, dass beim Anpassen von z.B. \latexcommand{chapter} nur das hintere \latexargument{section} durch \latexargument{chapter} ersetzt wird, der Commmandname \latexcommand{RedeclareSectionCommand} aber gleichbleibt.
Die Key/Value-Paare des optionalen Arguments haben hierbei die folgenden Auswirkungen:
\begin{itemize}
	\item \latexargument{beforeskip}: Abstand vor dem Abschnittstitel.
	Wird häufig als Vielfaches von \latexcommand{baselineskip}, dem Standard-Abstand zwischen zwei Zeilen, angegeben.
	Alternativ sind Einheiten wie cm, pt, em, ex, \dots erlaubt.
	\item \latexargument{afterskip}: Wie \latexargument{beforeskip}.
	\item \latexargument{afterindent}: Kann nur auf \latexargument{true} oder \latexargument{false} gesetzt werden.
	 Verhindert, dass der Absatz nach dem Abschnittstitel horizontal eingerückt wird.
	Je nachdem, welche KOMA-Version verwendet wird, kann es sein, dass zusätzlich oder stattdessen der \latexargument{beforeskip} negativ gegeben werden muss, damit dies funktioniert (der vorherige Abstand ist dann trotzdem positiv und existent).
\end{itemize}

\paragraph{Exkurs: Skips}
Beim Design von (Abschnitts-)titels ist es oft wünschenswert, an gewissen Stellen etwas leeren horizontalen oder vertikalen Platz zu lassen.
Dies geschieht in \LaTeX{} mithilfe von sogenannten \emph{Skips} (\enquote{Sprüngen}).
%TODO

\subsection{Abschnittszähler}
Zusätzlich können wir auch den Abschnittszähler, der vor dem Titel erscheint, formatieren:
\begin{latexlisting}
\renewcommand*{\sectionformat}{\Large\scshape\thesection\qquad}	
\end{latexlisting}
Hier werden wie bei der Formatierung des Titels selbst Anpassungen wie Größe oder Form vorgenommen.
Anders als dort muss die Nummerierung selbst mit dem Command \latexcommand{thesection} (respektive \latexcommand{\thechapter}, \dots) manuell eingefügt werden und wird andernfalls weggelassen.
Dies erlaubt es einem aber auch, die Nummerierung mit zusätzlichem Text zu versehen, etwa
\begin{latexlisting}
\renewcommand*{\sectionformat}{\Large\scshape\thesection . Abschnitt\qquad}	
\end{latexlisting}
Außerdem ist es in den meisten Fällen sinnvoll, nach der Nummerierung etwas Platz einzufügen, beispielsweise mit dem Command \latexcommand{quad} oder für etwas mehr Kontrolle \latexcommand{hspace\{2cm\}} oder ähnlich.
Andernfalls wird nämlich der Titel selbst ohne Abstand direkt hinter die Nummerierung gesetzt.

\subsection{Genauere Anpassungen}
Für genauere Anpassungen kann das Command \latexcommand{sectionlinesformat} (sowie seine Geschwister) redefiniert werden:
\begin{latexlisting}
	\renewcommand{\sectionlinesformat}[4]{%
	  \rule[-.15\baselineskip]{\linewidth}{.5pt}\par\nobreak%
	  #3%
	  #4% 
	}
\end{latexlisting}
Der innere Teil dieser Redefinition wird dann für jeden Section-Titel ausgeführt, wobei \latexargument{\#3} durch die (bereits formatierte) Nummerierung und \latexargument{\#4} durch den bereits formatierten Abschnittstitel ersetzt werden.
Hier instruieren wir \LaTeX{} also, eine horizontale Linie (\latexcommand{rule}) quer über die Seite zu zeichnen und dann Nummer und Titel zu drucken, aber es sind natürlich deutlich kompliziertere Konstruktionen möglich.
Es ist zu beachten, dass, anders als bei den vorherigen Anpassungen, dieses Command stets für alle drei Abschnittsebenen \latexcommand{section}, \latexcommand{subsection} und \latexcommand{subsubsection} verwendet wird und daher Änderungen all diese drei Ebenen betreffen.

\section{Matters}
Die \latexargument{scrbook}- und \latexargument{book}-Klassen verfügen noch über drei weitere Strukturcommands: \latexcommand{frontmatter}, \latexcommand{mainmatter} und \latexcommand{backmatter}.
Diese teilen ein Buch in drei inhaltliche Teile:
Die \emph{Front Matter} umfasst alles zwischen Titel und dem ersten Inhaltskapitel, also Vorwort, Einführung, Inhaltsverzeichnis, etc.
Kapitel in der Front Matter werden nicht nummeriert, und die Seitenzahlen sind römisch.
Die \emph{Main Matter} umfasst den Haupttext.
Kapitel hier sind nummeriert, die Seitenzahlen sind arabisch und werden mit Beginn der Main Matter zurückgesetzt.
Die \emph{Back Matter} enthält den Anhang, Index, Bibliographie, Abbildungsverzeichnisse, Glossare und weitere Zusatzinformationen, die nicht zum Haupttext gehören.
Kapitel hier sind nicht mehr nummeriert, aber die Seitenzahlen werden nicht zurückgesetzt.
\begin{latexlisting}
	\maketitle
	\frontmatter
	\chapter{Preface}
	\blindtext[4]

	\clearpage
	\tableofcontents
	\clearpage

	\chapter{Introduction}
	\blindtext[2]

	\mainmatter

	\chapter{Installation}
	\blindtext[10]

	\chapter{First Document}
	\blindtext[10]

	\backmatter
	\chapter{Appendix A}
	\blindtext[5]
\end{latexlisting}
Man beachte, dass für das obige Beispiel die Metadaten \latexcommand{title} und \latexcommand{author} gesetzt sein sollten.

\subsection{Seitenzahlen \& Counter}\label{sec:pages-and-counters}
In einigen Fällen möchte man Seitenzahlen auch manuell anpassen, um z.B. künstlichen Forderungen bei abzugebenden Arbeiten gerecht zu werden.
Seitenzahlen in \LaTeX{} werden in einem \emph{Counter} (deutsch: \emph{Zähler}) gespeichert, wie auch einige andere relevante Daten, die wir später noch kennenlernen.
Counter können durch zwei wichtige Commands modifiziert werden:
\begin{latexlisting}
	\setcounter{page}{3}
\end{latexlisting}
Dieser Befehl setzt einen Counter, hier also die aktuelle Seitenzahl.
Danach wird normal weitergezählt.
Soll stattdessen beispielsweise eine Seite übersprungen werden, können wir mit dem Command
\begin{latexlisting}
	\addtocounter{page}{2}
\end{latexlisting}
den Counter einmalig um $2$ erhöhen und dann normal weiterzählen.
Für den Sonderfall des Erhöhens um $1$ existiert auch der Befehl
\begin{latexlisting}
	\stepcounter{page}
\end{latexlisting}
Dies funktioniert auch mit Sections und Chaptern:
\begin{latexlisting}
	\section{Ein Abschnitt}
	\addtocounter{section}{1}
	\section{Oha, da fehlt was!}
\end{latexlisting}
Einen Counter auslesen können wir mit dem Command \latexcommand{the<countername>}, beispielweise \latexcommand{thepage} oder \latexcommand{thesection}, die wir schon kennengelernt haben.
Dies ist allerdings etwas veralteter \TeX -Syntax.
In \LaTeX{} sollte man stattdessen den Befehl \latexcommand{value} verwenden, wenn man den Wert des Counters für arithmetische Zwecke braucht, beispielsweise
\begin{latexlisting}
	\addtocounter{section}{\value{page}}
\end{latexlisting}
falls man, aus welchen Gründen auch immer, so viele Kapitel überspringen will wie die aktuelle Seitenzahl.
Möchte man den Wert des Counters stattdessen anzeigen, so ist eine der in \ref{tab:counter-representations} aufgeführten Optionen empfohlen, die direkt die gewünschte Repräsentation angibt.

\begin{table}
	\begin{tabular}{l p{5cm} p{5cm}}
		\toprule
		\textbf{Command} & \textbf{Ziel} & \textbf{\latexcommand{<command>\{page\}}} \tabularnewline
		\midrule
		\latexcommand{arabic} &
		Arabische Zahlen &
		\arabic{page}
		\tabularnewline
		\latexcommand{roman} &
		Römische Zahlen, klein &
		\roman{page}
		\tabularnewline
		\latexcommand{Roman} &
		Römische Zahlen, groß &
		\Roman{page}
		\tabularnewline
		\latexcommand{alph} &
		Kleinbuchstaben &
		\alph{page}
		\tabularnewline
		\latexcommand{Alph} &
		Großbuchstaben &
		\Alph{page}
		\tabularnewline
		\bottomrule
	\end{tabular}
	\caption{Darstellung von Zählern}
	\label{tab:counter-representations}
\end{table}

Schließlich lassen sich auch eigene Counter erzeugen:
\begin{latexlisting}
	\newcounter{mycounter}
\end{latexlisting}
Dies ist allerdings eine fortgeschrittene Anwendung von Countern, die im täglichen Gebrauch eher selten zum Einsatz kommt.


\section{Kopf- und Fußzeilen}
Eine weitere oft genutzte Anpassung sind die Kopf- und Fußzeilen des Dokuments.
Hierfür kann -- in KOMA-Klassen -- die Package \latexpackage{scrlayer-scrpage} verwendet werden.
In den Standardklassen bietet sich stattdessen die Package \latexpackage{fancyhdr} an, die wir an dieser Stelle nicht weiter thematisieren.
\begin{latexlisting}
	% Präambel
	\usepackage{scrlayer-scrpage}
\end{latexlisting}
Die Kopf- und Fußzeile besteht aus drei Regionen: Links, Zentrum und Rechts, die jeweils einzeln gesetzt werden können.
In zweiseitigen Dokumenten (\latexargument{scrbook}) können wir diese auch abwechselnd definieren, also Außen und Innen statt Links und Rechts.
Ein solches Command wäre
\begin{latexlisting}
	% Präambel
	\ofoot{Seite \arabic{page}}
\end{latexlisting}
welches unten außen (also links auf linken Seiten und rechts auf rechten Seiten) die Seitenzahl einfügt.
\latexcommand{ofoot} und seine Geschwister sind in Tabelle \autoref{tab:footer-commands} aufgeführt.

\begin{table}
	\begin{tabular}{l p{3cm} p{7cm}}
		\toprule
		\textbf{Command} & \textbf{Bedeutung} & \textbf{Effekt} \tabularnewline
		\midrule
		\latexcommand{lfoot}, \latexcommand{ihead} &
		Left &
		Links bei einseitigen Dokumenten
		\tabularnewline
		\latexcommand{cfoot}, \latexcommand{chead} &
		Center &
		Zentriert in der Mitte der Kopf- oder Fußzeile
		\tabularnewline
		\latexcommand{rfoot}, \latexcommand{rhead} &
		Rechts &
		Rechts bei doppelseitigen Dokumenten
		\tabularnewline
		\latexcommand{ifoot}, \latexcommand{ihead} &
		Inner &
		Innen bei doppelseitigen Dokumenten
		\tabularnewline
		\latexcommand{ofoot}, \latexcommand{ohead} &
		Outer &
		Außen bei doppelseitigen Dokumenten
		\tabularnewline
		\latexcommand{lefoot}, \latexcommand{lehead} &
		Left Even &
		Links auf den geraden Seiten von doppelseitigen Dokumenten
		\tabularnewline
		\latexcommand{refoot}, \latexcommand{rehead} &
		Right Even &
		Rechts auf den geraden Seiten von doppelseitigen Dokumenten
		\tabularnewline
		\latexcommand{lofoot}, \latexcommand{lohead} &
		Left Odd &
		Links auf den ungeraden Seiten von doppelseitigen Dokumenten
		\tabularnewline
		\latexcommand{rofoot}, \latexcommand{rohead} &
		Right Odd &
		Rechts auf den ungeraden Seiten von doppelseitigen Dokumenten
		\tabularnewline
		\bottomrule
	\end{tabular}
	\caption{Commands für Kopf- und Fußzeilen}
	\label{tab:footer-commands}
\end{table}

Ein solches Command kann natürlich jeden möglichen \LaTeX{}-Code enthalten und auch farbig oder sonstig hervorgehoben werden.
Inhaltlich nützlich sind oft die Commands \latexcommand{leftmark} und \latexcommand{rightmark}, die Informationen über den Titel des Dokuments/Kapitels/Abschnitts enthalten.

Häufig wird die Kopf- oder Fußzeile, wenn vorhanden, durch eine horizontale Linie vom Hauptblock des Textes abgetrennt.
Die \latexpackage{scrlayer-scrpage}-Package macht dies durch eine einfach Option möglich:
\begin{latexlisting}
\usepackage[footsepline, headsepline]{scrlayer-scrpage}	
\end{latexlisting}
Soll die Kopf- und Fußzeile auf einer Seite (z.B. der Titelseite, oder Seiten mit Kapitelüberschriften) ausgelassen werden, so kann dies mit dem Command
\begin{latexlisting}
	\thispagestyle{empty}
\end{latexlisting}
erreicht werden, dass den \emph{Page Style} setzt, der Dinge wie eben Kopf- und Fußzeilen aber auch die Größe des Textbereiches und der Seite anpasst. Mit
\begin{latexlisting}
	\pagestyle{empty}
\end{latexlisting}
wird der Page Style bis zur nächsten Änderung auf \latexargument{empty} gesetzt, dies kann durch
\begin{latexlisting}
	\pagestyle{scrheadings}
\end{latexlisting}
wieder aufgehoben werden.
Ebenfalls manchmal nützlich ist der Page Style \latexargument{plain}, der nur eine einfache Seitenzahl enthält.
Es ist auch möglich, eigene Page Styles zu definieren, dies werden wir an dieser Stelle aber nicht im Detail besprechen.

\section{Titelseite}
Wer seine Titelseite nicht automatisch von \latexcommand{maketitle} generieren lassen möchte, kann diese auch selbst erstellen.
Hierfür sind vorerst keine Packages nötig, da die KOMA-Klassen sowie die Standardklassen die Umgebung \latexenvironment{titlepage} für diesen Zweck bereitstellen:
\begin{latexlisting}
	\begin{titlepage}
		\centering
		~\vfill
		{
			\Huge
			\scshape
			MOBY DICK
		}\\
		\bigskip
		{
			\Large
			Or: The Whale
		}\\
		\vspace{2cm}
		{
			\itshape
			Herman Melville
		}
		\vfill
		1851
	\end{titlepage}
\end{latexlisting}
Diese Umgebung hat einige Effekte wie beispielsweise das Zurücksetzen der Seitenzahl und sollte niemals mit \latexcommand{titlepage} kombiniert werden.
Die Verwendung der Package \latexpackage{titling} erlaubt außerdem die Verwendugn der Command \latexcommand{thetitle} und \latexcommand{theauthor}, damit die für \latexcommand{maketitle} festgelegten Metadaten hier  und an anderen Stellen einheitlich verwendet werden können.
\begin{latexlisting}
	\begin{titlepage}
		\centering
		~\vfill
		{
			\Huge
			\thetitle
		}\\
		\bigskip
		{
			\Large
			Or: The Whale
		}\\
		\vspace{2cm}
		{
			\itshape
			\theautor
		}
		\vfill
		\thedate
	\end{titlepage}
\end{latexlisting}

\subsection{Abstract}
In wissenschaftlichen Texten stellt man dem Text gerne ein \emph{Abstract}, eine kurze inhaltliche Zusammenfassung, voran.
In den KOMA-Klassen geschieht dies mittels des environments \latexenvironment{abstract}.
Dieses kann vor oder nach dem Inhaltsverzeichnis, aber auch auf der Titelseite verwendet werden und markiert den Inhalt damit semantisch als das Abstract und nimmt einige kleine Formatierungsanpassungen vor.
\begin{latexlisting}
	\begin{titlepage}
		\centering
		~\vfill
		{
			\Huge
			\scshape
			Eine Einführung in \LaTeX{}
		}
		\vfill
		\begin{abstract}
			In diesem Dokument diskutieren wir Grundlagen der Textsatzprogrammsammlung \LaTeX{}.
		\end{abstract}
		\vfill
	\end{titlepage}
\end{latexlisting}

\section{Aufzählungen}

Als letzte Form der kleinteiligen Strukturierung existieren in \LaTeX{} noch \emph{Aufzählungen}.
Eine Aufzählung ist eine Umgebung (entweder \latexenvironment{itemize} oder \latexenvironment{enumerate}), in denen der Befehl \latexcommand{item} verwendet werden kann.
\begin{latexlisting}
	\begin{enumerate}
		\item Schokolade
		\item Honig
		\item Nüsse
		\item Succade
	\end{enumerate}
\end{latexlisting}
Dabei erzeugt \latexenvironment{itemize} unnummerierte Aufzählungen und \latexenvironment{enumerate} nummerierte.
Aufzählungen können beliebig verschachtelt werden, dies ändert dann auch das verwendete Aufzählungszeichen.
\begin{latexlisting}
	\begin{enumerate}
		\item Erster Weihnachtstag
		\begin{enumerate}
			\item Ein Rebhuhn im Birnenbaum
		\end{enumerate}
		\item Zweiter Weihnachtstag
		\begin{enumerate}
			\item Ein Rebhuhn im Birnenbaum
			\item Zwei Tauben
		\end{enumerate}
		\item \dots
	\end{enumerate}
\end{latexlisting}
Wir können das Aufzählungszeichen einzelner Punkte manuell ändern:
\begin{latexlisting}
	Weihnachtsliederranking:
	\begin{enumerate}
		\item Hört der Engel helle Lieder
		\item Deck the Halls
		\item[2.] Fröhliche Weihnacht überall
		\item Leise rieselt der Schnee
	\end{enumerate}
\end{latexlisting}
Man beachte, dass dabei der ersetzte Eintrag übersprungen wird und es also in Beispiel bei $4$ weitergeht.
Möchte man tatsächlich eine Zahl überspringen, kann man ähnlich wie in \autoref{sec:pages-and-counters} den Zählcounter der Aufzählung manipulieren.
Dieser Counter hat, je nach Ebene, den Namen \latexargument{enumi}, \latexargument{enumii}, \latexargument{enumiii} oder \latexargument{enumiv}.
\begin{latexlisting}
	\begin{enumerate}
		\item Erster Weihnachtstag
		\begin{enumerate}
			\item Ein Rebhuhn im Birnenbaum
		\end{enumerate}
		\item \dots
		\addtocounter{enumi}{9}
		\item Zwölfter Weihnachtstag
		\begin{enumerate}
			\item Ein Rebhuhn im Birnenbaum
			\item Zwei Tauben
			\item \dots
			\addtocounter{enumii}{8}
			\item Zwölf trommelnde Trommler
		\end{enumerate}
	\end{enumerate}
\end{latexlisting}

\subsection{Anpassung}
Standardmäßig zählt \LaTeX{} in der ersten Ebene mit arabischen Zahlen und dann mit Kleinbuchstaben.
Möchte man dies für eine einzelne Aufzählung oder das ganze Dokument ändern
Die Package \latexpackage{enumitem} erlaubt eine kleinteilige Anpassung der Formatierung von Aufzählungen.
\begin{latexlisting}
	% Präambel
	\usepackage{enumitem}
	\setlist[enumerate,1]{label=(\roman*)}
	\setlist[enumerate,2]{label=(\alph*)}
\end{latexlisting}
Das Command \latexcommand{setlist} wird hier genutzt, um global das Format von Aufzählungen zu ändern.
Im optionalen Argument wird dabei die Zielaufzählung (\latexargument{enumerate} oder \latexargument{itemize}) angegeben sowie die gewünschte Ebene.
Im Argument wird dann das gewünschte Format angegeben.
Für uns wichtig ist hier hauptsächlich der Key \latexargument{label}, der den Zähler verändert.
Hier können Klammern, Punkte, etc. frei verwendet werden, durch die in \autoref{tab:counter-representations} aufgeführten Commands mit einen schließenden \latexargument{*} werden dann die entsprechend formatierten Zähler eingefügt.
Im Beispiel wird also die erste Ebene durch eingeklammerte, kleine römische Zahlen gezählt und die zweite durch eingeklammerte Kleinbuchstaben.
\latexpackage{enumitem} verfügt neben \latexargument{label} noch über viele weitere Anpassungsoptionen, die sich in der Dokumentation (\url{https://ctan.net/macros/latex/contrib/enumitem/enumitem.pdf}) nachlesen lassen.

Soll die Formatierung nur für eine einzelne Aufzählung geändert werden, kann dies mittels einen optionalen Arguments der Umgebung erreicht werden:
\begin{latexlisting}
	Worauf das Kindlein liegt:
	\begin{enumerate}[label=|\Alph*|]
		\item Heu
		\item Stroh
	\end{enumerate}
	Wer es froh betrachtet:
	\begin{enumerate}[label=\textbf{\Roman*}]
		\item Maria
		\item Josef
	\end{enumerate}
\end{latexlisting}

	\chapter{Zusätzliche Inhalte}
In diesem Kapitel beschäftigen wir uns damit, wie wir zusätzliche Inhalte (keinen Fließtext) in unsere Dokumente einbauen können.
Insbesondere lernen wir dabei sogenannte \emph{floats} kennen, Objekte, deren Position nicht unbedingt mit ihrer Code-Position übereinstimmen und die frei über die Seite \enquote{fließen} können.

\section{Bilder}
\LaTeX -Dokumente können Bilder in einer Vielzahl an Formaten einbinden, insbesondere \filetype{png}, \filetype{jpg} und \filetype{jpeg}.
Hierzu verwenden wir das beliebte Package \latexpackage{graphicx}.
\begin{latexlisting}
	% Präambel
	\usepackage{graphicx}
\end{latexlisting}
Dann kann im Text der Befehl \latexcommand{includegraphics} benutzt werden, der an der entsprechenden Stelle ein Bild \enquote{inline}, d.h. wie ein (besonders großes) Wort einfügt:
\begin{latexlisting}
	Hier ist ein Bild:
	\includegraphics{my-image.png}
\end{latexlisting}
Dies funktioniert natürlich nur, wenn das Bild \filepath{my-image.png} auch existiert und im selben Order liegt wie das zentrale File \filepath{main.tex}.
Die Dateiendung \filetype{png} kann auch weggelassen werden:
\begin{latexlisting}
	Hier ist ein Bild:
	\includegraphics{my-image}
\end{latexlisting}
Dies hat Vor- und Nachteile:
\begin{itemize}
	\item \textbf{Vorteil}: Soll das Bild ausgetauscht werden, und ändert sich dabei der Filetype, so muss der Code nicht verändert werden.
	Außerdem muss man beim Schreiben des Codes nur den Filenamen kennen, nicht die genaue Erweiterung.
	\item \textbf{Nachteil}: Liegen im Ordner mehrere Dateien mit gleichem Namen, aber unterschiedlicher Erweiterung, so ist nicht immer ganz klar, welche davon von \LaTeX{} bevorzugt wird.
	Außerdem benötigt das Einfügen des Bildes so etwas mehr Rechenzeit, auch wenn dies in modernen Zeiten kaum noch relevant ist.
\end{itemize}
Weiterhin wird aufgefallen sein, dass das Bild immer in Originalgröße eingefügt wird.
Um das zu beheben, können wir die optionalen Argument von \latexcommand{includegraphics} verwenden:
\begin{latexlisting}
	Bilder:

	\includegraphics[scale=0.1]{my-image}

	\includegraphics[height=2cm]{my-image}

	\includegraphics[width=0.8\textwidth]{my-image}
\end{latexlisting}
Das Argument \latexargument{scale} skaliert das Bild um einen festen Faktor und ist eher selten nützlich.
Mit \latexargument{height} und \latexargument{width} kann dagegen die Höhe oder Breite des Bildes auf einen festen gewünschten Wert gesetzt werden.
Hierzu können auch vordefinierte kontextuelle Breiten wie \latexcommand{textwidth} oder Vielfache derer verwendet werden.
Wird nur eines dieser Argumente verwendet, wird die andere Dimension automatisch skaliert, um das Format des Bildes nicht zu verändern.
Werden beide verwendet, wird das Bild eventuell verzerrt.

Es ist in \LaTeX{} üblich, verwendete Bilder nicht direkt mit dem \filepath{main.tex}-File in einem Ordner zu speichern, sondern in einem getrennten Unterordner, der häufig \filepath{graphics} heißt.
Werden besonders viele Bilder verwendet, kann es sich auch lohnen, diesen Ordner weiter zu untergliedern.
Auf das Bild kann dann durch Angabe des vollständigen, relativen Pfades zugegriffen werden.
\begin{latexlisting}
	\includegraphics{graphics/my-image.png}
	\includegraphics{graphics/chapter1/my-image.png}
\end{latexlisting}
Das \latexpackage{graphicx}-Package erlaubt die Angabe eines Grafikpfades (also z.B. der Ordner \filepath{graphics}).
Wird dieser spezifiziert, so werden Pfade als relativ zum Grafikpfad gelesen statt relativ zum Hauptordner.
Das obige Beispiel sähe dann aus wie folgt:
\begin{latexlisting}
	% Präambel
	\graphicspath{{graphics}}
	% Text
	\includegraphics{my-image.png}
	\includegraphics{chapter1/my-image.png}
\end{latexlisting}
Man beachte, dass der Pfad in doppelte Klammern gesetzt werden muss.

\section{Positionierung von Bildern: Floats, Labels, Refs}
Wie bereits erwähnt wird das gewünschte Bild \enquote{inline} eingefügt, also ohne Zeilenumbrüche oder Rand.
Dies ist gewollt, denn es erlaubt uns Bilder zur Textdekoration zu verwenden, etwa wie folgt:
\begin{latexlisting}
	In der \emph{Sedanschlacht} am 01.09.1870 besiegten Truppen des Norddeutschen Bundes unter \includegraphics{prussian-flag} Helmut Moltke die zahlenmäßig unterlegenen französischen Streitkräfte der Generäle \includegraphics{french-flag} Patrice de Mac-Mahon und \includegraphics{french-flag} Auguste-Alexandre Ducrot.
\end{latexlisting}
Häufig ist das aber nicht die gewünschte Positionierung von Bildern.
Ein erste Verbesserung ist es, den \latexcommand{includegraphics}-Befehl mit Leerzeilen abzutrennen,
\begin{latexlisting}
	Nach dem Aufmarsch am 31. August standen sich die beiden Armeen wie folgt gegenüber:

	\includegraphics{aufmarschskizze}

	Aus dieser Position heraus begann am 01.09. um 4:00 Uhr morgens die Schlacht.
\end{latexlisting}
Hier ist das Bild allerdings noch immer linksbündig.
Um es zu zentrieren, bietet sich naiv die \latexenvironment{center}-Umgebung an:
\begin{latexlisting}
	Nach dem Aufmarsch am 31. August standen sich die beiden Armeen wie folgt gegenüber:

	\begin{center}
		\includegraphics{aufmarschskizze}		
	\end{center}

	Aus dieser Position heraus begann am 01.09. um 4:00 Uhr morgens die Schlacht.
\end{latexlisting}

\subsection{Figures \& Captions}

Das obige Vorgehen ist allerdings generell nicht empfehlenswert, das \LaTeX{} ein viel stärkeres, praktischeres Tool zur Positionierung von Bildern bereitstellt: Floats.
Beim Betrachten von professionell gesetzten Büchern -- oder auch Wikipedia-Artikeln -- fällt schnell auf, dass sich Bildern selten direkt an der Stelle im Fließtext befinden, an der sie referenziert werden, sondern stattdessen an einer passenden Stelle neben (über, unter) dem Text.
Sie \enquote{fließen} also über die Seite, weswegen man solche Objekte, die nicht \emph{im}, sondern \emph{neben} dem Text erscheinen, im Englischen als \emph{floats} bezeichnet.
In \LaTeX{} können wir dies durch die \latexenvironment{figure}-Umgebung umsetzen:
\begin{latexlisting}
	Nach dem Aufmarsch am 31. August standen sich die beiden Armeen wie folgt gegenüber:

	\begin{figure}
		\includegraphics{aufmarschskizze}		
	\end{figure}

	Aus dieser Position heraus begann am 01.09. um 4:00 Uhr morgens die Schlacht.
\end{latexlisting}
Wir bemerken, dass das Bild nun eben nicht genau zwischen den beiden Absätzen erscheint, sondern oben auf der Seite.
Wir sollten also entsprechend umformulieren:
\begin{latexlisting}
	Nach dem Aufmarsch am 31. August standen sich die beiden Armeen wie im beiligenden Bild aufgeführt gegenüber und begannen aus dieser Position heraus am 01.09. um 4:00 Uhr morgens die Schlacht.

	\begin{figure}
		\includegraphics{aufmarschskizze}		
	\end{figure}
\end{latexlisting}
Außerdem ist das Bild -- erneut -- linksbündig statt zentiert.
Dieser Mangel der \latexenvironment{figure}-Umgebung lässt sich mittels der Package \latexpackage{floatrow} beheben:
\begin{latexlisting}
	% Präambel
	\usepackage{floatrow}
\end{latexlisting}
Für gewöhnlich ist es üblich, solche beiligenden Bilder (engl. \emph{figures}, deutsch \emph{Abbildung}) mit einer Bildunterschrift (engl. \emph{caption}) zu versehen.
\begin{latexlisting}
	Nach dem Aufmarsch am 31. August standen sich die beiden Armeen wie im beiligenden Bild aufgeführt gegenüber und begannen aus dieser Position heraus am 01.09. um 4:00 Uhr morgens die Schlacht.

	\begin{figure}
		\includegraphics{aufmarschskizze}		
		\caption{Aufmarschskizze vom 31.08.1870}
	\end{figure}
\end{latexlisting}
Dies führt auch dazu, dass unsere Abbildung automatisch nummeriert wird.
Soll dies in der passenden Sprache geschehen, so ist -- wie so oft -- die \latexpackage{babel}-Package unersetzlich, wobei der verwendete Begriff notfalls auch manuell gesetzt werden kann.
Weiterhin erscheint die Abbildung damit im Abbildungsverzeichnis, das wie folgt angezeigt werden kann:
\begin{latexlisting}
	\listoffigures
\end{latexlisting}
Ist die übergebene Bildunterschrift zu lang, um sinnvoll im Verzeichnis angezeigt werden zu können, kann als optionales Argument eine Kurzversion übergeben werden:
\begin{latexlisting}
	Nach dem Aufmarsch am 31. August standen sich die beiden Armeen wie im beiligenden Bild aufgeführt gegenüber und begannen aus dieser Position heraus am 01.09. um 4:00 Uhr morgens die Schlacht.

	\begin{figure}
		\includegraphics{aufmarschskizze}		
		\caption[Aufmarschskizze]{Aufmarschskizze vom 31.08.1870}
	\end{figure}
\end{latexlisting}

\subsection{Positionierung}
Die genaue Positionierung von Floats ist kompliziert -- der gesamte Algorithmus lässt sich unter \url{https://www.latex-project.org/publications/2014-FMi-TUB-tb111mitt-float-placement.pdf} nachlesen.
Mit einem optionalen Argument von \latexenvironment{figure} können wir eine Positionierungspräferenz angeben.
\begin{latexlisting}
	Nach dem Aufmarsch am 31. August standen sich die beiden Armeen wie im beiligenden Bild aufgeführt gegenüber und begannen aus dieser Position heraus am 01.09. um 4:00 Uhr morgens die Schlacht.

	\begin{figure}[b]
		\includegraphics{aufmarschskizze}		
		\caption{Aufmarschskizze vom 31.08.1870}
	\end{figure}
\end{latexlisting}
Die verschiedenen Optionen sind in \autoref{tab:float-options} aufgeführt, die mit der \latexargument{b}-Position gesetzt wurde.
\begin{table}[b]
	\begin{tabular}{l p{8cm}}
		\toprule
		\textbf{Option} & \textbf{Position}\tabularnewline
		\midrule
		\latexargument{t} &
		Oberer Seitenrand
		\tabularnewline
		\latexargument{b} &
		Unterer Seitenrand
		\tabularnewline
		\latexargument{h} &
		\emph{Here}, also genau an der Position zwischen Absätzen, an der sich der Code befindet.
		\tabularnewline
		\latexargument{p} &
		Auf einer eigenen Seite (evtl. mit anderen Floats, um die Seite zu füllen)
		\tabularnewline
		\latexargument{!} &
		Erzwingt die Positionierung.
		\tabularnewline
		\bottomrule
	\end{tabular}
	\caption{Positionierungsoptionen von Floats}
	\label{tab:float-options}
\end{table}
Insbesondere die Option \latexargument{!h} ist sehr beliebt, denn sie erreicht genau die Positionierung, die der naive Typograph sich oft intuitiv wünscht.
\begin{latexlisting}
	Nach dem Aufmarsch am 31. August standen sich die beiden Armeen wie folgt gegenüber:

	\begin{figure}[!h]
		\includegraphics{aufmarschskizze}
		\caption{Aufmarschskizze vom 31.08.1870}
	\end{figure}

	Aus dieser Position heraus begann am 01.09. um 4:00 Uhr morgens die Schlacht.
\end{latexlisting}
Es sei allerding an dieser Stelle erwähnt, dass dies seltener optimal ist, als man denkt.

\subsection{Subfloats}
Manchmal ist es wünschenswert, mehrere Bilder in einer Abbildung zusammenzufassen.
Dies kann mit der Package \latexpackage{subfig} und dem Befehl \latexenvironment{subfloat} umgesetzt werden:
\begin{latexlisting}
	\begin{figure}
		\subfloat[Helmut Moltke]{
			\includegraphics[height=5cm]{moltke}
		}
		\hfil
		\subfloat[Friedrich Willhelm]{
			\includegraphics[height=5cm]{friedrich-willhelm}
		}
		\caption{Deutsche Oberbefehlshaber}
	\end{figure}
\end{latexlisting}
Der \latexcommand{hfil}-Befehl dient hierbei einer gleichmäßigen horizontalen Verteilung der Subabbildungen.

\subsection{Labels \& Refs}
Wenn das gewünschte Bild nun aber wild über die Seite fließt, ist ein Hinweis wie \enquote{das beiligende Bild} oft nur begrenzt nützlich.
Praktischer wäre ein Beweis wie \enquote{Bild 4.2}.
Wir können diesen natürlich einfach manuell in den Text schreiben, aber was wenn wir die Reihenfolge der Bilder verändern, oder weiter vorne eine neue Abbildung einfügen?
Hierfür bietet sich die Verwendung von \emph{Labels} und \emph{Refs} (\emph{Referenzen})  an.
\begin{latexlisting}
	Nach dem Aufmarsch am 31. August standen sich die beiden Armeen wie in Abbildung \ref{fig:aufmarschskizze} aufgeführt gegenüber und begannen aus dieser Position heraus am 01.09. um 4:00 Uhr morgens die Schlacht.

	\begin{figure}
		\includegraphics{aufmarschskizze}		
		\caption{Aufmarschskizze vom 31.08.1870}
		\label{fig:aufmarschskizze}
	\end{figure}
\end{latexlisting}
Der Befehl \latexcommand{label} versieht dabei die \latexenvironment{figure}-Umgebung mit einem nutzergegebenen, eindeutigen Label.
Dieses kann -- an jeder Stelle im Dokument, auch davor -- durch Verwendung des \latexcommand{ref}-Befehls referenziert werden und \LaTeX{} fügt dann automatich die korrekte Zahl ein.
Wird die Package \latexpackage{hyperref} verwendet, wird diese Zahl sogar mit der Position des Bildes verlinkt.
Der Befehl \latexcommand{ref} hat -- solange das Package \latexpackage{hyperref} eingebunden ist -- einige Varianten, wichtig sind \latexcommand{pageref} (zeigt nicht die Nummer des Ziels an, sondern die Seite und ist nützlich für Beweise über lange Strecken hinweg) sowie \latexcommand{autoref} (zeigt zusätzlich noch den Typ des Ziels an).
\begin{latexlisting}
	Am zweiten Tag der Schlacht hatte sich die ursprüngliche Aufstellung (siehe \autoref{fig:aufmarschskizze}, Seite \pageref{fig:aufmarschskizze}) deutlich verändert.
\end{latexlisting}
Nicht alle Objekte funktionieren out-of-the-box reibungslos mit \latexcommand{autoref}, und teilweise ist hier etwas manuelle Anpassung notwendig, die über das Ziel dieses Skripts hinaus geht.

Nicht nur Abbildungen und andere Floats können gelabelt (und folglich referenziert) werden, sondern auch eine Vielzahl anderer \LaTeX{}-Objekte: Subfloat, Kapitel, (Unter-)Abschnitte, Gleichungen (siehe später), mathematische Umgebungen wie Sätze oder Definitionen (siehe später), Tabellen und viele mehr. 

Das Label kann beliebige gewählt werden, üblich und empfehlenswert ist aber die folgende Formatierung wie oben:
Das Label beginnt mit einer kurzen Beschreibung seines Typs (\latexargument{fig}, \latexargument{sec}, \latexargument{def}, \dots) gefolgt von einem Doppelpunkt und dem tatsächlichen Namen.
Dieser sollte dabei niemals inhaltslos (\latexargument{kapitel3bild}) oder gar eine Nummer sein (\latexargument{kapitel3bild5}) sein, da dies dem Zweck eines Labels und einer automatischen Nummerierung widerspräche.
Leerzeichen in Labeln sind zulässig, aber ungewöhnlich.
Stattdessen werden für gewöhnlich Striche verwendet, wie in \latexargument{fig:general-von-der-tann}.
Sollen Label \enquote{kategorisiert} werden, so werden oft Doppelpunkte oder Schrägstriche als Trennzeichen verwendet, beispelsweise \latexargument{fig:chap3:general-de-wimpffen} oder \latexargument{fig:chap3/general-de-wimpffen}.


\section{Tabellen}
Neben Bildern sind Tabellen eine weitere Klasse an Nicht-Fließtext-Objekten, die wir häufig in unsere Dokumente einbinden wollen (man beachte die zahlreichen Tabellen in diesem Dokument).
In \LaTeX{} geschieht dies mittels der Umgebung \latexenvironment{tabular}:
\begin{latexlisting}
	\begin{tabular}{r l l}
		& Deutscher Bund & Frankreich \\
		Truppen & 200.000 & 130.000 \\
		Kanonen & 774 & 564
	\end{tabular}
\end{latexlisting}
Zur Verwendung:
\begin{itemize}
	\item Als Argument wird das Spaltenlayout der Tabelle übergeben.
	Jeder Buchstabe steht hierbei für eine Spalte, die möglichen Optionen sind in \autoref{tab:column-options} aufgelistet.
	\item Eine neue Spalte wird mit \latexargument{\&} begonnen.
	\item Eine neue Zeile wird mit \latexcommand{\textbackslash} begonnen -- dies ist der einzige Fall, in dem dieses Command verwendet werden sollte.
	Ist die korrekte Anzahl an Spalten noch nicht erreicht, wird ein Fehler geworfen.
\end{itemize}

\begin{table}
	\begin{tabular}{l p{8cm}}
		\toprule
		\textbf{Spaltenoption} & \textbf{Wirkung}\tabularnewline
		\midrule
		\latexargument{l} &
		Linksbündig
		\tabularnewline
		\latexargument{r} &
		Rechtsbündig
		\tabularnewline
		\latexargument{c} &
		Zentriert
		\tabularnewline
		\latexargument{p\{<Breite>\}} &
		Absatz mit gewisser Breite, hier werden auch automatisch Zeilenumbrüche eingefügt.
		\tabularnewline
		\bottomrule
	\end{tabular}
	\caption{Layoutoptionen für Spalten}
	\label{tab:column-options}
\end{table}
Ähnlich wie Bilder ist es üblich, Tabellen in eine Float-Umgebung zu setzen -- die entsprechende Umgebung heißt hier \latexenvironment{table}:
\begin{latexlisting}
	\begin{table}
		\begin{tabular}{r l l}
			& Deutscher Bund & Frankreich \\
			Truppen & 200.000 & 130.000 \\
			Kanonen & 774 & 564
		\end{tabular}
		\caption{Truppenstärken der Sedanschlacht}
		\label{tab:truppenstaerken-sedanschlacht}
	\end{table}
\end{latexlisting}
Eine Liste aller so semantisch markierten Tabellen lässt sich mit
\begin{latexlisting}
	\listoftables
\end{latexlisting}
drucken.

\subsection{Randlinien}
Unserer erzeugten Tabellen sehen bisher noch ein wenig nackt aus.
In \LaTeX{} können sowohl senkrechte als auch waagrechte Trennlinien eingefügt werden.
Obwohl beide Linientypen in \LaTeX{} out-of-the-box möglich sind, empfiehlt sich die Verwendung der Package \latexpackage{booktabs}, die auch einige Spacingprobleme der Standardimplementierung behebt.
\begin{latexlisting}
	\begin{tabular}{r l l}
		\toprule
		& Deutscher Bund & Frankreich \\ \midrule
		Truppen & 200.000 & 130.000 \\ 
		Kanonen & 774 & 564 \\ \bottomrule
	\end{tabular}
\end{latexlisting}
Man beachte die Semantik -- \latexcommand{toprule} und \latexcommand{bottomrule} schließen die Tabelle von oben und unten ab, während \latexcommand{midrule} zur Trennung von Zeilen verwendet wird.
Weiterhin existiert noch das Command \latexcommand{cmidrule}, mit dem Teile von Trennlinien gezogen werden können, etwa
\begin{latexlisting}
	\begin{tabular}{r l l}
		\toprule
		& Deutscher Bund & Frankreich \\ \midrule
		Truppen & 200.000 & 130.000 \\ \cmidrule{2-3}
		Kanonen & 774 & 564 \\ \bottomrule
	\end{tabular}
\end{latexlisting}
Es ist in \LaTeX{} üblich, die verwendeten Linien an das Ende der jeweiligen Zeile zu setzen.

\subsection{Verbotene Linien}
Neben diesen vier Linientypen gibt es noch weitere Möglichkeiten, Trennlinien in Tabellen zu erzeugen.
Zum einen können wir durch Wiederholung der Befehle doppelte Linien nutzen:
\begin{latexlisting}
	\begin{tabular}{r l l}
		\toprule
		& Deutscher Bund & Frankreich \\ \midrule \midrule
		Truppen & 200.000 & 130.000 \\ \cmidrule{2-3}
		Kanonen & 774 & 564 \\ \bottomrule
	\end{tabular}
\end{latexlisting}
Zum anderen haben wir auch die Möglichkeit, in der Layoutspezifikation vertikale Linien zu erzeugen, doppelt oder einfach:
\begin{latexlisting}
	\begin{tabular}{|r||l|l}
		\toprule
		& Deutscher Bund & Frankreich \\ \midrule \midrule
		Truppen & 200.000 & 130.000 \\ \cmidrule{2-3}
		Kanonen & 774 & 564 \\ \bottomrule
	\end{tabular}
\end{latexlisting}
Diese vertikalen Linien sehen mit den Horizontalen der \latexpackage{booktabs}-Package etwas gebrochen aus (aus Gründen, die uns gleich noch klar werden), daher sollte hier das Command \latexcommand{hline} verwendet werden, mit dem in Plain-\LaTeX{} Trennlinien erzeugt werden:
\begin{latexlisting}
	\begin{tabular}{|r||l|l}
		\hline
		& Deutscher Bund & Frankreich \\ \hline \hline
		Truppen & 200.000 & 130.000 \\ \hline
		Kanonen & 774 & 564 \\ \hline
	\end{tabular}
\end{latexlisting}
Dies sollte aber im allgemeinen nicht nötig sein, denn für typografisch gut gesetzte Tabellen gelten die folgenden drei Regeln:
\begin{enumerate}
	\item Verwende niemals vertikale Trennlinien.
	\item Verwende niemals doppelte Trennlinien.
	\item Zeilen, die gleiche Daten enthalten, sollten generell nicht getrennt werden.
\end{enumerate}
Unsere initielle Tabelle
\begin{latexlisting}
	\begin{tabular}{r l l}
		\toprule
		& Deutscher Bund & Frankreich \\ \midrule
		Truppen & 200.000 & 130.000 \\ 
		Kanonen & 774 & 564 \\ \bottomrule
	\end{tabular}
\end{latexlisting}
war also bereits das typografische Optimum.

\subsection{Mehrfachzellen}
Wollen wir mehrere Zellen verschmelzen, können wir die \latexpackage{multirow}-Package verwenden:
\begin{latexlisting}
	% Präambel
	\usepackage{multirow}
\end{latexlisting}
Diese erlaubt die Verwendung der Commands \latexcommand{multirow} und \latexcommand{multicolumn}:
\begin{latexlisting}
	\begin{table}
		\begin{tabular}{l r c c c}
			\toprule
			& & \multicolumn{2}{c}{Deutsche Koalition} & Frankreich \\  \cmidrule{3-4}
			& & Bayern & Andere & \\ \midrule
			\multirow{2}{2cm}{Tote} & Soldaten & 973 & 2.049 & 3.000 \\ 
			& Offiziere & 106 & 84 & Unbekannt \\ 
			\multirow{2}{2cm}{Verwundete} & Soldaten & 3.218 & 2.691 & 14.000 \\ 
			& Offiziere & 107 & 175 & Unbekannt \\ 
			\bottomrule
		\end{tabular}
		\caption{Tote und Verwundete}
		\label{tab:verlust-sedanschlacht}
	\end{table}
\end{latexlisting}
Zu beachten ist, dass dem Command \latexcommand{multicolumn} neben der Spaltenzahl übergeben werden muss, welche Ausrichtung der enthaltene Text haben soll.
Der Inhalt von \latexcommand{multirow} hingegen wird immer als \latexargument{p} interpretiert, mit einer Breite die ebenfalls übergeben wird.
Soll dieser Text daher zentiert oder rechtsbündig werden, muss ein Command wie \latexcommand{raggedright} verwendet werden.

\subsection{Fortgeschrittene Tabellen}
Einige formatierungstechnische Fragen bleiben vielleicht noch offen -- wollen wir beispielsweise die erste Spalte oder Zeile zur Hervorhebung fettdrucken, können wir natürlich jeder Zelle ein \latexcommand{textbf} mitgeben, dies erfüllt unser grundlegendes Prinzip, Inhalt und Formatierung trennen zu wollen, nur begrenzt.
Für solche Zwecke existieren zahlreiche Hilfspackages wie beispielsweise \latexpackage{easytable} oder \latexpackage{tabulararray}, die diese und viele weitere mächtige Optionen und Anpassungsmöglichkeiten bereitstellen, an dieser Stelle aber nicht näher diskutiert werden.

\section{Links}
Soll das erstellte Dokument nicht nur gedruckt, sondern auch digital angesehen werden, ist es oft sinnvoll, inkludierte URL-Links anklickbar zu machen.
Die Package \latexpackage{hyperref}, die wir inzwischen gut kennen, stellt hierzu einige Funktionalität bereit.
Das grundlegendste Command ist \latexcommand{url}:
\begin{latexlisting}
	Gedanken über Blocksatz findet man unter \url{https://xkcd.com/1676}.
\end{latexlisting}
Diese Methode erzeugt einen klickbaren Link, der damit sowohl für das gedruckte als auch das digitale Dokument funktioniert.
Ist man bereit, gedruckte Dokumente zu ignorieren, kann man auch den angzeigten Text abändern:
\begin{latexlisting}
	\LaTeX{}-Files sind \href{https://xkcd.com/1301}{sehr vertrauenswürdig}.
\end{latexlisting}
Das Aussehen dieser Links kann in den optionalen Argumenten von \latexcommand{usepackage} angepasst werden:
\begin{latexlisting}
	% Präambel
	\usepackage[
		colorlinks=true,
		linkcolor=black,
		urlcolor=red,
		citecolor=purple,
	]{hyperref}
\end{latexlisting}
Die optionalen Argument haben die folgende Funktion:
\begin{itemize}
	\item \latexargument{colorlinks} beeinflusst, ob Links überhaupt farbig angezeigt werden.
	Teilweise ist hier \latexargument{false} zu bevorzugen, allerdings ist es dann für den Leser nicht immer leicht erkennbar, welche Textelement klickbar sind.
	\item \latexargument{linkcolor} beeinfluss die Farbe des Linktextes -- dies schließt Referenzen, die Titel des Inhaltsverzeichnisse, etc.\ genauso ein wie die Farbe von manuell mit \latexcommand{href} erzeugten Links.
	\item \latexargument{urlcolor} beeinflusst die Farbe von mit \latexcommand{url} erzeugten Links.
	\item \latexargument{citecolor} beeinflusst die Farbe von Zitationslinks.
\end{itemize}
Insbesondere können für die letzten beiden Optionen eigene Farben verwendet werden, wenn diese zuvor mittels der \latexpackage{xcolor}-Package definiert wurden.
\begin{latexlisting}
	% Präambel
	\usepackage{xcolor}
	\definecolor{linkblue}{RGB}{58, 103, 181}
\end{latexlisting}

\section{Code}
Auch Codebeispiele bestehend aus \LaTeX{}-Code, wie beispielsweise diesem Dokument, oder beliebigen anderen Sprachen können leicht eingefügt werden.
Man nennt solche Beispiele im Englischen \emph{Listings}, und die hierzu nötige Package heißt entsprechend \latexpackage{lstlisting}.
\begin{latexlisting}
	% Präambel
	\usepackage{listings}
\end{latexlisting}
Anschließend kann im Text mittels des \latexenvironment{lstlisting}-Envirnoments Code gesetzt werden:
\begin{latexlisting}
	\begin{lstlisting}[language=C]
		float Q_rsqrt( float number )
		{
			long i;
			float x2, y;
			const float threehalfs = 1.5F;

			x2 = number * 0.5F;
			y  = number;
			i  = * ( long * ) &y;                       // evil floating point bit level hacking
			i  = 0x5f3759df - ( i >> 1 );               // what the fuck?
			y  = * ( float * ) &i;
			y  = y * ( threehalfs - ( x2 * y * y ) );   // 1st iteration
		//	y  = y * ( threehalfs - ( x2 * y * y ) );   // 2nd iteration, this can be removed

			return y;
		}
	\end{lstlisting}
\end{latexlisting}
Es mag nun auffallen, dass der gesetzte Code bisher noch extrem unansehlich ist.
Mittels des Befehls \latexcommand{lstset} können wir das Aussehen aller Listings anpassen.
Ein grundlegendes Setup sähe dabei etwa so aus:
\begin{latexlisting}
	\lstset{
	    basicstyle=\ttfamily\footnotesize,
		columns=fullflexible,
	    breakatwhitespace=false,         
	    breaklines=true,                 
	    keepspaces=true,                 
	    numbers=none,       
	    numbersep=5pt,                  
	    showspaces=false,                
	    showstringspaces=false,
	    showtabs=false,                  
	    tabsize=4,
	}
\end{latexlisting}
Ein Problem, dass bei Listings noch auftritt, ist, dass der Text innerhalb der Umgebung \emph{verbatim} interpretiert wird, also ohne Auflösung von \LaTeX{}-Macros oder sonstigen \LaTeX{}-typischen Verhaltensweisen.
Insbesondere werden die konsistenten Einrückungen vollständig abgedruckt und der Code daher etwas nach rechts verschoben.
Dies können wir durch die Package \latexpackage{lstautogobble} und die begleitende \latexcommand{lstset}-Option \latexargument{autogobble} korrigieren:
\begin{latexlisting}
	% Präämbel
	\usepackage{listings}
	\usepackage{lstautogobble}
	\lstset{
	    basicstyle=\ttfamily\footnotesize,
		...,
		tabsize=4,
		autogobble,
	}
\end{latexlisting}
Wird im gesamten Dokument dieselbe Sprache für alle Listings verwendet, so kann die \latexargument{language}-Option, die wir im optionalen Argument der Umgebung verwenden, auch in \latexcommand{lstset} gezogen werden.
Umgekehrt können alle Optionen von \latexcommand{lstset} auch in diesem optionalen Argument verwendet werden, um das Aussehen eines einzelnen Listings zu ändern.
Als letzte kosmetische Anpassung wollen wir unseren Code noch farblich hervorheben.
Hierzu verwenden wir erneut die Package \latexpackage{xcolor}:
\begin{latexlisting}
	% Präämbel
	\usepackage{listings}
	\usepackage{lstautogobble}
	\usepackage{xcolor}
	\definecolor{codegreen}{rgb}{0,0.6,0}
	\definecolor{codegray}{rgb}{0.5,0.5,0.5}
	\definecolor{codepurple}{rgb}{0.58,0,0.82}
	\definecolor{backcolour}{rgb}{0.95,0.95,0.92}
	\lstset{
	    backgroundcolor=\color{backcolour},   
	    commentstyle=\color{codegreen},
	    keywordstyle=\color{codepurple},
	    numberstyle=\tiny\color{codegray},
	    stringstyle=\color{codepurple},
	    basicstyle=\ttfamily\footnotesize,
		...,
		tabsize=4,
		autogobble,
	}
\end{latexlisting}
und erhalten so endlich das \latexcommand{lstset}-Command, mit dem dieser Text gesetzt wurde.

	\chapter{Mathematik}
Neben der wunderbaren überlegenen Typographie, dem großartigen Blocksatzalgorithmus und der Erweitarbeit ist einer der häufigsten Gründe, warum Wisschenschaftler \LaTeX{} verwenden, das formschöne Setzen mathematischer Formeln.

Noch in den 70ern wurden mathematische Paper gestaltet, indem man zuerst den Text per Hand schrieb.
Anschließend wurden mit einer Schreibmaschine die alphanumerischen Symbole getippt und die Sonderzeichen per Hand eingefügt -- wer im selben Text $a$, $\hat{a}$ und $\tilde{a}$ verwenden wollte, musste sich gut merken, wo er handschriftlich welche Verzierung einzufügen hatte.
Solche Dokumente konnten dann zur Verfielfältigung gefaxt oder fotokopiert werden.
Teilweise wurden sogar ganze Bücher mit dieser Technik geschrieben.

Über Zeit entstanden weitere Techniken, z.B. \emph{Varitype}, eine Art Schreibmaschine mit austauschbaren Rollen, auf der gleichzeitig lateinische und griechische Buchstaben geschrieben werden konnten, der \emph{IBM-Golfball}, eine Art Schreibmaschine, die einen rotierenden Golfball mit Sonderzeichen besaß, sowie \emph{tipits}, kleine, vom Nutzer anpassbare Schreibmaschinenlettern. 
Eines der dominierenden Systeme war für lange Zeit \emph{Photocomposition}, eine Schriftsatzmaschine für den Druck von Büchern oder professionellen Dokumenten, bei der statt Bleisatzlettern spezielle Belichtungsfolien verwendet wurden, von denen einige vom Nutzer mittels kleiner \emph{Symbolarme} angepasst werden konnten.
Diese Maschinen waren teilautomatisiert und übernamen Aufgaben wie Blocksatz oder Linespacing bereits selbst, einige konnten sogar vollautomatisch auf Disketten (\emph{Floppies}) gespeicherten Text setzen und en masse drucken.
Mathematik musste aber nach wie vor durch speziell ausgebildete Experten per Hand gesetzt werden.

Inspieriert von all diesen Systemen entwickelten sich schließlich \TeX{} und \LaTeX{}, die es Nutzern erstmals ermöglichen Mathematik am Computer zu spezifizieren und -- genauso wie normalen Text -- automatisch auszurichten und spacen zu lassen.
Das Ergebnis war, wie eingangs erwähnt, ein druckbares \filetype{dvi}-File.

\section{Grundlagen}
Um in \LaTeX{} Mathematik zu setzen, müssen wir in den sogeannten \emph{math mode} wechseln.
Dies geschieht mittels umgebenden Dollarzeichen:
\begin{latexlisting}
	Let $a$ be a number.
\end{latexlisting}
Man beachte den Unterschied zwischen den beiden a/$a$!
Nicht alle Zeichen sehen im Mathematikmodus anders aus, wie das folgende Beispiel zeigt (man vergleiche die 2 in 2000 mit der $2$ weiter hinten im Text):
\begin{latexlisting}
	The old greeks, more than 2000 years ago, already discussed wether the square root of $2$ was rational.
\end{latexlisting}
Trotzdem ist es aus semantischen Gründen immer sinnvoll, Variablen und mathematisch verwendete Zahlen im Mathemodus zu setzen, während z.B. Jahreszahlen als Textziffern gesetzt werden\footnote{Insbesondere mag auffallen, das je nach Schriftart eben doch ein Unterschied zwischen diesen Zahlen gemacht wird -- beispielsweise in diesem Dokument.}.
Ohne Modifikationen werden mathematische Formel außerdem kursiv gesetzt und verschlucken selbst einzelne Leerzeichen.
\begin{latexlisting}
	Dies ist ein Text. $Dies     ist auch Text und sollte nicht im Mathemodus stehen.$
\end{latexlisting}
Wer trotzdem etwas Raum haben möchte, kann die Commands \latexcommand{.}, \latexcommand{,}, \latexcommand{ }, \latexcommand{qquad} oder \latexcommand{qquad} oder das Sonderzeichen $\sim$ verwenden.
Außerdem funktionieren Trennung und Spacing im Mathematikmodus anders. So wird etwa \latexargument{sin} in
\begin{latexlisting}
	Eine trigonometrische Identität ist $sin(\pi) = 0$.
\end{latexlisting}
als \enquote{$s$ mal $i$ mal $n$} interpretiert, nicht als einzelnes Zeichen.
Es sollte stattdessen \latexcommand{mathrm\{sin\}} oder noch einfacher \latexcommand{sin} verwendet werden.
Besonderes Spacing wird auch für Operatoren verwendet:
\begin{latexlisting}
	Die einzige Lösung der Gleichung $2a - 23 = -5$ ist $a = 9$.
\end{latexlisting}
Man beachte, dass das Zeichen \key{-} nicht als Gedanken- oder Spiegelstrich, sondern als Minus interpretiert wird, und das Spacing um das unäre Minus bei $-5$ anders ist als das um $2a - 23$.
Dies ändert sich auch nicht, wenn die Leerzeichen in der Gleichung verändert werden -- es ist zwar gute \LaTeX{}-Praxis, Mathematik mit sinnvollen Leerzeichen zu schreiben, die den Code gut lesbar machen, aber es ist nicht zwingend notwendig.

Neben dieser sogenannten \emph{inline math} existiert auch \emph{display math}, d.h. Formeln, die abgehoben vom Text in einer eigenen Zeile gesetzt werden.
Die Begrenzer (engl. \emph{delimiters}) hierfür sind Doppeldollarzeichen:
\begin{latexlisting}
	Die Dreiecksungleichung $$a + b > c$$ ist in der nicht-euklidischen Geometrie manchmal falsch.
\end{latexlisting}
Für gewöhnlich wird Display-Mathematik wie eine Umgebung in einer eigenen Zeile und ein wenig eingerückt geschrieben:
\begin{latexlisting}
	Die Dreiecksungleichung
	$$
		a + b > c
	$$
	ist in der nicht-euklidischen Geometrie manchmal falsch.
\end{latexlisting}
Die Delimitation durch Dollarzeichen ist der Standard in \TeX{}.
\LaTeX{} führte stattdessen die Begrenzer \latexcommand{(}, \latexcommand{)} sowie \latexcommand{[}, \latexcommand{]} ein, also
\begin{latexlisting}
	Die einzige Lösung der Gleichung \(2a - 23 = -5\) ist \(a = 9\).
	Die Dreiecksungleichung
	\[
		a + b > c
	\]
	ist in der nicht-euklidischen Geometrie manchmal falsch.
\end{latexlisting}
und die Package \latexpackage{amsmath} (für \emph{American Mathematical Society}) stellt noch einige weitere Möglichkeiten, Displaymathematik zu begrenzen, zur Verfügung.
\begin{latexlisting}
	Die Dreiecksungleichung
	\begin{equation}
		a + b > c
	\end{equation}
	ist in der nicht-euklidischen Geometrie manchmal falsch.

	Die Dreiecksungleichung
	\begin{equation*}
		a + b > c
	\end{equation*}
	ist in der nicht-euklidischen Geometrie manchmal falsch.
\end{latexlisting}
Die Umgebung \latexenvironment{equation} ist dabei wie \latexcommand{[}, \latexcommand{]} einfach \enquote{nur} Displaymathematik ohne zusätzliche Funktionalitäten (tatsächlich definiert die \latexpackage{amsmath}-Package sogar \latexcommand{[}, \latexcommand{]} durch \latexenvironment{equation} um, da die Implementierung in \latexpackage{amsmath} leicht besseres Spacing hat).
Umgebungen ohne Sternchen erhalten in \latexpackage{amsmath} stets eine Nummerierung, solche mit Sternchen werden durchnummeriert.
Eine \latexenvironment{equation}-Umgebung erlaubt generell nur eine Zeile und schießt, falls die Formel zu lange ist, über den Rand hinaus.
Wer einen Zeilenumbruch in seiner Formel möchte, kann die \latexenvironment{gather}-Umgebung verwenden.
\begin{latexlisting}
	Wir lösen die Gleichung:
	\begin{gather*}
		2a + 7 = 25 \\
		2a = 18\\
		a = 9
	\end{gather*}
\end{latexlisting}
Auch diese existiert in einer gesternten und ungesternten Version, und -- ähnlich wie Tabellen -- sind hier Zeilenumbrüche mit \latexcommand{\textbackslash} sogar erwünscht.
Möchte man die verschiedenen Zeilen noch aneinander ausrichten, so bietet sich die \latexenvironment{align}-Umgebung an.
\begin{latexlisting}
	Wir berechnen:
	\begin{gather*}
		2a + 7(a + 5) - 13a + 12
		&= 2a + 7a + 35 - 13a + 12 \\
		&= (2 + 7 - 13)a + 35 + 12 \\
		&= -4a + 47
	\end{gather*}
\end{latexlisting}
Wie in Tabellen werden \key{\&}-Zeichen zur Spaltentrennung verwendet.
Hierbei sind die Spalten abwechselnd rechts- und linksbündig ausgerichtet, was für \enquote{Kommentarspalten} verwendet werden kann.
\begin{latexlisting}
	Wir berechnen:
	\begin{gather*}
		7a + 5
		&= 3a - 7
		&| -5 \\
		7a 
		&= 3a - 12
		&| -3a \\
		4a
		&= -12
		&| :4 \\
		a
		&= -3
	\end{gather*}
\end{latexlisting}
Es ist hierbei -- gerade bei längeren Zeilen -- guter Stil, jede Spalte in eine eigene Zeile zu setzen.

Wie wählen wir nun zwischen all diesen Umbegungen aus?
Für \emph{inline math} wäre theoretisch \latexcommand{(}, \latexcommand{)} die optimale Wahl, praktisch hat sich aber aufgrund des geringeren Tippaufwands und der typografischen Äquivalenz die Verwendung von \key{\$} durchgesetzt.
Für \emph{display math} gilt dies nicht!
Das alte \key{\$\$} hat merklich schlechteres Spacing und sollte \emph{niemals} verwendet werden!
Stattdessen sind immer die passenden \latexpackage{amsmath}-Commands zu nutzen, \latexenvironment{align} und \latexenvironment{gather} wo nötig und \latexenvironment{equation} überall sonst (natürlich je nach Wunsch in der unnummerierten Version).

\section{Sehr viel Syntax}
In diesem Unterkapitel besprechen wir eine lange Liste an Befehlen und Schreibweisen, die wir zum Setzen von Mathematik benötigen.

\subsection{Hoch- und Tiefstellung}
Hoch- und Tiefstellung erreichen wir durch die Symbole\ \key{~\^} und \key{\_}.
\begin{latexlisting}
	Meine Lieblings-Quadratzahl ist $q_2 = 2^2 = 4$.
\end{latexlisting}
Wir können diese auch an einem einzigen Symbol kombinieren.
\begin{latexlisting}
	Das Zeichen für Beryllium ist $Be^7_4$, oder besser $\mathrm{Be}^7_4$.
\end{latexlisting}
Es gibt hier keine kanonische Reihenfolge von Hoch- und Tiefstellung, aber es lohnt sich, zumindest selbst konsistent zu sein.
Mehrere Hochstellungen am selben Symbol hingegen erweisen sich als problematisch (und auch uneindeutig):
\begin{latexlisting}
	Berechne $2^2^3$. Ist es $4^3 = 64$ oder $2^8 = 256$?
\end{latexlisting}
Hier können teilweise Klamern helfen:
\begin{latexlisting}
	Ich meine natürlich $(2^2)^3 = 64$.
\end{latexlisting}
Das funktioniert aber nur, solange nur ein Zeichen hochgestellt werden soll, das folgende sieht etwas seltsam aus.
\begin{latexlisting}
	Oh, meinte doch $2^(2^3)$.
\end{latexlisting}
Mehrere Zeichen auf einmal können stattdessen mittels einer \emph{Gruppe}, die durch \key{\{} und \key{\}} begrenzt wird, hochgestellt werden:
\begin{latexlisting}
	Oh, meinte doch $2^{2^3} = 256$, oder noch klarer $2^{(2^3)} = 256$.
\end{latexlisting}
Die gleichen Techniken funktionieren natürlich auch für Tiefstellung, und können auch genutzt werden, um Hoch- und Tiefstellung verschieden zu kombinieren:
\begin{latexlisting}
	Betrachte das $4$-te Folgendglied, $a_{2^2}$.
\end{latexlisting}

\subsection{Schriftarten}
Um mathematische Symbole voneinander abzugrenzen, ist es oft üblich, sie in speziellen Schriftarten zu schreiben, allen voran die reellen Zahlen.
Die Package \latexpackage{amssymb} stellt hierzu einige nette Commands bereit.
\begin{latexlisting}
    Die reellen Zahlen sind $\mathbb{R}$.
    Die Menge aller reellwertigen $k$-mal differenzierbaren Funktionen auf einer Mannigfaltigkeit $M$ schreiben wir als $\mathcal{C}^k(M)$.
    Die Lie-Algebra der Lie-Gruppe $\mathscr{G}$ bezeichnen wir mit $\mathfrak{g}$.
\end{latexlisting}
Dies sind die Commands für \emph{Blackboard} (doppelgestrichene Schrift), \emph{caligraphy} (geschwungene Schrift), \emph{script} (anders geschwungene Schrift) sowie \emph{fraktur} (Frakturschrift).
Leider sind insbesondere in der \latexcommand{mathbb}-Schrift nicht alle Symbole enthalten -- beispielsweise fehlt die $\mathds{1}$.
Hierzu können wir stattdessen die Package \latexpackage{dsfont} und deren Command verwenden.
\begin{latexlisting}
	Das Symbol $\mathbb{1}$ gibt es nicht, $\mathds{1}$ aber schon.
\end{latexlisting}
Leider sind diese Commands semantisch nicht sehr wertvoll.
Sie kommunizieren nur, dass etwas doppelt gestrichen werden soll, nicht aber \emph{warum}.
Es wäre daher ratsam, eigene Commands definieren, die die obigen aliasen, aber semantisch wertvoll sind, etwa wie folgt\footnote{Der genaue Syntax der Commanddefinition wird später erklärt und muss hier nicht verstanden werden -- das Beispiel dient nur der Illustration.}:
\begin{latexlisting}
	\newcommand{\reals}{\mathds{R}}
	\newcommand{\field}[1]{\mathds{#1}}
	Die reellen Zahlen sind $\reals$ oder $\field{R}$.
\end{latexlisting}
Die Package \latexpackage{nchairx}, die hauseigen in Würzburg am Lehrstuhl 10 (Chair X) entwickelt wurde, nimmt einem einige dieser Arbeit ab und definiert semantisch sinnvolle mathematische Befehle (hauptsächlich für Differentialgeometrie, Analysis und lineare Algebra).
Dies erlaubt die Verwendung vieler semantischer Befehle.
\begin{latexlisting}
    Die reellen Zahlen sind $\field{R}$.
    Die Menge aller reellwertigen $k$-mal differenzierbaren Funktionen auf einer Mannigfaltigkeit $M$ schreiben wir als $\module{C}^k(M)$.
    Die Lie-Algebra der Lie-Gruppe $\group{G}$ bezeichnen wir mit $\liealg{g}$.
\end{latexlisting}

\subsection{Viele viele bunte Commands}

	\chapter{Commands}

In den vorhergehenden 5 Kapiteln haben wir viele der in \LaTeX{} enthaltenen Features kennengelernt und benutzt.
Eine der großen Stärken von \LaTeX{} ist es aber, dass wir diese Funktionalitäten selbstständig erweitern und anpassen können.
In diesem Kapitel lernen wir, eigene Command zu definieren -- von simplen Abkürzungen bis hin zu komplexen Funktionen mit mehreren Argumenten.

\section{Makros in \TeX}

Die einfachste Art, ein neues Makro zu definieren, ist das \TeX -Command \latexcommand{def}.
\begin{latexlisting}
	\def\derivative{\del}

	Betrachte $\derivative x$.
\end{latexlisting}
Was passiert hier?
\LaTeX{} evaluiert Commands in einem Prozess namens \emph{expansion}, d.h. immer wenn \LaTeX{} ein \emph{Token} findet, das expandierbar ist (beispielsweise \latexcommand{derivative}), so wir dieses durch seine Definition ersetzt, in diesem Fall \latexcommand{del}.
Auch \latexcommand{del} ist expandierbar, wird also erneut ersetzt.
Dieser Prozess wird so lange wiederholt, bis wir bei nur un-expandierbaren Tokens -- hier der Unicode-Character der partiellen Ableitung -- angekommen sind, wie beispielsweise Buchstaben oder einige spezielle Anweisungen wie \latexcommand{def}.
Wann und wie lange \LaTeX{} expandiert, ist kompliziert.
Man betrachte das folgende Beispiel und überlege sich, was das zu erwartende Ergebnis ist.
\begin{latexlisting}
	\def\aaa{Hello}
	\def\bbb{\aaa}
	\def\aaa{Hallo}

	Ich sage \bbb .
\end{latexlisting}
Was passiert, wenn man das \latexcommand{def} in der zweiten Zeile durch \latexcommand{edef} ersetzt?
Wir sehen, dass \LaTeX{} in einigen Kontexten (wie beispielsweise dem hinteren Teil von \latexcommand{def}) nicht sofort expandiert.
Theoretisch sind die geschweiften Klammern um die Definition nicht nötig, wenn es sich nur um ein einzelnes Token handelt, aber sie sind trotzdem guter Stil.
Weiterhin sollten solche Definitionen auch nur in der Präambel, vorzugsweise in einem eigenen File mit einen Titel wie \filepath{macros.tex}, leben und nicht im Fließtext, auch wenn dies möglich ist.
Neben \latexcommand{def} und \latexcommand{edef} existiert auch \latexcommand{let}, das seine Definition nur ein einziges Mal expandiert, bevor das Makro festgelegt wird.
Somit kann \latexcommand{let} genutzt werden, um einem Makro einen neuen Namen zu geben und wird daher oft gern genutzt, um Definitionen zu vertauschen, etwa wir folgt:
\begin{latexlisting}
	\let\oldepsilon\epsilon
	\let\epsilon\varepsilon
	\let\varepsilon\oldepsilon
\end{latexlisting}
Mithilfe des un-expandierbaren Command \latexcommand{undefined} können wir existierende Makros löschen.
Dies ist beispielsweise nützlich, um Makro-Konflikte zwischen verschiedenen Packages zu vermeiden.
\begin{latexlisting}
	% Präambel
	\usepackage{nchairx}
	\let\unit\undefined
	\usepackage{siunitx}
\end{latexlisting}

\subsection{Blöcke}

Weiterhin sei noch erwähnt, dass Makros sich auf Blöcke beschränken.
\begin{latexlisting}
	\def\aaa{Hello}
	{
		\def\aaa{Hallo}
		Ich sage \aaa .
	}
	Ich sage \aaa .
\end{latexlisting}
Dieses Verhalten lässt sich mittels des (nicht-expandierbaren) Makros \latexcommand{global} deaktivieren.
\begin{latexlisting}
	\def\aaa{Hello}
	{
		\global\def\aaa{Hallo}
		Ich sage \aaa .
	}
	Ich sage \aaa .
\end{latexlisting}
Beide diese Features sind allerdings meist nur in Situationen nützlich, in denen man spaßige Dinge mit dem \TeX-Makro-System anstellen möchte.

\section{Makros in \LaTeX}

Statt des alten \TeX-Systems ist es empfehlenswert, das modernere und sicherere \LaTeX-Makro-System zu nutzen.

\subsection{Makros ohne Argumente}
Zuerste betrachten wir wieder einfache Makros wie oben, die keine Argumente akzeptieren.
Statt \latexcommand{def} nutzen wir hier \latexcommand{newcommand}:
\begin{latexlisting}
	\newcommand{\integers}{\mathds{Z}}

	Die ganzen Zahlen schreiben wir als $\integers$.
\end{latexlisting}
Use-Cases für diese Art von simplen Makros sind vielfältig:
\begin{itemize}
	\item Verkürzung langer Fachbegriffe
	\item Vermeidung von Rechtschreibfehlern:
	\begin{latexlisting}
		\newcommand{\DNA}{Desoxyribonukleinsäure}
	\end{latexlisting}
	\item Semantisierung, insbesondere in mathematischen Formeln
	\begin{latexlisting}
		\newcommand{\iff}{\Leftrightarrow}
	\end{latexlisting}
	\item Vermindeter Schreibaufwand
	\item Einheitliche Formatierung zentraler Begriffe
	\begin{latexlisting}
		\newcommand{\germany}{\includegraphics[height=0.6\baselineskip]{germany-flag.png} \textbf{Germany}}
	\end{latexlisting}
	\item Erlaubt zentrale Änderungen
	\begin{latexlisting}
		\newcommand{\toolname}{NetVisard}
	\end{latexlisting}
\end{itemize}

\subsection{Makros mit Argumenten}
Zusätzlich können wir aber auch Makros definieren, deren Verhalten sich mithilfe von Argumenten anpassen lässt und die dadurch viel flexibler sind als die vorher bekannten\footnote{Es sei erwähnt, dass ich dies auch mit \latexcommand{def} umsetzen lässt, der Syntax ist nur weniger durchschaubar}.
\begin{latexlisting}
	\newcommand{\differential}[1]{\frac{\D}{\D #1}}

	Betrachte $\differential{x} f(x) = \differential{t} f(t)$.
\end{latexlisting}
Hier geben wir in eckigen Klammern die Anzahl der zu erwartenden Argumente an und können diese dann im Körper des Makros mit \latexargument{\#1}, \latexargument{\#2}, etc. nutzen (und sie werden zur Auswertungszeit expandiert).
Dies kann zur Semantiesierung von Formatierungsentscheidungen nützlich sein.
Das \filepath{macros.tex}-File dieses Dokuments, beispielsweise, besteht zu einem großen Teil aus verschiedenen Varianten, Schrift in Monospace zu setzen.
\begin{latexlisting}
	\newcommand{\latexcommand}[1]{\texttt{\textbackslash #1}}
	\newcommand{\latexpackage}[1]{\texttt{#1}}
	\newcommand{\latexargument}[1]{\texttt{#1}}
	\newcommand{\latexenvironment}[1]{\texttt{#1}}
	\newcommand{\filetype}[1]{.\texttt{#1}}
	\newcommand{\filepath}[1]{\texttt{#1}}
	\newcommand{\key}[1]{\texttt{#1}}
\end{latexlisting}
Wir können natürlich auch mehrere Argumente verwenden:
\begin{latexlisting}
	\newcommand{\add}[2]{#1 + #2 = \fpeval{#1 + #2}}

	Betrachte $\add{3}{5}$
\end{latexlisting}
Wobei das Command \latexcommand{fpeval} (bereitgestellt durch eine Package, die in KOMA bereits enthalten ist) in \LaTeX{} während des Kompilierungsprozesses Rechnungen durchführen kann.
Außerdem ist es möglich, Default-Werte für Argumente anzugeben und diese Argumente dadurch in optionale umzuwandeln:
\begin{latexlisting}
	\newcommand{\reals}[1][n]{\mathbb{R}^{#1}}

	Es ist $\reals = \reals[4]$ für $n = 4$.
\end{latexlisting}
Commands mit Argumenten haben erneut alle Vorteile derer ohne Argumente, und zusätzlich:
\begin{itemize}
	\item Deutlich weniger Schreibaufwand.
	\item Vermeidung von Code-Duplikation.
	\item Kontextsensitive Anpassung.
	\item Übersichtlicherer Code.
\end{itemize}
Abschließend sei noch ein Command definiert, dass mir bei der Erstellung von Arbeitsblättern sehr nützlich war und das wir im Laufe des nächsten Abschnitts erweitern werden:
\begin{latexlisting}
	\newcommand{\blank}[1]{%
		\underline{\phantom{#1#1#1}}%
	}
\end{latexlisting}
Die Kommentare am Ende der Zeile sind in \LaTeX{}-Commands, die mehrere Zeilen überspannen oft üblich und verhindern, dass sich ungewollte Leerzeichen oder gar Zeilenumbrüche bei der Verwendung einschleichen.
Dieses Command erzeugt eine unterstrichene Lücke groß genug, dass handschriftlich das passende Wort gut hineinpasst.
Später werden wir das Command so modifieren, dass auf einfache Art und Weise auch eine Lösung erstellt werden kann, in der das gewollte Wort bereits eingetragen ist.

\section{Kontrollstrukturen}


	\section{Literatur}

Ein gutes wissenschaftliches Dokument enthält eine Vielzahl an Quellen und Referenzen.
Natürlich stellt \LaTeX{} auch hierfür einige praktische Tools bereit, und wie üblich sind im Laufe der Geschichte einige standardmäßige Packages hinzugekommen, die wir im Folgenden kennenlernen werden.
Zu allem, was wir im Folgenden lernen, ist zusätzlich auch das Cheatsheet \url{https://tug.ctan.org/info/biblatex-cheatsheet/biblatex-cheatsheet.pdf} wärmstens zu empfehlen.

\subsection{Das Bibfile}
Unser gutes Prinzip der Trennung von Inhalt und Format fortführend, werden Quellen in \LaTeX{} in einem eigenen File, dem sogenannten \emph{bibfile} gespeichert.
Dieses dient hier als eine Art Datenbank, in der Einträge für alle möglichen Quellen eingetragen werden.
Ein Dokument kann dann eines oder mehrere solche Bibfiles einbinden und daraus zitieren, wobei die Zitationen automatisch mit den gespeicherten Daten gefüllt werden.
Quellen, die nicht zitiert werden, tauchen auch nicht im automatischen Literaturverzeichnis auf, so dass ein Bibfile deutlich mehr Quellen enthalten kann als nötig -- dies erlaubt es erneut, ein Bibfile zwischen vielen Dokumenten und Nutzern zu teilen.
Der Lehrstuhl X beispielsweise verwaltet ein Git-Repository mit einer Vielzahl an Bibfiles, die insgesamt Hunderte forschungsrelevante Quellen enthalten.

Ein Bibfile hat für gewöhnlich die Endung \filetype{bib} und einen Namen wie \filepath{bibliography.bib} oder \filepath{references.bib}.
Das Format der Einträge setzt wie folgt zusammen:
\begin{latexlisting}
	@book{mussmaecher:2025,
		title = {Schreiben wissenschaftlicher Arbeiten in LaTeX},
		author = {Mußmächer, Linus},
		year = {2025},
		publisher = {Universtität Würzburg},
	}
\end{latexlisting}
Wir beginnen mit dem Typ des Eintrags nach einem \key{\@}-Zeichen.
Mögliche Typen sind in \autoref{tab:bib-types} aufgeführt.
In geschweiften Klammern folgend anschließend die Daten des Eintrags, allen voran die ID.
Die ID ist der einzige Punkt, der unter allen verwendeten Bibliographien eindeutig sein muss und sollte sinnvoll gewähnlt werden.
Eine übliche Konvention ist beispielsweise
\begin{latexlisting}
	<autor1>-<autor2>:<jahr>
\end{latexlisting}
wobei Kleinbuchstaben angehängt werden, um Veröffentlichungen mit gleichen Autoren und Jahr zu unterscheiden.
Die übrigen Einträge sind im Key-Value-Format und enthalten Informationen über die Veröffentlichung.
Einige mögliche Keys sind in \autoref{tab:bib-keys} aufgeführt, alle weiteren finden sich im Cheatsheet.

\begin{table}
	\begin{tabular}{l p{6cm}}
		\toprule
		\textbf{Typ} & \tabularnewline
		\midrule
		\latexargument{book}
		& Bücher
		\tabularnewline
		\latexargument{article}
		& Artikel, Paper, Skript
		\tabularnewline
		\latexargument{online}
		& Webquelle
		\tabularnewline
		\latexargument{misc}
		& Sonstiges
		\tabularnewline
		\bottomrule
	\end{tabular}
	\caption{Bibliographieeintragstypen}
	\label{tab:bib-types}
\end{table}
\begin{table}
	\begin{tabular}{l p{8cm}}
		\toprule
		\textbf{Key} & \textbf{Bedeutung} \tabularnewline
		\midrule
		\latexargument{author}
		& Autor
		\tabularnewline
		\latexargument{title}
		& Titel
		\tabularnewline
		\latexargument{publisher}
		& Verlag
		\tabularnewline
		\latexargument{adress}
		& Adresse des Verlags
		\tabularnewline
		\latexargument{year}
		& Jahr der Veröffentlichung
		\tabularnewline
		\latexargument{url}
		& Web-Adresse
		\tabularnewline
		\latexargument{volume},\latexargument{number} 
		& Für Objekte in Journalen oder Sammlungen
		\tabularnewline
		\latexargument{comment}
		& Weitere Anmerkungen
		\tabularnewline
		\bottomrule
	\end{tabular}
	\caption{Verhalten von Command-Erzeugern}
	\label{tab:bib-keys}
\end{table}
Beim Feld \latexargument{author} ist zu beachten, dass Autoren im Format \latexargument{<nachname>, <vorname>} angegeben und mehrere Autoren immer (alle!) mit \latexargument{and} getrennt werden.
\LaTeX{} übernimmt dann selbständig die Formatierung der Autoren.
Viele Seiten, die wissenschaftliche Artikel bereitstellen (z.B. arXiV), können auch automatisch Bib-Einträge mit allen ihnen bekannten Informationen erzeugen.
Ähnliches gilt für die meisten handelsüblichen Zitationssoftware wie Zotero, Citavi etc.

Um ein solches Bibfile nun einbinden zu können, benötigen wir zuerst die Package \latexpackage{biblatex}.
In TeXStudio ist es außerdem nötig, entweder in den Einstellungen das Bibliography-Backend auf \latexargument{biber} umzustellen oder die Option
\begin{latexlisting}
	% Präambel
	\usepackage[backend=bibtex]{biblatex}
\end{latexlisting}
zu verwenden, damit das von TeXStudio verwendete Backend mit dem von \latexpackage{biblatex} erwarteten übereinstimmt.
In Overleaf ist nichts derartige nötig.
Ist die Package eingebunden, können wir mit
\begin{latexlisting}
	% Präambel
	\addbibresource{references.bib}
\end{latexlisting}
Wie bereits erwähnt können auf diese Art und Weise mehrere Bibfiles eingebunden werden.

\subsection{Zitationen}
Ist die Bibliographie erst einmal verlinkt, ist das Zitieren von Quellen recht einfach.
\begin{latexlisting}
	Wie man Quellen zitiert, haben wir aus \cite{mussmaecher:2025} gelernt.
\end{latexlisting}
Verwendet man dabei die \latexpackage{hyperref}-Package, wird das Zitat sogar zum Literaturverzeichnis (siehe \autoref{sec:literaturverzeichnis}) verlinkt.
Mehrere Quellen können auch auf einmal zitiert werden
\begin{latexlisting}
	Wie man Quellen zitiert, haben wir aus \cite{mussmaecher:2025,mussmaecher:2025a} gelernt.
\end{latexlisting}
Um anzupassen, wie eine solche Zitation aussieht, können wir den \emph{Zitationsstil} anpassen.
\begin{latexlisting}
	\usepackage[citestyle=alphabetic]{biblatex}
\end{latexlisting}
Mögliche Zitationsstile sind in \autoref{tab:cite-styles} aufgeführt.
\begin{table}
	\begin{tabular}{l l}
		\toprule
		\latexargument{numeric} &
		\latexargument{alphabetic} \tabularnewline
		\latexargument{authoryear} &
		\latexargument{authortitle} \tabularnewline 
		\latexargument{verbose} &
		\latexargument{reading} \tabularnewline
		\latexargument{draft} &
		\dots \tabularnewline
		\bottomrule
	\end{tabular}
	\caption{Zitationsstile}
	\label{tab:cite-styles}
\end{table}
Mittels eines optionalen Arguments können wir außerdem Anmerkungen zu unserer Zitation hinzufügen:
\begin{latexlisting}
	Wie man Quellen zitiert, haben wir aus \cite[Kapitel 7]{mussmaecher:2025} gelernt.
\end{latexlisting}
Besteht diese Anmerkunge nur aus einer Seite, genügt es, die Zahl anzugeben.
\begin{latexlisting}
	Wie man Quellen zitiert, haben wir aus \cite[69]{mussmaecher:2025} gelernt.
\end{latexlisting}
Die Packages \latexpackage{biblatex} und \latexpackage{babel} kümmern sich dann um ein sprachlich richtiges Label wie \enquote{S.} oder \enquote{pp.}.
Je nach PDF-Viewer sieht man hier eventuell grüne Ränder um seine Zitationen -- diese tauchen beim Drucken nicht auf, und können im PDF mittels einer Option der \latexpackage{hyperref}-Package unterdrückt werden.
\begin{latexlisting}
	\usepackage[hidelinks=true]{hyperref}
\end{latexlisting}
Neben dem gewöhnlichen Command \latexcommand{cite} existieren auch einige Alternativen, die in speziellen Situationen verwendet werden können.
\begin{latexlisting}
	In \textcite{mussmaecher:2025} finden wir interessante Informationen, auch das andere Buch \parencite{mussmaecher:2025a} ist toll.
	Eventuell findet man auch anderswo\footcite{mussmaecher:2025} etwas?
	Oder hier\supercite{mussmaecher:2025a}?
\end{latexlisting}

\subsection{Literaturverzeichnis}\label{sec:literaturverzeichnis}
Haben wir nun alle Quellen zitiert, können wir uns mittels \latexcommand{printbibliography} ein Literaturverzeichnis anzeigen lassen.
\begin{latexlisting}
	\printbibliography
\end{latexlisting}
Zur Sortierung dieses Literaturverzeichnises existiert wieder eine Package-Option.
\begin{latexlisting}
	\usepackage[sorting=nyt]{biblatex}
\end{latexlisting}
Mögliche Optionen sind dabei \latexargument{nyt}, \latexargument{nty}, \latexargument{ynt} und \latexargument{ydnt}, wobei die Buchstaben für \enquote{Name}, \enquote{Author} und \enquote{Year} (sowie \enquote{descending}) stehen.
Andere Sortierung sind standardmäßig nicht möglich.
Der Stil der Bibliographie kann ähnlich wie der Zitierstil angepasst werden.
\begin{latexlisting}
	\usepackage[sorting=nyt,bibstyle=numeric]{biblatex}
\end{latexlisting}
wobei man für gewöhnlich Zitier- und Bibliographiestil gemeinsam festlegt.
\begin{latexlisting}
	\usepackage[sorting=nyt,style=numeric]{biblatex}
\end{latexlisting}
Möglich Stile sind in \autoref{tab:bib-styles} aufgeführt.
\begin{table}
	\begin{tabular}{l l l l}
		\toprule
		\latexargument{numeric} &
		\latexargument{numeric-comp} &
		\latexargument{numeric-verb} &
		\tabularnewline
		\latexargument{alphabetic} &
		\latexargument{alphabetic-verb} &
		&
		\tabularnewline
		\latexargument{authoryear} &
		\latexargument{authoryear-comp} &
		\latexargument{authoryear-ibid} &
		\latexargument{authoryear-icomp} \tabularnewline
		\latexargument{authortitle} &
		\latexargument{authortitle-comp} &
		\latexargument{authortitle-ibid} &
		\latexargument{authortitle-icomp} \tabularnewline
		\latexargument{verbose} &
		\latexargument{reading} &
		\latexargument{draft} &
		\dots \tabularnewline		
		\bottomrule
	\end{tabular}
	\caption{Zitationsstile}
	\label{tab:bib-styles}
\end{table}
Es existieren noch weitere Stile, die von \latexpackage{biblatex} automatisch geladen werden können, z.B. \latexargument{apa}.
Man kann sich hierzu im Internet (\url{ctan.org}) oder dem Cheatsheet informieren.
Weiterhin können wir sogenannte Rückverweise in der Bibliographie anzeigen.
\begin{latexlisting}
	\usepackage[backref=true]{biblatex}
\end{latexlisting}
Ein letztes nützliches Feature ist die Möglichkeit, die Bibliographie nach dem Typ der Zitation aufzusplitten.
\begin{latexlisting}
	\printbibliography[title={Bücher}, type=book]
	\printbibliography[title={Andere Quellen}, nottype=book]
\end{latexlisting}

\subsection{Glossare}
In langen Texten ist oft nützlich ein Glossar an wichtigen Begriffen und Definitionen zu halten.
Hierzu existiert die Package \latexpackage{makeidx}.
Mittels des Commands \latexcommand{index} können Indexeinträge erstellt werden.
\begin{latexlisting}
	\begin{definition}[Magma]\label{def:magma}\index{Magma}
		Eine Menge $M$ gemeinsam mit einer zweistelligen Verknüpfung $\cdot: M \times M \to M$ heißt \emph{Magma.}
	\end{definition}
\end{latexlisting}
Anschließend muss in der Präambel, nach Inklusion der Package aber vor Beginn des Dokuments, das Command \latexcommand{makeindex} inkludiert werden.
Alles was noch zu tun ist, ist das fertige Glossar an einer angemessenen Stelle (im Anhang, seltener am Anfang zwischen Vorwort und Einleitung) durch \latexcommand{printindex} abzudrucken.
Mit der Erstellung eines Index sind wieder einige Kompilierungsprozesse und Hilfsfiles verbunden, um die sich glücklicherweise Overleaf oder Hilfsprogramme wie \filepath{latexmk} automatisch kümmern -- in TeXStudio ist es nötig, das Projekt zu kompilieren, dann unter \texttt{Tools/Index} den Index zu bauen und erneut zu kompilieren.
% Siehe: https://tex.stackexchange.com/questions/298271/makeindex-is-not-called-when-compiling-in-texstudio

Möchten wir \emph{Nomenklaturen} erzeugen, eine spezielle Art von Index, in der auch Kurzdefinitionen enthalten sind, können wir stattdessen die Package \latexpackage{nomencl} verwenden, deren Dokumentation unter \url{https://ctan.net/macros/latex/contrib/nomencl/nomencl.pdf} ihre Verwendung erklärt.

	\section{TikZ}

Die Package \emph{tikz} erlaubt in \LaTeX{} die Verwendung der Sprache \emph{TikZ} (kurz für \emph{TikZ ist kein Zeichenprogramm}), mit der sich Vektorgraphiken beschreiben lassen.
TikZ wurde speziell für die Verwendung mit \LaTeX{} entwickelt und erlaubt die Erstellung von zahlreichen Vektorgraphiken: Bilder, Diagramme, Plots, Symbole, Grundrisse, etc.
Eine vollständige Spezifikation und viele nützliche Hinweise lassen sich unter \url{tikz.dev} nachlesen.

Ähnlich wie \LaTeX{} selbst kann TikZ durch viele Libraries erweitert werden.
Dies geschieht mittels des Commands \latexcommand{usetikzlibrary}.
Libraries können klein sein und nur neue Commands hinzufügen, bestehenden Commands mehr Optionen geben (z.B. die Library \latexpackage{patterns}, die Musterfüllungen erlaubt) oder gar ganz neue Umgebungen und syntaktische Konstrukte einführen (z.B. die Library \latexpackage{cd}, die zum Erstellen kommutativer Diagramme verwendet wird).

Da TikZ-Code oft etwas mühsam zu schreiben ist, existieren Online-Lösungen wie \url{q.uiver.app} oder \url{tikzmaker.com}, die den User in einer graphischen Umgebungen Diagramme zeichnen und dann generierten Code in \LaTeX{} einfügen lassen.

Die allgemeine Umgebung zur Verwendung von TikZ ist \latexenvironment{tikzpicture}.
Wir nehmen im Folgenden an, dass sich alle Beispiele in diesem Kapitel -- sofern nicht anders erwähnt -- in einer solchen \latexenvironment{tikzpicture}-Umgebung befinden.

\subsection{Linien \& Formen}

Die grundlegendste in TikZ enthaltene Form ist eine einfache Linie.
\begin{latexlisting}
	\draw (0,0) -- (1,0);
\end{latexlisting}
Das Command \latexcommand{draw} zeichnet die darauffolgende Form, hier ein Strich von \latexargument{(0,0)} nach \latexargument{(1,0)}.
Koordinaten werden hier im mathematischen Sinne verwendet, d.h. die positiven Richtungen sind oben und rechts.
Da es sich um europäisches Package handelt, sind alle Angaben -- außer anders konfiguriert -- in \latexargument{cm}.
Die Form wird durch ein Semikolon (\key{;}) abgeschlossen.
Eine Form kann auch eine Folge an Linien sein.
\begin{latexlisting}
	\draw (0,0) -- (1,0) -- (0.5,0.5) -- (0,0);
\end{latexlisting}
Soll eine Folge an Linien geschlossen werden, kann auch der \enquote{Punkt} \latexargument{cycle} verwendet werden.
\begin{latexlisting}
	\draw (0,0) -- (1,0) -- (0.5,0.5) -- cycle;
\end{latexlisting}
Ähnliche wie viele Strukturen in \LaTeX{} können wir Linien mit optionalen Argumenten ausstatten.
Statt \latexargument{--} muss dann aber \latexargument{to} verwendet werden.
\begin{latexlisting}
	\draw (0,0) to[bend left] (1,0) -- (0.5,0.5) to[bend right,distance=0.1cm] cycle;
\end{latexlisting}
So können wir auch die Stärke und Art der Linie anpassen:
\begin{latexlisting}
	\draw[thick] (0,6) to (1,6);
	\draw[ultra thin] (0,5) to (1,5);
	\draw[color=myownblue] (0,4) to (1,4);
	\draw[color=red] (0,3) to (1,3);
	\draw[dashed] (0,2) to (1,2);
	\draw[dotted] (0,1) to (1,1);
	\draw (0,0) to (1,0);
\end{latexlisting}
Wie wir sehen, können erneut auch durch \latexpackage{xcolor} selbst definierte Farben verwendet werden.
Die Standard-Liniendicke ist \latexargument{thin}.
Argumente können direkt an \latexargument{to} geschrieben werden, oder an \latexcommand{draw}, oder an das gesamte \latexenvironment{tikzpicture}.
\begin{latexlisting}
	\begin{tikzpicture}
		\draw (0,0) to[dashed] (1,0) -- (0.5,0.5) -- cycle;
		\draw[dashed] (0,2) to (1,2) -- (0.5,2.5) -- cycle;
	\end{tikzpicture}

	\begin{tikzpicture}[dashed]
		\draw (0,0) to (1,0) -- (0.5,0.5) -- cycle;
		\draw (0,2) to (1,2) -- (0.5,2.5) -- cycle;
	\end{tikzpicture}
\end{latexlisting}
Wichtig sind auch die optionalen Argumente \latexargument{in} und \latexargument{out}, mit der für krumme Linien die Ein- und Ausgangswinkel angepasst werden können.
\begin{latexlisting}
	\draw[out=30,in=150] (0,0) to (3,0);
	\draw[out=30,in=150] (0,0) to (6,0);
	\draw[out=30,in=150] (0,0) to (9,0);
\end{latexlisting}
Schließlich existiert noch die Möglichkeit, Linien mit Pfeilspitzen zu versehen.
\begin{latexlisting}
	\draw[->] (0,2) to (3,2);
	\draw[|->] (0,1) to (2,1);
	\draw[<->] (0,0) to (1,0);
\end{latexlisting}
Für eine vollständige Liste aller optionalen Argument konsultiere man \url{tikz.dev}.

Auch wenn sich die meisten Formen aus Linien und Kurven zusammensetzen lassen, ist es doch oft praktischer, einige Formen direkt zu zeichnen.
Mit \latexargument{rectangle} beispielsweise lässt sich ein Rechteck zwischen zwei Ecken zeichnen.
\begin{latexlisting}
	\draw (0,0) rectangle (3,2);
\end{latexlisting}
Verwandt dist \latexargument{grid}, das ein ganzes Linienraster zeichnet -- dies ist besonders praktisch, um Hilfslinien einzufügen und sich im Bild zu orientieren. Ist das Bild fertig, kann die Zeile dann auskommentiert werden.
\begin{latexlisting}
	\draw (0,0) grid[help lines, step=0.5cm] (8,5);
\end{latexlisting}
Die \latexargument{help lines}-Option stellt hier die Liniendicke auf das Minimum ein, \latexargument{step} konfiguriert das Intervall zwischen den Rasterlinien -- wird es nicht angegeben, ist der Standard das auf häufigsten nützliche \latexargument{1cm}.
Mit Linien etwas schwerer nachzubilden sind Kreise und Ellipsen.
Hier ist das zweite \enquote{Argument} kein Punkt, sondern der Radius bzw. die Halbachsen.
\begin{latexlisting}
	\draw (0,0) circle (1cm);
	\draw (0,0) ellipse (5cm and 3cm);
\end{latexlisting}
Für Funktionen existiert schließlich noch die \latexargument{plot}-Funktion.
\begin{latexlisting}
	\draw[<->] (0,2.2) to (0,0) to (4,0);
	\draw[thin,blue] plot[domain=0:3.8] (\x,{2/(1+pow(\x,2))});
\end{latexlisting}
Beim Plotten ist zu beachten, dass die Laufvariable \latexcommand{x} einen Backslash benötigt, aber arithmetische Operationen wie \latexargument{pow} nicht (genauso wie bei \latexpackage{fpeval}).
Der zu evaluierende Ausdruck muss außerdem in geschweiften Klammern stehen.
Mit \latexargument{domain} wird der Defintionsbereich angepasst.

Für die letztgenannten Plots ist es eher selten nützlich, aber für alle anderen Formen geschieht es oft, dass sie nicht nur aus Linien bestehen sondern auch gefüllt sein sollen.
Hierzu existiert die Alternative \latexcommand{fill} zu \latexcommand{draw}.
\begin{latexlisting}
	\fill (0,0) rectangle (3,2);
\end{latexlisting}
Auch hier kann die Farbe angepasst werden:
\begin{latexlisting}
	\fill[color=red] (0,0) rectangle (3,2);
\end{latexlisting}
Wir können \latexcommand{fill} und \latexcommand{draw} beliebig kombinieren.
Die vier folgenden Zeilen erzeugen dasselbe Rechteck.
\begin{latexlisting}
	\fill[color=red, draw=black] (0,9) rectangle (3,11);
	\draw[fill=red, color=black] (0,6) rectangle (3,8);
	\fill[fill=red, draw=black] (0,3) rectangle (3,5);
	\draw[fill=red, draw=black] (0,0) rectangle (3,2);
\end{latexlisting}
Man beachte, dass \latexcommand{draw} die Liniendicke zur Formgröße hinzufügt.
Die beiden folgenden Zeilen erzeugen also \emph{nicht} dasselbe Rechteck.
\begin{latexlisting}
	\fill[draw=black] (0,0) rectangle (0,2);
	\fill (4,0) rectangle (4,2);
\end{latexlisting}
Dies sieht man besser mit stärkerer Linienstärke.
\begin{latexlisting}
	\fill[draw=black,ulta thick] (0,0) rectangle (0,2);
	\fill (4,0) rectangle (4,2);
\end{latexlisting}
Wem diese Füllungen etwas zu stark sind, der kann sie halbtransparent machen.
\begin{latexlisting}
	\fill[red,opacity=0.5] (360:0.5cm) circle (0.6cm);
	\fill[blue,opacity=0.5] (120:0.5cm) circle (0.6cm);
	\fill[green,opacity=0.5] (240:0.5cm) circle (0.6cm);
\end{latexlisting}
Dieses Beispiel demonstriert auch Polarkoordinaten, für die wir in diesem Skript leider keinen Raum mehr haben.
Ebenfalls nicht erklärt sind \emph{transparency groups}, die verwendet werden können, um komplexere Formen einheitlich transparent zu machen.
Siehe für beides \url{tikz.dev}.
Als Alternative zur Transparenz sind auch Musterfüllungen mit der \latexpackage{patterns}-Library möglich.
\begin{latexlisting}
	\fill[pattern=dots,pattern color=blue] (0,0) rectangle (1,1);
	\fill[draw=black, pattern=north west lines, pattern color=green] (2,0) -- (2.577,1) -- (3.154,0) -- cycle;
\end{latexlisting}
Die verfügbaren Patterns lassen sich im Internet nachlesen.

\subsection{Nodes}
Möchte man seine Diagramm beschriften, verwendet man dazu in TikZ sogenannte \emph{nodes}.
\begin{latexlisting}
	\draw[|-|] (0,0) -- (2,0);
	\node at (1,0.2) {\qty{2}{cm}};
\end{latexlisting}
Der gewünschte Text wird innerhalb der hinteren geschweiften Klammern platziert.
Hier sind alle gewöhnlichen \LaTeX{}-Commands und Konstrukte, insbesondere die Verwendung des Mathematik-Modus, erlaubt.
Mit den optionalen Argumenten \latexargument{above}, \latexargument{below}, \latexargument{left}, \latexargument{right} und Kombinationen können Nodes genauer positioniert werden.
\begin{latexlisting}
	\fill (0,0) circle (0.5mm);
	\fill (2,0) circle (0.5mm);
	\draw (0,0) -- (2,0);
	\node[left] at (0,0) {$P_1$};
	\node[above right] at (2,0) {$P_2$};
\end{latexlisting}
Längere Texte können mit \latexcommand{\textbackslash} gebrochen werden.
In diesem Fall ist die Angabe eines \latexargument{align}-Parameters (\latexargument{left},\latexargument{center},\latexargument{right}) nötig.
\begin{latexlisting}
	\fill (0,0) circle (0.5mm);
	\fill (2,0) circle (0.5mm);
	\draw (0,0) -- (2,0);
	\node[left] at (0,0) {$P_1$};
	\node[below,align=center] at (2,0) {Dieser Punkt ist\\nicht $P_2$.};
\end{latexlisting}
Nodes können auch wie alle anderen Formen gefüllt oder umrandet werden.
\begin{latexlisting}
	\fill (0,0) circle (0.5mm);
	\fill (2,0) circle (0.5mm);
	\draw (0,0) -- (2,0);
	\node[left, draw=black] at (0,0) {$P_1$};
\end{latexlisting}

\subsection{Ausblick}
Die hier aufgeführten Features sind nur eine kleine Teilmenge von all dem, was in TikZ möglich ist.
Auch wenn dieses Skript keinen Raum für eine vollständige Ausführung hat, nennen wir im Folgenden einige dieser Features -- bei Interesse kann man sich dann unter \url{tikz.dev} oder allgemein im Internet weiter informieren.
\begin{itemize}
	\item \emph{Scopes} erlauben die Gruppierung von Formen, gemeinsame Verschiebung, Anpassung und Teilen von optionalen Argumenten.
	\item \emph{Clips} erlauben das nur teilweise Zeichnen von Formen, z.B. wenn der Schnitt zwischen zwei Formen eingefärbt werden soll.
	\item \emph{Coordinates} erlauben das Wiederverwenden von Punkten.
	\item Neben dem klassischen kartesischen Koordinatensystem erlaubt TikZ auch die Verwendung von Polarkoordinaten, xyz-Koordinaten, node-Koordinaten, baryzentrischen Koordinaten und relativen Koordinaten.
	\item \emph{Kommutative Diagramme} können mit \latexenvironment{tikz-cd} erstellt werden.
	\item \emph{Fill Rules} konfigurieren die Füllung von Formen.
	\item \emph{Transparency Groups} machen auch komplexe, mehrteilige Formen konsistent transparent.
	\item Mit \latexpackage{pgf} können auch 3-dimensionale Objekte geplottet werden.
\end{itemize}


	\backmatter

	\printbibliography
	
\end{document}
	
